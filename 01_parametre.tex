\section Elektrické parametre vonkajších vedení

Chovanie a vlastnosti vonkajších vedení pri prenose elektrickej energie je možné
popísať pomocou štyroch základných -- primárnych -- parametrov:
\beginitemize
	\item rezistancie $R_1$ ($\rm \Omega$/km),
	\item indukčnosti $L_1$ (H/km),
	\item konduktancie $G_1$ (S/km),
	\item kapacity $C_1$ (F/km).
\enditemize
Po zohľadnení frekvencie a zavedení uhlovej frekvencie $\omega$
respektíve frekvencie $f$ používame aj ďalšie parametre -- sekundárne:
\beginitemize
	\item indukčná reaktancia $X_1 = \omega L_1 = 2{\rm\pi} f L_1$,
	\item kapacitná susceptancia $B_1 = \omega C_1 = 2{\rm\pi} f C_1$,
	\item pozdĺžna impedancia $Z_{\rm l1} = R_1 + \j X_1$,
	\item priečna admitancia $Y_{\rm q1} = G_1 + \j B_1$,
	\item vlnová impedancia $Z_{\rm V1} = \sqrt{Z_{\rm l1}/Y_{\rm q1}}$
	($\rm \Omega$)
	\item konštanta prenosu $\gamma = \sqrt{\vphantom{/}Z_{\rm l1} \cdot Y_{\rm q1}}$
	(1/km)
\enditemize

Ako vodiče používame najčastejšie drôty, pásy a laná. Základným
materiálom je meď, hliník a v~prípade lán sa často používa kombinácia
hliníku s~oceľou (AlFe laná).

%%%%%%%%%%%%%%%%%%%%%%%%%%%%%%%%%%%%%%%%%%%%%%%%%%%%%%%%%%%%%%%%%%%%%%%%%%%%%%%%%%%%%

\subsection Rezistancia -- činný odpor

V~prípade jednosmerného prenosu uvádzame hodnotu
$$
R_{\rm S01} = {\rho_0 \over S} \quad (\rm \Omega/m; \Omega
\cdot m, m^2)
$$
kde $\rho_0$ je merná rezistivita materiálu vodiča a $S$ je jeho
prierez.

Vplyv materiálu na rezistanciu je vyjadrený hodnotou mernej rezistivity
uvedenej v~\citetable[rho]. V~prípade dobrých vodičov narastá nelineárne
so zväčšovaním obsahu nečistôt a v~prípade feromagnetických materiálov
je určovaná experimentálne.
\tablelabel[label=rho][caption=Hodnoty $\rho_0$ pre základné materiály] 
\centertable{
\begintable
\begintableformat
\center " \center " \center " \center
\endtableformat
\-
\br{\:|} materiál | Cu | Al | Fe (oceľ) \er{|}
\-
\br{\:|} $\rho_0$ ($\rm \Omega\cdot m$) |$1{,}78 \times 10^{-8}$ |  $2{,}87 \times
10^{-8}$ | $20 \times 10^{-8}$ \er{|}
\-
\endtable
}

Vplyv teploty, prípadne oteplenia, rešpektujeme zavedením teplotného
koeficientu odporu
$$
k_T = 1 + \alpha(T-T_0) \quad (\rm\hbox{--}; 1/K, K)
$$
a koeficientu $\alpha$, ktorý vyjadruje tepelné vlastnosti materiálu.
\tablelabel[label=alfa][caption=Hodnoty $\alpha$ pre základné materiály] 
\centertable{
\begintable
\begintableformat
\center " \center " \center " \center
\endtableformat
\-
\br{\:|} materiál | Cu | Al | Fe (oceľ) \er{|}
\-
\br{\:|} $\alpha$ ($\rm K^{-1}$) | $3{,}93 \times 10^{-3}$ |  $4{,}03 \times
10^{-3}$ | $4{,}5 \times 10^{-3}$ \er{|}
\-
\endtable
}

V~prípade, že vedením prechádza striedavý prúd, uplatní sa povrchový
efekt a je zmenšovaný efektívny prierez vodiča. Rozloženie hustoty
striedavého prúdu po priereze nie je rovnomerné. Pre prípad AlFe lana
s~väčším počtom vrstiev je možné použiť vzťah pre duté laná, pretože
oceľovým vnútorným lanom preteká len 2\az3\,\% ceľkového prúdu. Nárast
odporu potom vyjadruje koeficient
$$
k_{\rm s} = 1 + 0{,}0375 \times 10^{-12} \cdot \left[{(r_2 - r_1) f
\over r_2 \cdot R_{\rm S01}}\right]^2 \quad (\rm\hbox{--}; m, m, Hz, m,
\Omega/m)
$$
v~ktorom $r_2$ a $r_1$ je vonkajší a vnútorný polomer Al vrstvy.

Ďalej má na hodnotu odporu vplyv krútenia lán po skrutkovici -- plný
vodič rovnakej osovej dĺžky by mal menšiu skutočnú dĺžku. Pri tomto
zohráva úlohu aj to, že na jednotlivých vodičoch v~zväzku sa často
vytvára tenká oxidová vrstva a prúd tečie vo všetkých vodičoch
paralelne, sledujúc dráhu skrutkovice. Tento koeficient, označený
$k_{l}$, býva zahrnutý v~katalógu.

Pri zavesení vodiča na stožiar dochádza k~jeho priehybu podľa reťazovky.
Jeho skutočná dĺžka je teda väčšia podľa vzťahu
$$
l_{\rm h} = 2c \sinh{d \over 2c} \quad ({\rm m; m, m})
$$
v~ktorom $c$ je parameter reťazovky a $d$ je vzdialenosť medzi stožiarmi
meraná po úsečke. Činiteľ zväčšenia odporu je daný vzťahom
$$
k_{\rm p} = {l_{\rm h} \over d} \quad ({\rm m; m, m})
$$
ktorý v~praxi býva menší než 1,03.

Ako ďalší parameter zväčšujúci odpor môžme uvažovať odchýlku skutočného prierezu od menovitého.

Celkový odpor teda môžme vyjadriť pomocou vzťahu
\zvyrazni
$$
R_1 = R_{\rm S01} \cdot k_T \cdot k_{\rm s} \cdot k_l \cdot
k_{\rm p} \quad (\rm \Omega m^{-1}; \Omega m^{-1}, \hbox{--}).
$$

%%%%%%%%%%%%%%%%%%%%%%%%%%%%%%%%%%%%%%%%%%%%%%%%%%%%%%%%%%%%%%%%%%%%%%%%%%%%%%%%%%%%%

\subsection Konduktancia -- zvod

Konduktancia predstavuje priečne straty činného výkonu. Na zaťažení
závisí málo, viac na ňu vplýva napätie a poveternostné vplyvy okolia.
Je spôsobovaná stratami cez izolátory a najmä korónou. Straty korónou
rešpektujeme až od napätí 100\,kV.

Ak uvážime, že cez konduktanciu $G_1$ prechádza prúd $I_{\rm s}$ a sieť
má fázové napätie $U_{\rm f}$, respektíve združené napätie $U$, budú straty
výkonu zvodom pre tri fázy a 1\,km dĺžky
\zvyrazni
$$
P_{\rm z} = 3 U_{\rm f} I_{\rm s} = 3 U_{\rm f} G_1 U_{\rm f} = 3 G_1
U_{\rm f}^2 =  G_1 U^2 \quad (\rm W\cdot km^{-1}).
$$

\tablelabel[label=zvod][caption=Približné hodnoty konduktancie pre základné
napäťové hladiny] 
\centertable{
\begintable
\begintableformat
\center " \center " \center " \center " \center " \center
\endtableformat
\-
\br{\:|} menovité napätie $U_{\rm n}$ (kV) | 110 | 220 | 380 | 500 \er{|}
\-
\br{\:|} konduktancia $G_1$ (S/km) | $(3{,}6 \div 5) \times 10^{-8}$ | $(2{,}5 \div
3{,}6) \times 10^{-8}$ | $(1{,}4 \div 2) \times 10^{-8}$ | $2{,}5 \times 10^{-8}$ \er{|}
\-
\endtable
}

%%%%%%%%%%%%%%%%%%%%%%%%%%%%%%%%%%%%%%%%%%%%%%%%%%%%%%%%%%%%%%%%%%%%%%%%%%%%%%%%%%%%%

\subsection Indukčnosť a pozdĺžna impedancia

Cieľom je vytvorenie matematického modelu, ktorým sa sústava vodičov
rovnobežných so zemou a navzájom nahradí rovnocennou sústavou dvojíc
vodičov (skutočného a jemu zodpovedajúceho fiktívneho) navzájom
rovnobežných. Model zjednoduší výpočty.

\subsection * Indkučnosť a impedancia v~slučke

Uvažujme slučku tvorenú dvoma rovnakými, priamymi a rovnobežnými
vodičmi kruhového prierezu s~polomerom $r$, s~rovnakým konštrukčným
vyhotovením.

\beginfigure
	\filename={img/indukcnost-slucka.png}
	\label={indukcnost-slucka}
	\caption={Slučka, k~odvodeniu indukčnosti}
\endfigure

Predpokladajme, že platí $r \ll d \ll l$ a $d_{\rm kk'} = d$, kde $d$ je
kolmá vzdialenosť ich osí a $l$ ich dĺžka. Pre fázory prúdu platí
$\faz I_{k} = - \faz I_{k'}$. Uvažujme pomery na jednotku dĺžky vnútri
slučky v~značnej vzdialenosti od koncov oboch vodičov, tak, aby sa vplyv
deformácie magnetického poľa vplyvom koncov vodičov neuplatnil.

Vnútorná indukčnosť, rešpektujúca magnetický tok na jednotku dĺžky
vnútri vodiča, je
$$
L_{{\rm i}k} = {\mu_0 \mu_{\rm rv} \over 8 {\rm \pi}} \alpha
$$
kde $\mu_{\rm rv}$ je relatívna permeabilita materiálu vodiča a $\alpha$
rešpektuje nerovnomerné rozdelenie prúdu po priereze.

Vonkajšia indukčnosť na jednotku dĺžky je
$$
L_{{\rm e}k} = {\mu_0 \mu_{\rm r} \over 2 {\rm \pi}} \ln{R \over r}
$$
a rešpektuje magnetický tok mimo vodiča. Polomer $R \gg d$ má zatiaľ
neurčenú, ale konečnú hodnotu.

Časti magnetického toku druhého vodiča $k'$, ktorá je v~zábere
s~uvažovaným vodičom $k$, zodpovedá indukčnosť
$$
L_{kk'} = {\mu_0 \mu_{\rm r} \over 2{\rm\pi}} \ln{R \over d}.
$$

Pre celkové napätie $\faz U_{{\rm i}k}$, indukované v~uvažovanom vodiči
na jednotku dĺžky, platí
$$
-\faz U_{{\rm i}k} = \j \omega \left[ ( L_{{\rm i}k} + L_{{\rm e}k}) \faz
I_k + L_{kk'} \faz I_{k'} \right] = \j \omega (L_{{\rm i}k} + L_{{\rm
e}k} - L_{kk'}) \faz I_k = \j \omega L_{k{\rm v}} \faz I_k
$$
kde $L_{k\rm v}$ je vlastná indukčnosť jedného vodiča slučky na jednotku
dĺžky. Po dosadení z~predchádzajúcich rovníc pre ňu dostávame vzťah
$$
L_{k{\rm v}} = {\mu_0 \mu_{\rm rv} \over 8{\rm \pi}} \alpha + {\mu_0
\mu_{\rm r} \over 2{\rm\pi}} \ln{R \over r} - {\mu_0 \mu_{\rm r} \over
2{\rm\pi}}\ln{R \over d}
$$
a po úpravách dostaneme vzťah
$$
L_{k{\rm v}} = 0{,}05 \mu_{\rm rv} \alpha + 0{,}46 \log{d \over r}.
$$

Po zavedení činiteľa $\xi$ sa vzťah zjednoduší. Hodnoty $\xi$ sú pritom
pre rôzne typy vodičov uvedené v~\citetable[hodnotyxi].
$$
\eqalign{
\xi &= 10^{-(0{,}05 \mu_{\rm rv} \alpha) / 0{,}46}\cr
L_{k\rm v} &= 0{,}46 \log {d \over \xi r}\cr
}
$$

\tablelabel[label=hodnotyxi][caption=Hodnoty činiteľa $\xi$ pre rôzne
typy vodičov]
\centertable{
\begintable
\begintableformat
\left " \left
\endtableformat
\-
\br{\:|} \hfil Typ vodiča | \hfil $\xi$ \er{|}
\-
\br{\:|} masívny vodič kruhového prierezu | 0,779 \er{|}
\noalign{
	\global\tmpa={jednomateriálové lano\quad}
	\global\tmpb={čiastkových vodičov}
}
\br{\:|} \the\tmpa ~~7 \the\tmpb | 0{,}726 \er{|}
\br{\:|} \phantom{\the\tmpa}~19 \phantom{\the\tmpb} | 0,758 \er{|}
\br{\:|} \phantom{\the\tmpa}~37 \phantom{\the\tmpb} | 0,768 \er{|}
\br{\:|} \phantom{\the\tmpa}~61 \phantom{\the\tmpb} | 0,772 \er{|}
\br{\:|} \phantom{\the\tmpa}~91 \phantom{\the\tmpb} | 0,774 \er{|}
\br{\:|} \phantom{\the\tmpa}127 \phantom{\the\tmpb} | 0,776 \er{|}
\noalign{
	\global\tmpa={laná AlFe\quad}
	\global\tmpb={čiastkových vodičov v\ }
	\global\tmpc={vrstvách}
}
\br{\:|} \the\tmpa 26 \the\tmpb 2 \the\tmpc | 0,809 \er{|}
\br{\:|} \phantom{\the\tmpa}30 \phantom{\the\tmpb}2 \phantom{\the\tmpc}
| 0,826 \er{|}
\br{\:|} \phantom{\the\tmpa}54 \phantom{\the\tmpb}3 \phantom{\the\tmpc}
| 0,810 \er{|}
\br{\:|} \phantom{\the\tmpa} s~jednou vrstvou Al vodičov | 0,55 až 0,7 \er{|}
\br{\:|} vodič obdĺžnikového profilu so stranami $a$, $b$ | $0{,}2235 (a
+ b)$ \er{|}
\-
\endtable
}

Impedancia jedného vodiča v~slučke dvoch rovnobežných vodičov na
jednotku dĺžky bude
$$
\faz Z_{k\rm v} = R_k + \j\omega L_{{k\rm v}} = R_k + \j\omega 0{,}46
\times 10^{-6} \log{d \over \xi r} \quad ({\rm \Omega \cdot m^{-1}})
$$
kde $R_k$ je rezistancia vodiča na jednotku dĺžky.

\subsection * Zem ako vodič stacionárneho striedavého prúdu, Rüdenbergova
koncepcia

V~priestore pozdĺž vedenia, v~dostatočnej vzdialenosti od elektród
(30\,m a viac), prechádza ustálený jednosmerný prúd tak širokým priečnym
prierezom, že výsledný odpor cesty zemou v~tejto časti je nepatrný a
technicky zanedbateľný. V~prípade striedavého prúdu sa v~tej istej
oblasti uplatní vplyv magnetického poľa. V~dôsledku javu analogického
s~povrchovým javom je hustota striedavého prúdu v~zemi nerovnomerná.
Striedavý prúd v~zemi sleduje presne trasu vodičov nad zemou, jeho
najväčšia hustota je priamo pod vedením a rýchlo klesá ako do strán, tak
do hĺbky.

Rüdenbergova koncepcia vychádza z~nasledovných predpokladov: Rezistivita
zeme je koštantná a má konečnú veľkosť, skutočné usporiadanie zeme a
vodiča vo výške $h$ nad zemou sa nahrádza v~častiach, kde sa neuplatní
vplyv elektród, modelom na \citefigure[vodic-zem]. Vodič sa pomyselne
kladie na zem a zemina sa v~polvalci s~polomerom $h$ vypúšťa. Vodič je
v~ose polvalca. Prúd vo vodiči je sínusový bez vyšších harmonických
zložiek s~konštantnou amplitúdou a to isté platí aj pre intenzitu
magnetického poľa a hustotu prúdu v~zemi.

\beginfigure
	\filename={img/indukcnost-vodic-v-zemi.png}
	\label={vodic-zem}
	\caption={Náhrada skutočného vedenia vodičom na povrchu zeme}
\endfigure

Podľa 1. Maxwellovej rovnice, použitej na kružnice s~polomermi $x$ a $x
+ \d x$ (ktoré sú približne siločiarami) platí
$$
\displaylines{
2{\rm \pi} x \faz H =
\faz I_{\rm v} - \int\limits_h^x \faz J {\rm \pi} x \,\d x,\cr
2{\rm\pi} (x + \d x)(\faz H + \d \faz H) =
\faz I_{\rm v} - \int\limits_h^{x + \d x} \faz J {\rm\pi} x \,\d x.
}
$$
Po odčítaní rovníc, zanedbaní malých veličín druhého rádu a úprave
dostaneme
$$
{\faz H \over x} + {\d \faz H \over \d x} = - {1 \over 2} \faz J.
\equation[1mr-upr]
$$

Podľa Ohmovho zákona pre prúdové vlákno platí vzťah
$$
\faz E = \rho \faz J \longrightarrow {\d \faz E \over \d x} = \rho {\d \faz J \over
\d x}.
\equation[ohmz-pv]
$$

Podľa 2. Maxwellovej rovnice, použitej pre elementárny úsek dĺžky $z$ a
šírky $\d x$, dostaneme vzťah
$$
\left(\faz E + {\d \faz E \over \d x} \d x \right) z - z \faz E = -\j
\omega \mu \faz H_z \d x \longrightarrow {\d \faz E \over \d x} = -\j \omega \mu
\faz H.
\equation[2mr]
$$

Porovnaním rovníc \citeequation[ohmz-pv] a \citeequation[2mr] a 
derivovaním dostávame vzťahy
$$
\displaylines{
\faz H = \j {\rho \d\faz J \over \mu \omega \d x},\cr
{\d \faz H \over \d x} = \j {\rho \d^2\/ \faz J \over \mu \omega \d
x^2}.\cr
}
$$
Ich dosadením do rovnice \citeequation[1mr-upr] dostávame vzťah
$$
{\d^2 \faz J \over \d x^2} + {1 \over x}{\d \faz J \over \d x} - \j {\mu
\omega \over 2\rho}\faz J = 0.
\equation[bes]
$$
Pre túto rovnicu platia okrajové podmienky, ktoré je možné formulovať
ako:
\beginitemize
	\item prúd v~zemi je prúdom v~slučke vodič--zem,
	\item hustota prúdu v~zemi pre $x \to \infty$ musí byť rovná nule.
\enditemize
To spĺňajú rovnice
$$
\displaylines{
\int\limits_h^{\infty} \faz J(x) {\rm\pi} x \,\d x = \faz I_{\rm g} = -
\faz I_{\rm v},\cr
\lim_{x \to \infty} \faz J(x) = 0.\cr
}
$$

Rovnica \citeequation[bes] pre prúdovú hustotu $\faz J(x)$ je Besselova
rovnica 2. rádu. Pri jej riešení využijeme vlastnosti Besselovych
funkcií 3. druhu definovaných pomocou rady. Členy s~vyššími mocninami
premennej je možné zanedbať a tak výpočet zjednodušiť. Presnosť, ktorú
dosiahneme, je v~technických úlohách dostatočná. Dostávame tak vzťah
$$
\faz J(x) = \left(
	{{\rm\pi}^2 f \over \rho} +
	\j {4{\rm\pi} f \over \rho}
	\ln{{0{,}178 \sqrt{\rho \times 10^7} \over x \sqrt{f\vphantom{|}} }}\right)
\times 10^{-7} \faz I_{\rm g}
$$
ktorý platí pre frekvencie $f \le 5\,\rm kHz$.

\subsection * Vlastná impedancia slučky vodič--zem

Celý magnetický tok vyvolaný prúdom v~zemi je spriahnutý tiež s~prúdom
v~elementárnej vrstve na povrchu polvalca s~polomerom $h$, kde v~prúdovom
vlákne vyvolá úbytok napätia na jednotku dĺžky. Súčasne platia vzťahy:
$$
\eqalignno{
\Delta \faz U(h) &= \rho \faz J(h) =
\left({\rm\pi}^2 f + \j 4{\rm\pi} f \ln{0{,}178 \sqrt{\rho \times 10^7}
\over h\sqrt{\vphantom{|}f}}\right) \times 10^{-7} \faz I_{\rm g},\cr
\Delta \faz U(h) &= (R_{1\rm g} + \j 2{\rm\pi} L_{1\rm g}) \faz I_{\rm
g}.&\equation[deltau]\cr
}
$$
Ich porovnaním dostaneme hodnotu rezistancie zeme na jednotku dĺžky
vyjadrenú vzťahom
$$
R_{1\rm g} = {\rm\pi}^2 f \times 10^{-4} \quad (\rm \Omega/km; Hz)
$$
ktorá je pre štandardnú frekvenciu $f = 50\,\rm Hz$ rovná $R_{1\rm g} =
0{,}0495\,\rm \Omega/km$.

Pri výpočte indukčnosti slučky uvážime vplyv magnetických tokov
spriahnutých so slučkou.

Magnetický tok vnútri vodiča nad zemou rešpektuje {\it vnútorná
indukčnosť}
$$
L_{{\rm i}k} = 0{,}46 \log{1 \over \xi},
$$
a magnetický tok vo vzduchu, kde $x \in \langle r,h\rangle$ rešpektuje
{\it vonkajšia indukčnosť}
$$
L_{{\rm e}k} = 0{,}46 \log {h \over r}.
$$
Magnetický tok v~zemi rešpektuje indukčnosť, ktorú
dostaneme porovnaním rovníc \citeequation[deltau] a po úpravách je rovná
$$
L_{1\rm g} = 0{,}46 \log {0{,}178 \sqrt{\rho \times 10^7} \over h
\sqrt{\vphantom{|} f}} = 0{,}46 \log{D_{\rm g} \over h} \quad (\rm mH/km)
$$
kde hodnota
\zvyrazni
$$
D_{\rm g} = {0{,}178 \sqrt{\rho \times 10^{7}} \over
\sqrt{\vphantom{|}f}} \quad (\rm m; \Omega \cdot m, Hz)
$$
sa nazýva hĺbka fiktívneho \uv{vodiča}, ktorý by viedol prúd
v~zemi a nahradzuje prúd v~zemi. Orientačné hodnoty $\rho$ rozdelené podľa
typu pôdy sú uvedené v~\citetable[rho-poda].

\tablelabel[label=rho-poda][caption=Orientačné hodnoty rezistivity pôdy] 
\centertable{
\begintable
\begintableformat
\left " \center
\endtableformat
\-
\br{\:|} \hfil Typ zeminy | $\rho$ ($\rm \Omega \cdot m$) \er{|}
\-
\br{\:|} rašelina | 30 \er{|}
\br{\:|} ornica a íľ | 100 \er{|}
\br{\:|} vlhký piesok | 200 až 300 \er{|}
\br{\:|} suchý štrk a piesok | 1000 až 3000 \er{|}
\br{\:|} kamenitá pôda | 3000 až 10000 \er{|}
\-
\endtable
}
Celková indukčnosť v~slučke vodič--zem na
jednotku dĺžky bude
$$
L_{kk} = L_{{\rm i}k} + L_{{\rm e}k} + L_{1{\rm g}} = 0{,}46 \log{D_{\rm
g} \over \xi r} \quad (\rm \mu H/m; \mu H/m, m, m)
$$
a reaktancia $X_{kk}$ bude rovná
$$
X_{kk} = \omega L_{kk} = \omega 0{,}46\log{D_{\rm g} \over \xi r}.
$$
Vlastná impedancia slučky vodič--zem na jednotku dĺžky bude teda 
$$
\faz Z_{kk} = R_{1k} + R_{1\rm g} + \j X_{kk} = R_{1k} + {\rm\pi}^2 f
\times 10^{-4} + \j \omega \times 10^{-3} \cdot 0{,}46 \log {D_{\rm g} \over \xi r} \quad
(\rm \Omega/km).
\equation[zkk]
$$

\subsection * Vzájomná impedancia dvoch slučiek vodič--zem

Dvojvodičové jednofázové vedenie je možné s~ohľadom na magnetické pole
ním vytvárané považovať za rovnocenné dvom vedeniam vodič--zem ako je
znázornené na \citefigure[dvojv-zem], pretože spätné prúdy v~zemi sa
navzájom kompenzujú a nedávajú výsledne žiaden účinok. To však platí len
ak je vzdialenosť medzi skutočnými vodičmi $k$, $m$ menšia alebo rovná
ich výške nad zemou:
$$
d_{km} \le h.
$$

\beginfigure
	\filename={img/impedancia-dve-smycky-zem.png}
	\label={dvojv-zem}
	\caption={Náhrada pomocou dvoch vedení vodič--zem}
\endfigure

Pretože $D_{\rm g} \gg d_{km}$, je možno vzdialenosti každého
z~fiktívnych vodičov od každého skutočného považovať za približne
rovnaké, teda
$$
d_{km'} = d_{k'm} = d_{kk'} = d_{mm'}.
$$
Potom je však výsledné elektromagnetické pôsobenie spätných prúdov vo
vodičoch $k'$, $m'$ na skutočné vodiče $k$, $m$ takmer nulové. Zámena
dvojvodičového jednofázového vedenia dvoma slučkami vodič--zem umožňuje
určiť výslednú impedanciu jedného vodiča $\faz Z_{k{\rm v}}$ ako
$$
\faz Z_{k{\rm v}} = \faz Z_{kk} - \faz Z_{km}.
$$

Po dosadení za $\faz Z_{k{\rm v}}$ a $\faz Z_{kk}$ dostaneme pre
vzájomnú impedanciu na jednotku dĺžky po úprave
$$
\faz Z_{km} = R_{1\rm g} + \j \omega \times 10^{-3} \cdot 0{,}46\log{ D_{\rm g}
\over d_{km}} \quad (\rm \Omega/km).
\equation[zkm]
$$

\subsection * Sústava $\bi n$ vodičov

Predchádzajúce úvahy rozšírime na sústavu $n$ skutočných vodičov a zem.
Vodiče sú rovnobežné navzájom aj so zemou. Príslušným matematickým modelom je
sústava $n$ dvojíc vodičov, z~ktorých je jeden skutočný a druhý fiktívny,
rešpektujúci spätný prúd zemou.

Pri bežnej frekvencii siete $f = 50\,\rm Hz$ dostávame z~rovníc
\citeequation[zkk] a \citeequation[zkm] pre {\it vlastnú impedanciu} slučky
($k$~--~$k'$) vzťah
$$
\faz Z_{kk} =
R_{1k} + R_{1\rm g} + \j 0{,}1445 \log{D_{\rm g} \over \xi r_k}
\quad({\rm \Omega/km})
$$
a pre {\it vzájomnú impedanciu} medzi slučkami ($k$~--~$k'$) a ($m$~--~$m'$)
vzťah
$$
\faz Z_{km} =
\faz Z_{mk} = R_{1\rm g} + \j 0{,}1445 \log{D_{\rm g} \over d_{km}}
\quad({\rm \Omega/km}).
$$

Výsledné pôsobenie prúdov všetkých slučiek na uvažovaný vodič $k$
poskytuje rovnica
$$
\Delta \faz U_k = \sum\limits_{m=1}^n \faz Z_{km} \faz I_m,
$$
prípadne v~maticovom zápise
$$
[\Delta \faz U] =[\faz Z] \cdot [\faz I]
$$
kde $[\Delta \faz U]$ je matica úbytkov napätí na vodičoch,
$[\faz Z]$ je symetrická matica impedancií (na diagonále
vlastné impedancie $\faz Z_{kk}$ a ostatné prvky vzájomné impedancie  $\faz
Z_{km} = \faz Z_{mk}$) a $[\faz I]$ matica prúdov v~skutočných vodičoch.

V~praxi sa definične zavádza {\it prevádzková impedancia} $\faz Z_k$, ktorá je
určená impedanciou jednej slučky, ktorá by mala rovnaký účinok ako
výsledne všetky slučky na uvažovaný vodič.
\zvyrazni
$$
\Delta \faz U_k =
\sum\limits_{m=1}^n \faz Z_{km} \faz I_m = \faz Z_k \faz I_k
\longrightarrow
\faz Z_k = {\sum\limits_{m=1}^n \faz Z_{km} \faz I_m \over \faz I_k}
$$

\subsection * Trojfázové vonkajšie vedenie bez uzemňovacích lán

\beginfigure
	\filename={img/trojfazove-bez-lan.png}
	\label={trojfazove}
	\caption={Trojfázové vedenie bez uzemňovacích lán}
\endfigure

Pre úbytky napätí vo fázach platí
$$
\hmatrix{\Delta \faz U_{\rm a} \cr
\Delta \faz U_{\rm b}\cr
\Delta \faz U_{\rm c}\cr} = 
\hmatrix{
\faz Z_{\rm aa} & \faz Z_{\rm ab} & \faz Z_{\rm ac} \cr
\faz Z_{\rm ba} & \faz Z_{\rm bb} & \faz Z_{\rm bc} \cr
\faz Z_{\rm ca} & \faz Z_{\rm cb} & \faz Z_{\rm cc} \cr
}
\hmatrix{
\faz I_{\rm a} \cr \faz I_{\rm b} \cr \faz I_{\rm c}
}.
$$

V~prípade symetrických prúdov a nesymetrického vedenia budú prevádzkové
impedancie fáz rôzne a komplexné. To spôsobí nesymetriu napätí a
predávanie činného výkonu medzi fázami elektromagnetickou väzbou bez
ďalšieho zaťažovania zdrojov.

Ak bude vedenie súmerne usporiadané, teda ak $d_{ab} = d_{ac} = d_{bc} =
d$, dostávame vzťahy
$$
\displaylines{
\faz Z_{\rm ab} =
\faz Z_{\rm bc} =
\faz Z_{\rm ac} =
\faz Z' =
R_{1\rm g} + \j 0{,}1445 \log {D_{\rm g} \over d}\cr
\faz Z_{\rm aa} =
\faz Z_{\rm bb} =
\faz Z_{\rm cc} =
\faz Z =
R_1 + R_{1\rm g} + \j 0{,}1445 \log{D_{\rm g} \over \xi r}\cr
}
$$

Pre prevádzkové impedancie dostávame rovnaký vzťah a
\zvyrazni
$$
\faz Z_1 =
R_1 + \j 0{,}1445 \log {d \over \xi r} = R_1 + \j X_1 = R_1 + \j \omega L_1
\quad(\rm \Omega/km)
$$
je prevádzková impedancia úplne súmerného trojfázového vedenia. V~tejto
rovnici sa nevyskytuje rezistancia zeme, pretože zemou výsledne
neprechádza žiaden prúd.

\subsection * Transpozícia trojfázových vedení

Transpozícia, kríženie vodičov, je spôsob, ako môžme dosiahnuť symetriu
vedenia. Po dĺžke úplného cyklu transpozície sa dostane každý vodič do
svojej pôvodnej polohy. Transpozíciou strácajú jednotlivé fázy svoje
výsadné postavenia.

\beginfigure
	\filename={img/transpozicia.png}
	\label={transpozicia}
	\caption={Transpozícia trojfázového vedenia}
\endfigure

Pre úbytky napätí vo fázach na 1\,km dĺžky platí
$$
\hmatrix{
\Delta \faz U_{\rm a} \cr
\Delta \faz U_{\rm b} \cr
\Delta \faz U_{\rm c} \cr
} = 
{1\over 3}
\left\{
\hmatrix{
\faz Z_{11} & \faz Z_{12} & \faz Z_{13} \cr
\faz Z_{12} & \faz Z_{22} & \faz Z_{23} \cr
\faz Z_{13} & \faz Z_{23} & \faz Z_{33} \cr
} + 
\hmatrix{
\faz Z_{33} & \faz Z_{13} & \faz Z_{23} \cr
\faz Z_{13} & \faz Z_{11} & \faz Z_{12} \cr
\faz Z_{23} & \faz Z_{12} & \faz Z_{22} \cr
}
+
\hmatrix{
\faz Z_{22} & \faz Z_{23} & \faz Z_{12} \cr
\faz Z_{23} & \faz Z_{33} & \faz Z_{13} \cr
\faz Z_{12} & \faz Z_{13} & \faz Z_{11} \cr
}
\right\}
\hmatrix{
\faz I_{\rm a} \cr
\faz I_{\rm b} \cr
\faz I_{\rm c} \cr
}
$$
\zvyrazni
$$
\hmatrix{
\Delta \faz U_{\rm a} \cr
\Delta \faz U_{\rm b} \cr
\Delta \faz U_{\rm c} \cr
} = 
\hmatrix{
\faz Z\hfill & \faz Z' & \faz Z' \cr
\faz Z' & \faz Z\hfill & \faz Z' \cr
\faz Z' & \faz Z' & \faz Z\hfill \cr
}
\hmatrix{
\faz I_{\rm a} \cr
\faz I_{\rm b} \cr
\faz I_{\rm c} \cr
}
$$
kde pre impedancie $\faz Z$ a $\faz Z'$ platí
\zvyrazni
$$
\eqalign{
\faz Z &= {1\over 3} (\faz Z_{11} + \faz Z_{22} + \faz Z_{33}) =
R_1 + R_{1\rm g} + \j 0{,}1445 \log{D_{\rm g} \over \xi r}\cr
\faz Z' &= {1\over 3} (\faz Z_{12} + \faz Z_{13} + \faz Z_{23}) =
R_{1\rm g} + \j 0{,}1445 \log {D_{\rm g} \over d}.\cr
}
$$
Skutočné vzdialenosti medzi vodičmi nahrádzame vzťahom $d = \root 3 \of
{d_{12} d_{13} d_{23}}$.

Transpozícia sa robí iba na tých vedeniach, kde je to naozaj treba,
pretože transpozičné stožiare sú nákladnejšie a poruchovejšie.

\subsection * Dvojité vedenie s~dvoma uzemňovacími lanami

Tvoria ich dva súbory trojfázových vedení na spoločných stožiaroch. Na
\citefigure[trojfazove-s-lanami] je znázornené jedno z~možných
priestorových usporiadaní vodičov.

\beginfigure
	\filename={img/trojfazove-s-lanami.png}
	\label={trojfazove-s-lanami}
	\caption={Dvojité vedenie s~uzemňovacími lanami}
\endfigure

Uzemňovacie laná, umiestnené vo vrcholoch stožiarov, majú niekoľko úloh:
\beginitemize
	\item Znižujú počet priamych atmosférických výbojov do fázových vodičov
				vedenia.
	\item Znižujú indukované elektrostatické prepätia.
	\item Znižujú krokové a dotykové napätia pri nesymetrických zemných
				skratoch.
	\item Znižujú vplyv na vedenia a zariadenia v~súbehu, a to tým lepšie,
				čím väčšia je vodivosť lana (obvykle sú z~Fe).
\enditemize

Dvojité vedenie môžme popísať rovnicami
$$
\displaylines{
\hmatrix{
\Delta \faz U_{\rm a} \cr
\Delta \faz U_{\rm b} \cr
\Delta \faz U_{\rm c} \cr
\Delta \faz U_{\rm A} \cr
\Delta \faz U_{\rm B} \cr
\Delta \faz U_{\rm C} \cr
\Delta \faz U_{\rm z_1} \cr
\Delta \faz U_{\rm z_2} \cr
} = 
\hmatrix{
\faz Z_{\rm aa} & \faz Z_{\rm ab} & \faz Z_{\rm ac} & \faz Z_{\rm aA} & \faz Z_{\rm aB} & \faz Z_{\rm aC} & \faz Z_{\rm az_1} & \faz Z_{\rm az_2} \cr 
\faz Z_{\rm ba} & \faz Z_{\rm bb} & \faz Z_{\rm bc} & \faz Z_{\rm bA} & \faz Z_{\rm bB} & \faz Z_{\rm bC} & \faz Z_{\rm bz_1} & \faz Z_{\rm bz_2} \cr 
\faz Z_{\rm ca} & \faz Z_{\rm cb} & \faz Z_{\rm cc} & \faz Z_{\rm cA} & \faz Z_{\rm cB} & \faz Z_{\rm cC} & \faz Z_{\rm cz_1} & \faz Z_{\rm cz_2} \cr 
\faz Z_{\rm Aa} & \faz Z_{\rm Ab} & \faz Z_{\rm Ac} & \faz Z_{\rm AA} & \faz Z_{\rm AB} & \faz Z_{\rm AC} & \faz Z_{\rm Az_1} & \faz Z_{\rm Az_2} \cr 
\faz Z_{\rm Ba} & \faz Z_{\rm Bb} & \faz Z_{\rm Bc} & \faz Z_{\rm BA} & \faz Z_{\rm BB} & \faz Z_{\rm BC} & \faz Z_{\rm Bz_1} & \faz Z_{\rm Bz_2} \cr 
\faz Z_{\rm Ca} & \faz Z_{\rm Cb} & \faz Z_{\rm Cc} & \faz Z_{\rm CA} & \faz Z_{\rm CB} & \faz Z_{\rm CC} & \faz Z_{\rm Cz_1} & \faz Z_{\rm cz_2} \cr 
\faz Z_{\rm z_1a} & \faz Z_{\rm z_1b} & \faz Z_{\rm z_1c} & \faz Z_{\rm z_1A} & \faz Z_{\rm z_1B} & \faz Z_{\rm z_1C} & \faz Z_{\rm z_1z_1} & \faz Z_{\rm z_1z_2} \cr 
\faz Z_{\rm z_2a} & \faz Z_{\rm z_2b} & \faz Z_{\rm z_2c} & \faz Z_{\rm z_2A} & \faz Z_{\rm z_2B} & \faz Z_{\rm z_2C} & \faz Z_{\rm z_2z_1} & \faz Z_{\rm z_2z_2} \cr 
}
\hmatrix{
\faz I_{\rm a} \cr
\faz I_{\rm b} \cr
\faz I_{\rm c} \cr
\faz I_{\rm A} \cr
\faz I_{\rm B} \cr
\faz I_{\rm C} \cr
\faz I_{\rm z_1} \cr
\faz I_{\rm z_2} \cr
}\cr
\noalign{\smallskip}
%
\hmatrix{
[\Delta \faz U_{\rm v}]\cr
[\Delta \faz U_{\rm V}]\cr
[\Delta \faz U_{\rm z}]\cr
}
=
\hmatrix{
[\faz Z_{\rm vv}] & [\faz Z_{\rm vV}] & [\faz Z_{\rm vz}] \cr
[\faz Z_{\rm Vv}] & [\faz Z_{\rm VV}] & [\faz Z_{\rm Vz}] \cr
[\faz Z_{\rm zv}] & [\faz Z_{\rm zV}] & [\faz Z_{\rm zz}] \cr
}
\hmatrix{
[\faz I_{\rm v}] \cr
[\faz I_{\rm V}] \cr
[\faz I_{\rm z}] \cr
}\cr
}
$$

V~prípade uzemňovacích lán sa obvykle prijíma predpoklad, že sú
uzemnené spojito, teda v~každom mieste pozdĺž vedenia, nie len na
stožiaroch. Potom platí 
$$
\Delta \faz U_{\rm z_1} = 0 = \Delta \faz U_{\rm z_2} \longrightarrow
[\Delta \faz U_{\rm z}] = [0]
$$
a môžme písať rovnice
$$
\eqalign{
[\Delta \faz U_{\rm v}]       &= [\faz Z_{\rm vv}] [\faz I_{\rm v}] + [\faz Z_{\rm vV}] [\faz I_{\rm V}] + [\faz Z_{\rm vz}] [\faz I_{\rm z}]\cr
[\Delta \faz U_{\rm V}]       &= [\faz Z_{\rm Vv}] [\faz I_{\rm v}] + [\faz Z_{\rm VV}] [\faz I_{\rm V}] + [\faz Z_{\rm Vz}] [\faz I_{\rm z}]\cr
[0] = [\Delta \faz U_{\rm z}] &= [\faz Z_{\rm zv}] [\faz I_{\rm v}] + [\faz Z_{\rm zV}] [\faz I_{\rm V}] + [\faz Z_{\rm zz}] [\faz I_{\rm z}]\cr
}
$$
pomocou ktorých môžme určiť prúdy v~uzemňovacích lanách ako
$$
[\faz I_{\rm z}] = -[\faz Z_{\rm zz}]^{-1} \cdot
( [\faz Z_{\rm zv}] [\faz I_{\rm v}] + [\faz Z_{\rm zV}][\faz I_{\rm V}]). 
$$

Môžme vytvoriť modifikované vedenie -- pomyselné vedenie bez
uzemňovacích lán, ktoré by sa chovalo rovnako ako skutočné vedenie
s~uzemňovacími lanami. Tento krok je potrebný pri prevode impedancií do
súmerných zložiek. Po dosadení za $[\faz I_{\rm z}]$ dostávame pre
modifikované vedenie vzťahy
$$
\displaylines{
[\Delta \faz U_{\rm v}] =
\left([\faz Z_{\rm vv}] - [\faz Z_{\rm vz}] [\faz Z_{\rm zz}]^{-1} [\faz Z_{\rm zv}]\right) [\faz I_{\rm v}]
+ \left([\faz Z_{\rm vV}] - [\faz Z_{\rm vz}] [\faz Z_{\rm zz}]^{-1} [\faz Z_{\rm zV}]\right) [\faz I_{\rm V}]\cr
[\Delta \faz U_{\rm V}] =
\left([\faz Z_{\rm Vv}] - [\faz Z_{\rm Vz}] [\faz Z_{\rm zz}]^{-1} [\faz Z_{\rm zv}]\right) [\faz I_{\rm v}]
+ \left([\faz Z_{\rm VV}] - [\faz Z_{\rm Vz}] [\faz Z_{\rm zz}]^{-1} [\faz Z_{\rm zV}]\right) [\faz I_{\rm V}]\cr
}
$$

%%%%%%%%%%%%%%%%%%%%%%%%%%%%%%%%%%%%%%%%%%%%%%%%%%%%%%%%%%%%%%%%%%%%%%%%%%%%%%%%%%%%%

\subsection Kapacity

Pri výpočte kapacít vychádzame zo sústavy rovníc $[\faz U] = [\delta]
[\faz Q]$ a $[\faz Q] = [k] [\faz U]$, kde $[\faz U]$ je matica
potenciálov vodičov, $[\faz Q]$ je matica lineárnych hustôt nábojov,
$[\delta]$ je matica potenciálových súčiniteľov a $[k]$ je matica
kapacitných súčiniteľov. Vodič a zem, prípadne dva
vodiče tvoria elektródy kondenzátorov. Dielektrikom je vzduch
s~relatívnou permitivitou $\epsilon_{\rm r} = 1$.

Obe sústavy dovoľujú vypočítať čiastkové {\it kapacity k~zemi} a čiastkové
{\it kapacity vzájomné}.

\subsection * Potenciál v~poli navzájom rovnobežných vodičov

Dva priamkové, rovnobežné vodiče dĺžky $l$ vo vzájomnej vzdialenosti
$d_{kk} \ll l$, s~lineárnou hustotou náboja $\faz Q_k$ a $\faz Q_{k'} =
- \faz Q_k$ vyvolajú pri zanedbaní koncov vedenia v~ľubovoľnom bode $P$
potenciál
$$
\faz U_P = \faz U_{Pk} + \faz U_{Pk'} = {\faz Q_k \over 2{\rm\pi}
\epsilon} \ln{d_{Pk'} \over d_{Pk}}
$$
kde $\epsilon = \epsilon_0 \epsilon_r$ a $\epsilon_0 = 8{,}854\times
10^{-12}\,\rm F\cdot m^{-1}$.

\beginfigure
	\filename={img/kapacita-bod-P.png}
	\label={kapacita-bodP}
	\caption={Pre určenie potenciálu v~bode $P$}
\endfigure

Výpočet rozšírime na $n$ dvojíc priamkových vodičov navzájom
rovnobežných. Ku každému vodiču $k$ s~lineárnou hustotou $\faz Q_k$
priradíme zodpovedajúci vodič s~lineárnou hustotou $\faz Q_{k'} = \faz
Q_k$, kde $k, k' \in {\rm N}$. Podľa vety o~supoerpozícii bude pre
potenciál v~bode $P$ platiť vzťah
$$
\faz U_P = \sum_{{k=1 \atop k'=1}}^n \left( \faz U_{Pk} + \faz
U_{Pk'}\right) = \sum_{{k=1 \atop k'=1}}^n {\faz Q_k \over 2 {\rm\pi}
\epsilon} \ln {d_{Pk'} \over d_{Pk}}
$$
pričom $d_{kP} = d_{Pk}$ a $d_{k'P} = d_{Pk'}$. Príspevky na potenciál
v~bode $P$ od vodiča $k$ respektíve $k'$ sú $\faz U_{Pk}$ respektíve $\faz
U_{Pk'}$.

Skutočné vodiče nie sú priamkové zdroje s~nábojmi v~osách, ale majú
určitý polomer $r_k$. Predpokladáme, že $r_k \ll d_{km}$.
Ekvipotenciálne plochy sú valcové plochy. Povrch vodiča môžme považovať
za jednu z~ekvipotenciálnych plôch a môžme mu prisúdiť rovnaký potenciál
ako má osa vodiča. Vzdialenosť vodiča od seba samého položíme rovnú jeho
polomeru, teda $d_{kk} = r_k$.

Ak položíme bod $P$ na povrch $m$-tého vodiča, bude jeho potenciál rovný
potenciálu $m$-tého vodiča
$$
\faz U_m = \sum_{{k=1\atop k'=1}}^n {\faz Q_k \over 2{\rm\pi} \epsilon}
\ln {d_{mk'} \over d_{mk}}.
\equation[Um]
$$

\subsection * Rovnice pre výpočet kapacít

V~prípade vonkajších vedení musíme okrem vodičov fází a uzemňovacích lán
rešpektovať aj zem. Zanedbáme nerovnosti povrchu zeme a budeme ju
považovať za ekvipotenciálnu plochu. Taktiež zanedbáme priehyb vodiča a
umiestnime ho do fiktívnej výšky zhodnej s~výškou ťažiska reťazovky.
Ak $H$ je výška závesného bodu, $p$ priehyb, bude $h = H - 0{,}7p$.

Použijeme metódu zrkadlenia a systém vodič--zem nahradíme skutočným
vodičom $k$ a fiktívnym vodičom $k'$ uloženým zrkadlovo pod zemou a
s~opačným nábojom. Ktorýkoľvek bod $Z$, ležiaci na povrchu zeme, bude
mať od oboch vodičov rovnakú vzdialenosť, $d_{Zk} = d_{Zk'}$. Ak túto
podmienku dosadíme do rovnice \citeequation[Um] vidíme, že povrch zeme sa
stáva ekvipotenciálnou plochou s~nulovým potenciálom.

Rovnicu \citeequation[Um] môžme písať v~tvare
$$
\faz U_m = \sum_{k=1}^n \delta_{km} \faz Q_k.
\equation[Umdelty]
$$

Súčinitele $\delta_{mm}$ a $\delta_{mk}$ vyjadríme ako
$$
\eqalign{
\delta_{mm} &= {1 \over 2{\rm\pi}\epsilon} \ln {2 h_m \over r_m}\cr
\delta_{mk} &= {1 \over 2{\rm\pi}\epsilon} \ln {\sqrt{4 h_m h_k +
d_{mk}^2} \over d_{mk}}\hbox{\quad pre } m \ne k.\cr
}
$$

Z~geometrického usporiadania môžme vyjadriť vzťahy medzi potenciálom a
potenciálovými súčiniteľmi. Ak uvažujeme, že $c_{m0}$ je čiastková
kapacita $m$-tého vodiča k~zemi a $c_{km}$ je čiastková vzájomná
kapacita medzi $m$-tým a $k$-tým vodičom, potom pre náboj $m$-tého
vodiča v~systéme s~$n$ vodičmi môžme písať
$$
\faz Q_m =
\faz Q_{m0} + \sum_{{k=1 \atop k\ne m}}^n \faz Q_{km} =
c_{m0} \faz U_m + \sum_{{k=1 \atop k\ne m}}^n c_{km} \left(\faz U_m -
\faz U_k\right)
$$
prípadne
$$
\faz Q_m =
\left(c_{m0} + \sum_{{k=1 \atop k\ne m}}^n c_{km}\right) \faz U_m +
\sum_{{k=1 \atop k\ne m}}^n \left(-c_{km}\right) \faz U_k.
$$

Aby sme tieto rovnice formálne zjednodušili, definujeme kapacitné
súčinitele
$$
\eqalign{
k_{mm} &= c_{m0} + \sum_{{k=1 \atop k \ne m}}^n  c_{km}\cr
k_{mk} &= -c_{km} = k_{km}\cr
}
$$
a môžme písať vzťah
$$
\faz Q_m = \sum_{k=1}^n k_{km} \faz U_k.
\equation[Qm]
$$

Rovnice \citeequation[Umdelty] a \citeequation[Qm] môžme zapísať
v~maticovom tvare ako
\zvyrazni
$$
[\faz U] = [\delta] [\faz Q] \qquad [\faz Q] = [k] [\faz U]
\equation[rovnice-kapacity-maticovo]
$$
kde $[\faz U]$ a $[\faz Q]$ sú stĺpcové matice a $[\delta]$ a $[k]$ sú
štvorcové symetrické a regulárne matice. Z~nich je možné dosadením
dostať rovnosť
$$
[k] = [\delta]^{-1}.
$$

Čiastkové kapacity určíme pri uvážení rovnosti $c_{km} = - k_{km}$ ako
$$
c_{m0} = k_{mm} + \sum_{{k=1 \atop k\ne m}}^n k_{km}.
$$

Pre praktické výpočty používame vzťahy pre potenciálové súčinitele
upravené tak, že nahradíme prirodzený logaritmus za dekadický a jednotky
súčiniteľov požadujeme v~jednotkách $(\rm km / \mu F)$:
\zvyrazni
$$
\eqalign{
\delta_{mm} &= {1 \over 0{,}0242} \log {2h_m \over r_m}\cr
\delta_{km} &= {1 \over 0{,}0242} \log {\sqrt{4h_m h_k + d_{km}^2 }\over d_{km}}\cr
}
$$

Definujeme prevádzkovú kapacitu jednej fázy $m$ ako
$$
\faz C_m =
{c_{m0} \faz U_m + \sum_{m=1}^n c_{km} (\faz U_m - \faz U_k) \over \faz U_m}=
{\faz Q_m \over \faz U_m}.
$$
Obecne je táto kapacita komplexným číslom. 

\subsection Kapacity jednoduchého trojfázového vedenia

Predpokladáme vedenie bez uzemňovacích lán podľa obrázku
\citefigure[trojfazove] a s~použitím vzťahov $\delta_{km} = \delta_{mk}$
a $k_{km} = k_{mk}$ môžme písať rovnice
\citeequation[rovnice-kapacity-maticovo] v~tvare
$$
\displaylines{
\hmatrix{
	\faz U_a \cr
	\faz U_b \cr
	\faz U_c \cr
}
=
\hmatrix{
	\delta_{aa} & \delta_{ab} & \delta_{ac} \cr
	\delta_{ab} & \delta_{bb} & \delta_{bc} \cr
	\delta_{ac} & \delta_{bc} & \delta_{cc} \cr
}
\hmatrix{
	\faz Q_a \cr
	\faz Q_b \cr
	\faz Q_c \cr
},\cr
%
\hmatrix{
	\faz Q_a \cr
	\faz Q_b \cr
	\faz Q_c \cr
}
=
\hmatrix{
	k_{aa} & k_{ab} & k_{ac} \cr
	k_{ab} & k_{bb} & k_{bc} \cr
	k_{ac} & k_{bc} & k_{cc} \cr
}
\hmatrix{
	\faz U_a \cr
	\faz U_b \cr
	\faz U_c \cr
}.
}
$$

Inverziou matice potenciálových súčiniteľov určíme kapacitné súčinitele
a následne tiež čiastkové kapacity:
$$
[\delta]^{-1} =
\hmatrix{
	k_{aa} & k_{ab} & k_{ac} \cr
	k_{ab} & k_{bb} & k_{bc} \cr
	k_{ac} & k_{bc} & k_{cc} \cr
}.
$$

Ak vedenie napríklad transpozíciou {\it symetrizujeme}, vzťahy sa zjednodušia
a platí
$$
\delta =
{1 \over 3} \left\{
\hmatrix{
	\delta_{11} & \delta_{12} & \delta_{13} \cr
	\delta_{12} & \delta_{22} & \delta_{23} \cr
	\delta_{13} & \delta_{23} & \delta_{33} \cr
} +
\hmatrix{
	\delta_{33} & \delta_{13} & \delta_{23} \cr
	\delta_{13} & \delta_{11} & \delta_{12} \cr
	\delta_{23} & \delta_{12} & \delta_{22} \cr
} +
\hmatrix{
	\delta_{22} & \delta_{23} & \delta_{12} \cr
	\delta_{23} & \delta_{33} & \delta_{13} \cr
	\delta_{12} & \delta_{13} & \delta_{11} \cr
} 
\right\}
=
\hmatrix{
\delta & \delta' & \delta' \cr
\delta' & \delta & \delta' \cr
\delta' & \delta' & \delta \cr
}
$$
kde
\zvyrazni
$$
\eqalign{
\delta &= {1 \over 3} (\delta_{11} + \delta_{22} + \delta_{33}) =
{1 \over 0{,}0242} \log {2h \over r}\cr
\delta' &= {1 \over 3} (\delta_{12} + \delta_{13} + \delta_{23}) =
{1 \over 0{,}0242} \log{\sqrt{4h^2 + d^2} \over d}\cr
}
$$
kde $h = \root 3 \of {h_1 h_2 h_3}$ je stredná výška a $d = \root 3 \of
{d_{12} d_{23} d_{13}}$ je stredná vzájomná vzdialenosť vodičov.

Z~inverznej matice potenciálových súčiniteľov získame kapacitné
súčinitele:
$$
\hmatrix{
	\delta & \delta' & \delta' \cr
	\delta' & \delta & \delta' \cr
	\delta' & \delta' & \delta \cr
}^{-1} =
\matrix{
	k & k' & k' \cr
	k' & k & k' \cr
	k' & k' & k \cr
}
$$
kde pre $k$ a $k'$ platí
$$
\displaylines{
k = {\delta + \delta' \over (\delta - \delta')(\delta + 2\delta')}\cr
k' = {-\delta' \over (\delta - \delta')(\delta + 2\delta')}.\cr
}
$$

Vzájomná kapacita je potom rovnaká medzi všetkými fázami a je daná
výrazom
\zvyrazni
$$
c' = -k' = {\delta' \over (\delta - \delta')(\delta + 2\delta')}.
$$

Kapacita k~zemi je tiež rovnaká pre všetky fázy a je daná výrazom
\zvyrazni
$$
c_0 = k + 2k' = {1 \over \delta + 2\delta'}.
$$

Prevádzková kapacita $C$ jednej fázy transponovaného vedenia, ktoré má
symetrické napätia fázovo posunuté o~$120\degree$, je určená vzťahom
\zvyrazni
$$
C = c_0 + 3c' = {1 \over \delta - \delta'}.
$$
Ako dôsledok transpozície je prevádzková kapacita ktorejkoľvek fázy
rovnaké reálne číslo. Po dosadení za potenciálové súčinitele $\delta$ a
$\delta'$ dostaneme vzťah
$$
C = {0{,}0242 \over \log {2 h d \over r\sqrt{4h^2 + d^2}}} \quad ({\rm
\mu F/km})
$$
a špeciálne pri splnení podmienky $4h^2 \gg d^2$, kedy platí $\sqrt{4h^2
+ d^2} \doteq 2h$, dostávame pre prevádzkovú kapacitu približný výraz
\zvyrazni
$$
C \doteq {0{,}0242 \over \log{d\over r}} \quad ({\rm \mu F/km}).
$$

\subsection * Kapacity dvojitého trojfázového vedenia s~dvoma zemniacimi
lanami

Uvážme usporiadanie podľa \citefigure[trojfazove-s-lanami] a
predpokladajme, že obe vedenia sú príslušné jednej elektrizačnej
sústave. Rovnice \citeequation[rovnice-kapacity-maticovo] dostávajú
potom tvar
$$
\displaylines{
\hmatrix{
\faz U_{\rm a} \cr
\faz U_{\rm b} \cr
\faz U_{\rm c} \cr
\faz U_{\rm A} \cr
\faz U_{\rm B} \cr
\faz U_{\rm C} \cr
\faz U_{\rm z_1} \cr
\faz U_{\rm z_2} \cr
} = 
\hmatrix{
\delta_{\rm aa} & \delta_{\rm ab} & \delta_{\rm ac} & \delta_{\rm aA} & \delta_{\rm aB} & \delta_{\rm aC} & \delta_{\rm az_1} & \delta_{\rm az_2} \cr 
\delta_{\rm ab} & \delta_{\rm bb} & \delta_{\rm bc} & \delta_{\rm bA} & \delta_{\rm bB} & \delta_{\rm bC} & \delta_{\rm bz_1} & \delta_{\rm bz_2} \cr 
\delta_{\rm ac} & \delta_{\rm bc} & \delta_{\rm cc} & \delta_{\rm cA} & \delta_{\rm cB} & \delta_{\rm cC} & \delta_{\rm cz_1} & \delta_{\rm cz_2} \cr 
\delta_{\rm aA} & \delta_{\rm bA} & \delta_{\rm cA} & \delta_{\rm AA} & \delta_{\rm AB} & \delta_{\rm AC} & \delta_{\rm Az_1} & \delta_{\rm Az_2} \cr 
\delta_{\rm aB} & \delta_{\rm bB} & \delta_{\rm cB} & \delta_{\rm AB} & \delta_{\rm BB} & \delta_{\rm BC} & \delta_{\rm Bz_1} & \delta_{\rm Bz_2} \cr 
\delta_{\rm aC} & \delta_{\rm bC} & \delta_{\rm cC} & \delta_{\rm AC} & \delta_{\rm BC} & \delta_{\rm CC} & \delta_{\rm Cz_1} & \delta_{\rm cz_2} \cr 
\delta_{\rm az_1} & \delta_{\rm bz_1} & \delta_{\rm cz_1} & \delta_{\rm Az_1} & \delta_{\rm Bz_1} & \delta_{\rm Cz_1} & \delta_{\rm z_1z_1} & \delta_{\rm z_1z_2} \cr 
\delta_{\rm az_2} & \delta_{\rm bz_2} & \delta_{\rm cz_2} & \delta_{\rm Az_2} & \delta_{\rm Bz_2} & \delta_{\rm Cz_2} & \delta_{\rm z_1z_2} & \delta_{\rm z_2z_2} \cr 
}
\hmatrix{
\faz Q_{\rm a} \cr
\faz Q_{\rm b} \cr
\faz Q_{\rm c} \cr
\faz Q_{\rm A} \cr
\faz Q_{\rm B} \cr
\faz Q_{\rm C} \cr
\faz Q_{\rm z_1} \cr
\faz Q_{\rm z_2} \cr
},\cr
\noalign{\smallskip}
\hmatrix{
[\faz U_{\rm v}]\cr
[\faz U_{\rm V}]\cr
[\faz U_{\rm z}]\cr
}
=
\hmatrix{
	[\delta_{\rm vv}] & [\delta_{\rm vV}] & [\delta_{\rm vz}] \cr
	[\delta_{\rm Vv}] & [\delta_{\rm VV}] & [\delta_{\rm Vz}] \cr
	[\delta_{\rm zv}] & [\delta_{\rm zV}] & [\delta_{\rm zz}] \cr
}
\hmatrix{
	[\faz Q_{\rm v}] \cr
	[\faz Q_{\rm V}] \cr
	[\faz Q_{\rm z}] \cr
}.\cr
}
$$

Ak predpokladáme, že zemniace laná sú dobre uzemnené a majú potenciál
zeme, ktorý je nulový -- teda $[\faz U_z] = 0$, môžme určiť
$$
[\faz Q_z] = - [\delta_{zz}]^{-1} \left([\delta_{zv}] [\faz Q_v] +
[\delta_{zV}] [\faz Q_V]\right).
$$

Po úprave dostaneme vzťahy pre fiktívne dvojité vedenie bez zemniacich
lán
$$
\eqalign{
[\faz U_v] &=
\left([\delta_{vv}] -
[\delta_{vz}][\delta_{zz}]^{-1}[\delta_{zv}]\right)[\faz Q_v] +
\left([\delta_{vV}]-[\delta_{vz}][\delta_{zz}]^{-1}[\delta_{zV}]\right)[\faz Q_V],\cr
[\faz U_V] &=
\left([\delta_{Vv}] -
[\delta_{Vz}][\delta_{zz}]^{-1}[\delta_{zv}]\right)[\faz Q_v] +
\left([\delta_{VV}]-[\delta_{Vz}][\delta_{zz}]^{-1}[\delta_{zV}]\right)[\faz Q_V].\cr
}
$$
Tieto sústavy rovníc umožňujú transformáciu do zložkových sústav.

%%%%%%%%%%%%%%%%%%%%%%%%%%%%%%%%%%%%%%%%%%%%%%%%%%%%%%%%%%%%%%%%%%%%%%%%%%%%%%%%%%%%%

\section Elektrické parametre káblových vedení

Káblové vedenia sa používajú najmä v~zastavaných oblastiach, kde je
nemožné stavať stožiare. Pre nižšie napätia môžu byť trojfázové, ale pre
napätia nad 110\,kV najmä jednofázové. Možnosti konštrukčného
vyhotovenia sú znázornené na \citefigure[kable-konstrukcia].

\beginfigure
	\filename={img/kable-konstrukcia.png}
	\label={kable-konstrukcia}
	\caption={Možné konštrukčné vyhotovenia káblových vedení; a)
	jednožilové s~vlastným plášťom, b) trojžilové s~plášťom na každej
	žile, c) trojžilové s~kovovým papierom na každej žile, d) náhradná
	schéma pre b), c); 1 -- vodič,
	2 -- izolácia vodiča, 3 -- vodivý plášť, 4 -- pokovený papier, 5 --
	ocelový pancier}
\endfigure

\subsection Pozdĺžna impedancia

Tvorí ju rezistancia $R$ a indukčná reaktancia $X_L = \omega L$, podobne
ako v~prípade vonkajších vedení. V~prípade rezistancie sa uplatňujú
vírivé prúdy a hysterézia a taktiež jav blízkosti.

Pri výpočte indukčnej reaktancie použijeme rovnaké vzťahy ako v~prípade
vonkajších vedení, ale pretože podmienka $d \gg r$ nie je splnená, sú
vypočítané hodnoty menej presné. Pre technické účely sú však použiteľné.

\subsection Priečna admitancia

Tvorí ju konduktancia $G$ a kapacitná susceptancia $B = \omega C$. Rôzne
konštrukčné usporiadania a izolačné materiály vodičov spôsobujú značne
rozdielne hodnoty v~jednotlivých prípadoch.

\subsection * Konduktancia

Súvisí s~dielektrickými stratami v~izolácii káblu. 

\beginfigure
	\filename={img/kabel-konduktancia.png}
	\label={kabel-konduktancia.png}
	\caption={Fázorový diagram pre určenie konduktancie káblového vedenia}
\endfigure

Dielektrické straty na 1\,km dĺžky a {\it jednu fázu} sú
$$
\Delta P_{\rm d1} = U_{\rm f} I_{\rm qč} = U_{\rm f} I_{\rm qj}
\tan\delta = Q_{\rm c1} \tan\delta = \omega C \cdot U_{\rm f}^2
\tan\delta \quad ({\rm W/km})
$$
kde $U_{\rm f}$ vo (V) je fázové napätie, $Q_{\rm c}$ vo (VAr/km) je
nabíjací výkon, $C$ vo (F/km) je prevádzková kapacita a $\tan\delta$ je
tangenta stratového uhlu, značne závislá na druhu dielektrika a teplote.

Konduktancia na 1\,km dĺžky na jednu fázu je rovná
$$
G_1 = {\Delta P_{\rm d1} \over U_{\rm f}^2} \quad ({\rm S/km; W/km, V})
$$
a v~prípade dobre udržovaných káblov býva uvažovaná až od menovitých
napätí 220\,kV, ak nejde o~výpočet strát.

\subsection * Kapacity káblových vedení

Rozlišujeme dve skupiny káblov. Prvú tvoria jednožilové a viacžilové
s~vlastným kovovým obalom na každej žile, druhú tvoria viacžilové so
spoločným kovovým plášťom pre všetky žily.

V~prvom prípade máme iba jedinú kapacitu, ktorá je zároveň kapacitou
vodiča oproti plášťu a zároveň prevádzková. Elektrické pole je radiálne
a kapacita sa vypočíta ako kapacita valcov s~totožnou osou:
\zvyrazni
$$
C = c_{\rm ko} = {0{,}0242 \epsilon_{\rm r} \over \log{r_2 \over r_1}} \quad
({\rm \mu F/km})
$$
kde $\epsilon_{\rm r}$ je pomerná permitivita izolačnej hmoty medzi
vodičom a kovovým obalom, $r_1$ je polomer vodiča a $r_2$ je vnútorný
polomer kovovej obálky.

V~druhom prípade sa vyskytujú ďalšie tri čiastkové kapacity voči plášťu
$c_o$ a vzájomné $c'$ (ako je znázornené na \citefigure[kabel-kapacity])
pre všetky fázy rovnaké, vďaka geometrickej súmernosti. Pre výpočet
vytvoríme náhradný matematický model. Bude ním sústava troch dvojíc
vodičov navzájom rovnobežných, ktorá spĺňa podmienku, že obálka je
ekvipotenciálnou plochou. Jeden vodič je skutočný, druhý v~dvojici je
fiktívny. 

\beginfigure
	\filename={img/kabel-kapacity.png}
	\label={kabel-kapacity}
	\caption={K výpočtu kapacity káblového vedenia so spoločným kovovým
	plášťom pre všetky žily}
\endfigure

Podľa predpokladu je obalový valec ekvipotenciálnou plochou. Pre jeho
potenciál $\faz U_{\rm pl}$ bude s~použitím bodov $P_1$ a $P_2$ platiť
$$
\faz U_{\rm pl} = \faz U_{P_1} =
{\faz Q_k \over 2{\rm\pi} \epsilon} \ln {a' - R \over R - a} =
\faz U_{P_2} =
{\faz Q_k \over 2{\rm\pi}\epsilon} \ln {a' + R \over a + R}.
$$
Aby táto rovnica bola splnená, musí platiť 
$$
{a' + R \over a + R} = {a' - R \over R - a} \longrightarrow a' = {R^2
\over a}.
$$
Potom však dostávame vzťah
$$
\faz U_{\rm pl} = {\faz Q_k \over 2{\rm\pi}\epsilon} \ln {R \over a} =
\faz U_{P_1} = \faz U_{P_2}.
$$

Pretože kapacity $c_o$ vzťahujeme k~plášťu, nesmieme pracovať
s~potenciálmi vodičov a zeme, ale musíme používať rozdiely
potenciálov vodičov a plášťa. Príspevok na potenciál vodiča $k$ od
vodičov $m$, $m'$ potom bude
$$
\faz U_{km}^* = \faz U_{km} - \faz U_{\rm pl} = {\faz Q_m \over
2{\rm\pi}\epsilon}\left[\ln{d_{km'} \over d_{km}} - \ln{R \over
a}\right] = {\faz Q_m \over 2{\rm\pi} \epsilon} \ln{{d_km'} a \over
d_{km} R}
$$
a teda platí
$$
\delta_{km}^* = \delta_{mk}^* = {1 \over 2{\rm\pi}\epsilon} \ln{d_{km'} a
\over d_{km} R}.
\equation[delta*]
$$

\beginfigure
	\filename={img/kapacity-zrkadlenie.png}
	\label={kapacity-zrkadlenie}
	\caption={Rozmiestnenie skutočných a fiktívnych vodičov v~trojfázovom
	kábli}
\endfigure

Potrebné vzdialenosti pre výpočet potenciálových súčiniteľov zistíme
z~geometrie kábla určenej voľbou konkrétneho typu kábla. Tým máme určené polomery vodičov
$$
d_{kk} = r_a = r_b = r_c = r,
$$
vzdialenosti dvoch skutočných vodičov pre $m\ne k$ 
$$
d_{km} = d_{ab} = d_{ac} = d_{bc} = a\sqrt{3},
$$ 
vzdialenosti sebe zodpovedajúcich si vodičov
$$
d_{kk'} = d_{aa'} = d_{bb'} = d_{cc'} = a' - a = {R^2 - a^2 \over a},
$$
a vzdialenosti nezodpovedajúcich si vodičov
$$
d_{km'} = d_{ab'} = d_{ac'} = d_{bc'} = R \sqrt{{R^2 \over a^2} + 1 +
{a^2 \over R^2}}.
$$


Z~rovnice \citeequation[delta*] po zavedení dekadického logaritmu a
požadovaných jednotiek dostávame
\zvyrazni
$$
\eqalign{
\delta_{kk}^* = \delta &=
{1 \over 0{,}0242 \epsilon_{\rm r}} \log{R^2 - a^2 \over Rr}
\quad ({\rm km/\mu F})\cr
\delta_{km}^* = \delta' &=
{1 \over 0{,}0242 \epsilon_{\rm r}}
\log\sqrt{{1 + (R^2/a^2) + (a^2/R^2) \over 3}}\quad ({\rm km/\mu F})\qquad\hbox{ pre } k \ne m\cr
}
$$

Trojžilový kábel je trojfázové súmerné vedenie a preto preň môžme písať
analogické rovnice ako pre symetrizované trojfázové vonkajšie vedenie
bez uzemňovacích lán, teda platí, že čiastková kapacita vodiča k~plášťu
je
\zvyrazni
$$
c_o = {1 \over \delta + 2\delta'} \quad ({\rm \mu F/km}),
$$
čiastková vzájomná kapacita je
\zvyrazni
$$
c' = {\delta' \over (\delta - \delta')(\delta + 2\delta')} \quad ({\rm \mu F/km})
$$
a prevádzková kapacita je
\zvyrazni
$$
C = c_o + 3c' = {1 \over \delta - \delta'}\quad ({\rm \mu F/km}).
$$
