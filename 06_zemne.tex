\section Zemné spojenia

V~sieťach prevádzkovaných s~izolovaným uzlom, prípadne v~sieťach,
ktorých uzol je pripojený na zem cez zhášaciu tlmivku, nazývame vodivé
spojenie jednej fázy so zemou zemné spojenie. Zásadný rozdiel medzi
jednofázovým skratom a zemným spojením je v~tom, že skratový prúd je
väčšinou niekoľkonásobne väčší než prúd prevádzkový a má indukčný
charakter, zato v~mieste zemného spojenia prechádza iba malý prúd
kapacitného charakteru. Ďalšou zvláštnosťou tohto prúdu je, že nezávisí
na vzdialenosti od zdroja (je približne rovnaký vo všetkých miestach
siete).

\beginfigure
	\filename={img/zs-3f-izol-u.png}
	\label={zs-3f-izol-u}
	\caption={Trojfázová sústava s~izolovaným uzlom}
\endfigure

Pre napätia a prúdy podľa obrázku \citefigure[zs-3f-izol-u] platia vzťahy
$$
\displaylines{
\faz U_{{\rm f}k} + \faz U_0 - \faz U_{k} = 0\cr
\faz I_k = \j \omega k_{k0} \faz U_k
}
$$
pre $k \in {a, b, c}$.

Pretože sieť má izolovaný uzol, musí pre prúdy platiť
$$
\faz I_a + \faz I_b + \faz I_c = 0 = \sum \faz I_k.
$$

Vzhľadom na to, že napätia sú rovnaké a iba fázovo posunuté, teda platí
$$
\faz U_{{\rm f}b} = \faz a^2 \faz U_{{\rm f}a} \qquad
\faz U_{{\rm f}c} = \faz a \faz U_{{\rm f}a},
$$
môžme vzťah pre $\faz U_0$, zostavený na základe predošlých rovníc,
napísať ako
$$
\faz U_0 = - 
{k_{a0} + \faz a^2 k_{b0} + \faz a k_{c0} \over
k_{a0} + k_{b0} + k_{c0}}
\faz U_{{\rm f}a}.
$$

Je vidieť, že v~prípade kapacitne symetrickej siete, kde platí 
$$
k_{a0} = k_{b0} = k_{c0} = k_0,
$$
bude napätie izolovaného uzla voči zemi pri bezporuchovom stave nulové,
$$
\faz U_0 = 0.
$$

\beginfigure
	\filename={img/zs.png}
	\label={zs}
	\caption={Približné vyjadrenie veľkostí prúdov vo fázach a v zemi v
	závislosti na polohe zemného spojenia}
\endfigure

\subsection Dokonalé trvalé zemné spojenie

Pri dokonalom (kovovom) trvalom zemnom spojení bude schéma symetrickej
siete podľa \citefigure[zs-trvale].

\beginfigure
	\filename={img/zs-trvale.png}
	\label={zs-trvale}
	\caption={Trvalé zemné spojenie}
\endfigure

Poruchový prúd $\faz I_p$ je zložený z~dvoch prúdov, ktoré tečú cez
kapacity nepostihnutých fáz. Kapacita postihnutej fázy je preklenutá
poruchou a preto sa neuplatní.

Z~\citefigure[zs-trvale] plynú nasledovné vzťahy:
$$
\displaylines{
\faz I_p = \faz I_a = \faz I_b + \faz I_c, \cr
\faz U_a = 0, \cr
\faz I_b = \j \omega k_0 \faz U_b \qquad
\faz I_c = \j \omega k_0 \faz U_c, \cr
\faz U_{{\rm f}k} + \faz U_0 - \faz U_k = 0 \quad \forall k \in \{a,b,c\}.\cr
}
$$

Pretože $\faz U_a = 0$, platí pre napätia
$$
\displaylines{
\faz U_0 = - \faz U_{{\rm f}a}\cr
\faz U_b = \faz U_0 + \faz U_{{\rm f}b} = (-1 + \faz a^2) \faz U_{{\rm
f}a} = - \sqrt3 {\rm e}^{\j 30\degree} \faz U_{{\rm f}a}\cr
\faz U_c = \faz U_0 + \faz U_{{\rm f}c} = (-1 + \faz a) \faz U_{{\rm f}a}
= -\sqrt3 {\rm e}^{-\j 30\degree} \faz U_{{\rm f}a}\cr
}
$$

Pri dokonalom zemnom spojení sa zväčšia napätia nepostihnutých fáz
oproti zemi na združenú hodnotu.


Pre poruchový prúd platí vzťah
$$
\faz I_p = \faz I_b + \faz I_c = \j \omega k_0 (\faz U_b + \faz U_c)
$$
kam dosadíme za $\faz U_b$ a $\faz I_c$ a dostávame
$$
\eqalign{
\faz I_p &=
\j \omega k_0 (-\faz U_{{\rm f}a} + \faz a^2 \faz U_{{\rm f}a} - \faz
U_{{\rm f}a} + \faz a \faz U_{{\rm f}a}) = \cr
&= \j \omega k_0 \faz U_{{\rm f}a} [-1 + \faz a^2 - 1 +\faz a] =\cr
&= \j \omega k_0 \faz U_{{\rm f}a} [-1 + (\faz a^2 + \faz a + 1) -1 -1] = \cr
&= \j \omega k_0 \faz U_{{\rm f}a} [-1 + 0 - 1 -1] = \cr
&= -3 \j \omega k_0 \faz U_{{\rm f}a} = 3\j \omega k_0 \faz U_0.\cr
}
$$

Pre veľkosť poruchového prúdu platí
$$
I_p = 3 \omega k_0 U_0 \quad ({\rm A; s^{-1}, F, V}).
$$

Je vidieť, že poruchový prúd prebieha o~${\rm\pi/2}$ napätie v~uzle
sústavy a je súčtom kapacitných prúdov z~ostatných, nepostihnutých,
fáz. Závisí na celkovej rozlohe siete pripojenej k~transformátoru. Jeho
veľkosť v~danej sieti prakticky nezáleží na vzdialenosti miesta poruchy
od transformátoru.

Napäťové a prúdové pomery sú znázornené vo fázorovom diagrame
v~\citefigure[zs-trvale-fazd].

\beginfigure
	\filename={img/zs-trvale-fazd.png}
	\label={zs-trvale-fazd}
	\caption={Fázorový diagram trvalého zemného spojenia}
\endfigure

\subsection Odporové zemné spojenie

V~prípade odporového zemného spojenia je poruchový prúd obmedzený
prechodovým odporom, ktorého hodnota je niekoľko stoviek ohmov.

Základná schéma je zobrazená na \citefigure[zs-odporove-schema].

\beginfigure
	\filename={img/zs-odporove-schema.png}
	\label={zs-odporove-schema}
	\caption={Odporové zemné spojenie}
\endfigure

V~tomto prípade platia rovnice
$$
\displaylines{
\faz U_{{\rm f}k} + \faz U_0 - \faz U_k = 0 \cr
\faz I_k = \j\omega k_{k0} \faz U_k \cr
\faz I_p = - R_p^{-1} \faz U_a \cr
%
\faz U_0 =
- {\j\omega(k_{a0} + \faz a^2 k_{b0} + \faz a k_{c0}) + R_p^{-1} \over
\j\omega(k_{a0} + k_{b0} + k_{c0}) + R_p^{-1}} \faz U_{{\rm f}a}. \cr
}
$$
Symbolicky zapísané,
$$
\faz U_0 = - {\faz A + G_p \over \faz B + G_p} \faz U_{{\rm f}a} \qquad
\faz U_0 = f(G_p).
$$

Rozlišujeme dva krajné prípady vzhľadom na $G_p$:
\beginitemize
	\item $G_p = 0$
	$$
		[\faz U_0]_{G_p = 0} = 0,
	$$
	\item $G_p \to \infty$
	$$
		[\faz U_0]_{G_p \to \infty} = - \faz U_{{\rm f}a}.
	$$
\enditemize

Napäťové a pomery sú znázornené vo fázorovom diagrame
v~\citefigure[zs-odporove-fazd].

\beginfigure
	\filename={img/zs-odporove-fazd.png}
	\label={zs-odporove-fazd}
	\caption={Fázorový diagram odporového zemného spojenia}
\endfigure

\subsection Zhášacia tlmivka

Zhášacia tlmivka predstavuje najpoužívanejší spôsob kompenzácie zemných
prúdov. Zapája s~medzi uzol transformátoru a zem. Jej indukčnosť
nastavujeme tak, aby indukčný prúd tečúci od nej k~miestu zemného
spojenia kompenzoval kapacitné prúdy zdravých fáz a aby došlo k~uhaseniu
oblúka.

\beginfigure
	\filename={img/zs-tlmivka.png}
	\label={zs-tlmivka}
	\caption={Zemné spojenie so zhášacou  tlmivkou}
\endfigure

Pre poruchový prúd v~takomto usporiadaní platí vzťah
$$
\faz I_p = \faz I_L + \faz I_b + \faz I_c.
$$

Je vidieť, že ak sa nám podarí dosiahnuť vhodnú veľkosť prúdu tlmivkou,
poruchový prúd môžme vykompenzovať.

Prúd tečúci ideálnou zhášacou tlmivkou bude oneskorený o~$\rm\pi/2$ za
napätím $\faz U_0$ a bude preň platiť vzťah
$$
\faz I_L = -\j {1 \over \omega L} \faz U_0.
$$


S~uvážením vzťahov
$$
\displaylines{
\faz I_p = \faz I_b + \faz I_c = 3 \j \omega k_0 \faz U_0,\cr
\faz I_p \pust 0\cr
}
$$
dostávame pre poruchový prúd a následne pre indukčnost vzťahy
$$
\faz I_p = \j \faz U_0 \left(3\omega k_0 - {1 \over \omega L} \right)
$$
\zvyrazni
$$
L={1\over 3\omega^2 k_0}
$$

Zdanlivý výkon zhášacej tlmivky určíme ako
$$
\faz S = \faz U_0 \faz I_L^* = \j {U_0^2\over \omega L} = \faz j {k_0
U_0^2 3\omega^2 \over \omega} = \j \omega k_0 U^2
$$

Pri ideálnej kompenzácii netečie miestom poruchy žiaden prúd a teda
oblúk sa neudrží. V~skutočnosti však zostáva v~mieste poruchy zbytkový
prúd -- kompenzácia nie je nikdy ideálna, napríklad vplyvom nepresného
nastavenia indukčnosti, nevykompenzovateľnou zložkou spôsobenou zvodmi
vedení a činným odporom zhášacej tlmivky alebo prúdmi vyšších
harmonických. Avšak ak je zbytkový prúd malý, aj tak môže dôjsť
k~uhaseniu oblúka. 

Napäťové a prúdové pomery sú znázornené vo fázorovom diagrame
v~\citefigure[zs-tlmivka-fazd].

\beginfigure
	\filename={img/zs-tlmivka-fazd.png}
	\label={zs-tlmivka-fazd}
	\caption={Fázorový diagram zemného spojenia so zhášacou tlmivkou}
\endfigure

\subsection * Ladenie zhášacej tlmivky

Vhodná veľkosť indukčnosti zhášacej tlmivky sa stanovuje pri návrhu a
voľbe tlmivky výpočtom. Správne nastavenie sa potom vykonáva
v~bezporuchovom stave danej siete. Pri odpojení alebo pripojení niektorého
vedenia do rozvodne je potrebné tlmivku preladiť. Preladenie je možné
odbočkami, alebo jednoducho a plynulo, na diaľku, ručne alebo samočinne,
zmenou magnetického obvodu pomocou motoru.

\beginfigure
	\filename={img/zs-tlmivka-ladenie.png}
	\label={zs-tlmivka-ladenie}
	\caption={Ladenie zhášacej tlmivky pripojenej do uzla trojfázovej
	siete}
\endfigure

Pre zapojenie na \citefigure[zs-tlmivka-ladenie] platia rovnice
$$
\displaylines{
\faz U_{0}^+ = \j \omega L \faz I_L\cr
\faz U_k = \faz U_{{\rm f}k} + \faz U_0^+ \cr
}
$$
pre $k \in {a, b, c}$. Pre prúdy platia rovnice
$$
\displaylines{
- \faz I_L =
\faz I_a + \faz I_b + \faz I_c =
\j\omega (k_{a0} \faz U_a + k_{b0} \faz U_b + k_{c0} \faz U_c)
}
$$
kam po dosadení za napätia a za prúd $\faz I_L$ dostávame vzťah
$$
\faz U_0^+ =
{-\omega^2 L \left(k_{a0} + \faz a^2 k_{b0} + \faz a k_{c0}\right) \over
\omega^2 L \left(k_{a0} + k_{b0} + k_{c0}\right) -1} \faz U_{{\rm f}a}.
$$

Graf funkcie
$$
\left|{U_0^+ \over U_{{\rm f}a}}\right| = f(L)
$$
znázornený na \citefigure[zs-rezon-kr] je rezonančnou krivkou obvodu na
\citefigure[zs-tlmivka-ladenie].

\beginfigure
	\filename={img/zs-rezon-kr.png}
	\label={zs-rezon-kr}
	\caption={Rezonančné krivky}
\endfigure

Priamka prechádzajúca vrcholom určuje indukčnosť $L_{\rm rez}$ pri
ktorej dochádza v~prípade zemného spojenia k~úplnej kompenzácii zemného
prúdu.

V~prípade káblového vedenia je krivka plochá, pretože je zaručená dobrá
symetria. V~prípade nesymetrických vedení by pri naladení do rezonancie
bol uzol transformátoru namáhaný veľkým napätím. Preto je zhášacia
tlmivka zámerne rozladená o~hodnotu $\Delta L$. Tým poklesne aj napätie.
Hlavné je, aby aj pri tomto rozladení dokázala tlmivka vykompenzovať
poruchový prúd aspoň na takú hodnotu, aby zhasol oblúk.


\subsection Trvalé zemné spojenie metódou súmerných zložiek

V~prípade poruchy na fáze $a$ a jej spojením so zemou môžme napísať
charakteristické rovnice
$$
\faz U_a = 0 \quad \faz I_b = 0 \quad \faz I_c = 0.
$$

\beginfigure
	\filename={img/tzs.png}
	\label={tzs}
	\caption={Trvalé zemné spojenie}
\endfigure


Na \citefigure[tzs] je $\faz Z_{\rm s}$ impedancia sústavy, 
$\faz Z_{\rm t}$ impedancia transformátora, $\faz Z_{\rm v}$
impedancia vedenia a $\faz Y_c$ je kapacitná admitancia vedenia. Pre ich
veľkosti platí
$$
\displaylines{
\faz Y_c^{-1} = \faz Z_c = - \j X_c \cr
\faz Z_c \gg \faz Z_{\rm v}, \faz Z_{\rm t}, \faz Z_{\rm s} \cr
}
$$

Prechodom k~zložkovým veličínám získame pre zložkové prúdy rovnice
$$
\faz I_F = \faz F^{-1} \faz I = {1 \over 3} \hmatrix{
1 & \faz a & \faz a^2 \cr
1 & \faz a^2 & \faz a \cr
1 & 1 & 1 \cr
}
\hmatrix{\faz I_a \cr 0 \cr 0 \cr} =
{1 \over 3} \hmatrix{\faz I_a \cr \faz I_a \cr \faz I_a \cr}
$$
Pre prúdy v~zložkových sústavách platia vzťahy
$$
\faz I_1 = \faz I_2 = \faz I_0 = {1 \over 3} \faz I_a
$$
a podľa nich môžme vzájomne prepojiť náhradné schémy zložkových sústav.
Ak však zohľadníme veľkosti jednotlivých impedancií, môžme zapojenie
zjednodušiť tak, ako je na \citefigure[tzs-zlozky].

\beginfigure
	\filename={img/tzs-zlozky.png}
	\label={tzs-zlozky}
	\caption={Prepojenie náhradných schém zložkových sústav pri zanedbaní
	indukčných reaktancií článkov ES}
\endfigure

Na základe \citefigure[tzs-zlozky] môžme napísať rovnice pre zložkové
prúdy
$$
\faz I_1 = {\faz E \over -\j X_{c0}} \quad \faz I_2 = \faz I_1 \quad
\faz I_0 = \faz I_1.
$$

Zložkové napätia v~mieste poruchy sú
$$
\faz U_1 = \faz E \quad \faz U_2 = 0 \quad \faz U_0 = - \faz E.
$$

Fázové prúdy sú rovné
$$
\faz I_a = 3 \faz I_1 \quad \faz I_b = 0 \quad \faz I_c = 0
$$
a poruchový prúd je rovný
$$
\faz I_p = -\faz I_a = -3 \j {\faz E \over X_{c0}} = -3 \j \omega k_0
\faz E.
$$

Fázové napätia stanovíme ako
$$
\eqalign{
\faz U_a &= 0, \cr
\faz U_b &= \faz U_{b1} + \faz U_{b2} + \faz U_{b0} = \faz a^2\faz U_1 +
\faz a \faz U_2 + \faz U_0 = \faz a^2 \faz E - \faz E = (\faz a^2 - 1)\faz E, \cr
\faz U_c &= \faz a \faz U_1 + \faz a^2 \faz U_2 + \faz U_0 = \faz a \faz
E - \faz E = (\faz a - 1) \faz E.\cr
}
$$

Fázor napätia uzla stanovíme ako
$$
\displaylines{
\faz U_N = {1 \over 3} (\faz U_a + \faz U_b + \faz U_c) =
{1 \over 3} (\faz a^2 -1 + \faz a - 1) \faz E,\cr
\faz U_N = - \faz E.\cr
}
$$


\beginfigure
	\filename={img/zs-nap.png}
	\label={zs-nap}
	\caption={Fázorový diagram napätí pri trvalom zemnom spojení}
\endfigure

\beginfigure
	\filename={img/zs-pru.png}
	\label={zs-pru}
	\caption={Fázorový diagram prúdov pri trvalom zemnom spojení}
\endfigure

\subsection * S~uzlom transformátoru uzemneným cez zhášaciu tlmivku

Náhradná zložková schéma sa bude líšiť od predchádzajúceho prípadu tým,
že v~obvode transformátoru bude v~schéme nulovej zložky zapojená
trojnásobná reaktancia $X_L$. Dostávame tak náhradnú zložkovú schému,
ktorá je znázornená na \citefigure[zs-zlozk-ind-ns].

\beginfigure
	\filename={img/zs-zlozk-ind-ns.png}
	\label={zs-zlozk-ind-ns}
	\caption={Náhradná schéma zložkových sústav pri zemnom spojení a pri
	uzemnení uzla transformátora cez zhášaciu tlmivku}
\endfigure

Aj v~tomto prípade zanedbáme rezistancie a indukčné reaktancie ostatných
článkov oproti kapacitným reaktanciám. Pre celkové impedancie zložkových
schém potom platí
$$
\faz Z_{c1} = 0 \quad \faz Z_{c2} = 0 \quad
\faz Z_{c0} = {\j 3 X_L (- \j X_c) \over \j 3X_L - \j X_c} =
\j {3X_L X_c \over X_c - 3 X_L}.
$$

Pre súsledný prúd platí vzťah
$$
\faz I_1 = {\faz E \over \faz Z_{c1} + \faz Z_{c2} + \faz Z_{c0}}=
{\faz E \over \j {3X_L X_{c0} \over X_{c0} - 3X_L}} =
-\j {X_{c0} - 3X_L \over 3X_L X_{c0}} \faz E
$$
a pre poruchový zemný prúd platí vzťah
$$
\faz I_p = -\faz I_a = -3 \faz I_1 =
\j {X_c - 3 X_L \over X_L X_c} \faz E.
$$

Z~podmienky aby bol poruchový prúd nulový, $\faz I_p \pust 0$, plynie
vzťah
$$
\displaylines{
X_{c0} - 3 X_L = 0 \cr
X_L = {1 \over 3} X_{c0} = {1 \over 3\omega k_0}
}
$$

