\section Transformácia do zložkových sústav

Ide v~podstate o~náhradu fázorov fyzikálnych veličín zložkami. Často sa
používa názov \uv{rozklad do zložiek}. Motiváciou je zjednodušenie
výpočtu v~nesúmerných sústavách.

\subsection Prechod do zložkových sústav

V~prípade súmerných trojfázových sústav je dostačujúce uvažovať iba
jednu fázu, druhé dve sú iba časovo posunuté. V~prípade nesymetrických
sústav však je situácia komplikovanejšia (indukčné a kapacitné väzby
medzi fázami neumožňujú obecne nezávislé a izolované uvažovanie
jednotlivých fáz) a preto si pomáhame transformáciami, pomocou ktorých
nahrádzame veličiny pôvodnej sústavy veličinami zložkovej sústavy.
Pritom platí, že fyzikálne zákony platia aj v~sústave pôvodnej, aj
v~sústave zložkovej.

V~pôvodnej trojfázovej sústave platí
$$
[\faz U_{\rm a}, \faz U_{\rm b}, \faz U_{\rm c}]^T = [\faz U_{\rm abc}] \qquad
[\faz I_{\rm a}, \faz I_{\rm b}, \faz I_{\rm c}]^T = [\faz I_{\rm abc}]
$$
a v~transformovanej sústave platí
$$
[\faz U_{\rm o}, \faz U_{\rm m}, \faz U_{\rm n}]^T = [\faz U_{\rm omn}] \qquad
[\faz I_{\rm o}, \faz I_{\rm m}, \faz I_{\rm n}]^T = [\faz I_{\rm omn}].
$$


Transformáciu zabezpečujú štvorcové regulárne prevodné matice $[T_u]$ a
$[T_i]$, pre ktoré z~definície platí
$$
\eqalign{
[\faz U_{\rm abc}] = [T_u] [\faz U_{\rm omn}] &\quad
[\faz U_{\rm omn}] = [T_u]^{-1} [\faz U_{\rm abc}],\cr
[\faz I_{\rm abc}] = [T_i] [\faz I_{\rm omn}] &\quad
[\faz I_{\rm omn}] = [T_i]^{-1} [\faz I_{\rm abc}]\cr
}
$$

Transformačná matica je obvykle rovnaká pre napätie aj pre prúd, teda
$$
[T_u] = [T_i] = [T].
$$

Vzťah medzi napätím a prúdom v~trojfázovej sústave popíšeme rovnicou
$$
\hmatrix{
\faz U_{\rm a} \cr \faz U_{\rm b} \cr \faz U_{\rm c} \cr}
=
\hmatrix{
\faz Z_{11} & \faz Z_{12} & \faz Z_{13} \cr 
\faz Z_{21} & \faz Z_{22} & \faz Z_{23} \cr 
\faz Z_{31} & \faz Z_{32} & \faz Z_{33} \cr 
}
\hmatrix{
\faz I_{\rm a} \cr
\faz I_{\rm b} \cr
\faz I_{\rm c} \cr
}
$$
a vzťah medzi napätím a prúdom v~zložkovej sústave rovnicou
$$
\hmatrix{
\faz U_{\rm o} \cr \faz U_{\rm m} \cr \faz U_{\rm n} \cr}
 =
\hmatrix{
\faz Z_{\rm oo} & \faz Z_{\rm om} & \faz Z_{\rm on} \cr 
\faz Z_{\rm mo} & \faz Z_{\rm mm} & \faz Z_{\rm mn} \cr 
\faz Z_{\rm no} & \faz Z_{\rm nm} & \faz Z_{\rm nn} \cr 
}
\hmatrix{
\faz I_{\rm o} \cr
\faz I_{\rm m} \cr
\faz I_{\rm n} \cr
}.
$$

Medzi maticami impedancií pôvodnej sústavy a zložkovej sústavy platí
vzťah, ktorý postupne dostaneme ako
$$
\displaylines{
[\faz U_{\rm abc}] = [\faz Z_{\rm abc}][\faz I_{\rm abc}]\cr
[T_u][\faz U_{\rm omn}] = [\faz Z_{\rm abc}][T_i][\faz I_{\rm omn}]\cr
[\faz U_{\rm omn}] = [T_u]^{-1}[\faz Z_{\rm abc}][T_i][\faz I_{\rm omn}]\cr
[\faz Z_{\rm omn}] = [T_u]^{-1} [\faz Z_{\rm abc}] [T_i]\cr
}
$$
a analogicky platí vzťah pre admitanciu, ktorý postupne dostaneme ako
$$
\displaylines{
[\faz I_{\rm abc}] = [\faz Y_{\rm abc}] [\faz U_{\rm abc}]\cr
[T_i] [\faz I_{\rm omn}] = [\faz Y_{\rm abc}] [T_u] [\faz U_{\rm omn}]\cr
[\faz I_{\rm omn}] = [T_i]^{-1} [\faz Y_{\rm abc}] [T_u][\faz U_{\rm omn}]\cr
[\faz Y_{\rm omn}] = [T_i]^{-1} [\faz Y_{\rm abc}] [T_u].\cr
}
$$

Obecne použitie zložiek neprináša žiadne výhody, ani zjednodušenie
výpočtov. Prínos transformácie do zložiek je iba vtedy, ak sa stanú
matice $[\faz Z_{\rm omn}]$ respektíve $[\faz Y_{\rm omn}]$
diagonálnymi. Diagonalizáciu je možné dosiahnuť pre matice cyklicky
súmerné $[\faz Z_{\rm cs}]$ a matice fázovo súmerné $[\faz Z_{\rm fs}]$,
ktoré majú tvar
$$
[\faz Z_{\rm cs}] =
\matrix{
\faz Z & \faz Z' & \faz Z'' \cr
\faz Z'' & \faz Z & \faz Z' \cr
\faz Z' & \faz Z'' & \faz Z \cr
}\qquad
[\faz Z_{\rm fs}] =
\matrix{
\faz Z & \faz Z' & \faz Z' \cr
\faz Z' & \faz Z & \faz Z' \cr
\faz Z' & \faz Z' & \faz Z \cr
}
$$

Matica $[\faz Z_{\rm fs}]$ predstavuje symetrickú sústavu, kde sú prvky
v~hlavnej diagonále rovnaké pre všetky fázy (vlastné impedancie) a mimo
nej rôzne (vzájomné impedancie).

Podmienky pre prvky transformačnej matice za účelom diagonalizácie matíc
stanovíme pomocou charakteristických čísel $\lambda$ a
charakteristických vektorov príslušných matici $[T]$. Charakteristické
vektory sú nenulové riešenia maticovej rovnice
$$
\left([\faz Z_{\rm abc}] - \lambda[E]\right)[t] = [0]
\equation[charvekt]
$$
kde $[E]$ je jednotková matica a $[0]$ nulový vektor. Nenulové riešenia
tejto rovnice existujú iba vtedy ak
$$
\det\left([\faz Z_{\rm abc}] - \lambda[E]\right) = 0.
$$
Riešením tejto rovnice sú charakteristické čísla
$$
\lambda_1 = \faz Z + 2\faz Z' \quad
\lambda_2 = \lambda_3 = \faz Z - \faz Z'.
$$

Ak ich vložíme do rovnice \citeequation[charvekt] dostaneme podmienky
pre určenie charakteristických vektorov
$$
\hmatrix{
-2 \faz Z' & \faz Z' & \faz Z' \cr 
\faz Z' & -2 \faz Z' & \faz Z' \cr 
\faz Z' & \faz Z' & -2 \faz Z' \cr 
}
\hmatrix{
t_{11} \cr t_{21} \cr t_{31} \cr
}
=
\hmatrix{
0 \cr 0 \cr 0 \cr
}
$$
z~čoho plynie
$$
t_{11} = t_{21} = t_{31} \qquad
t_{12} + t_{22} + t_{32} = 0 \qquad
t_{13} + t_{23} + t_{33} = 0.
$$

\subsection Súmerné zložky v~trojfázovej sústave

Pôvodná sústava fázorov sa rozkladá do troch zložkových sústav:
netočivej (index~0), súslednej (index~1), spätnej (index~2).

Na transformáciu trojfázovej sústavy pre základnú harmonickú potrebujeme
doplniť ešte ďalšie podmienky. Pre fázory napätí platí
\zvyrazni
$$
\faz U_{\rm a} = U_{\rm a} \quad
\faz U_{\rm b} = \faz a^2 U_a \quad
\faz U_{\rm c} = \faz a U_a
$$
kde 
\zvyrazni
$$
\faz a = \exp\left(\j {2 \over 3}{\rm\pi}\right) \quad
\faz a^2 = \exp\left(\j {4 \over 3}{\rm\pi}\right) \quad
1 + \faz a + \faz a^2 = 0.
$$

Ak uvažujeme súmernú zložkovú sústavu,
$$
[\faz U_{\rm o}, \faz U_{\rm m}, \faz U_{\rm n}]^T =
[\faz U_0, \faz U_1, \faz U_2]^T,
$$
potom platí pre súslednú zložku (súmernej sústave po prevode prislúcha
nenulová hodnota iba súslednej zložky)
$$
\hmatrix{
U_a \cr \faz a^2 U_a \cr \faz a U_a \cr
} =
\hmatrix{
t_{11} & t_{12} & t_{13} \cr
t_{21} & t_{22} & t_{23} \cr
t_{31} & t_{32} & t_{33} \cr
}
\hmatrix{
0 \cr U_a \cr 0 \cr
}
$$
z~čoho ihneď plynie
$$
t_{12} = 1 \qquad t_{22} = \faz a^2 \qquad t_{32} = \faz a.
$$

Pre spätnú zložku platí
$$
\hmatrix{
U_a \cr \faz a U_a \cr \faz a U_a^2 \cr
} =
\hmatrix{
t_{11} & t_{12} & t_{13} \cr
t_{21} & t_{22} & t_{23} \cr
t_{31} & t_{32} & t_{33} \cr
}
\hmatrix{
0 \cr 0 \cr U_a \cr
}
$$
z~čoho ihneď plynie
$$
t_{13} = 1 \qquad t_{23} = \faz a \qquad t_{33} = \faz a^2.
$$

Pre netočivú zložku platí
$$
\hmatrix{
U_a \cr U_a \cr U_a \cr
} =
\hmatrix{
t_{11} & t_{12} & t_{13} \cr
t_{21} & t_{22} & t_{23} \cr
t_{31} & t_{32} & t_{33} \cr
}
\hmatrix{
U_a \cr 0 \cr 0 \cr
}
$$
z~čoho ihneď plynie
$$
t_{11} = 1 \qquad t_{21} = 1 \qquad t_{31} = 1.
$$

Prevodnú maticu a maticu k~nej inverznú potom môžme zapísať ako
$$
[T] = 
\hmatrix{
1 & 1 & 1 \cr
1 & \faz a^2 & \faz a \cr
1 & \faz a & \faz a^2 \cr
}\qquad
[T]^{-1} = 
{1 \over 3}
\hmatrix{
1 & 1 & 1 \cr
1 & \faz a & \faz a^2 \cr
1 & \faz a^2 & \faz a \cr
}
$$
a prevod medzi sústavami realizujeme ako
\zvyrazni
$$
\hmatrix{
\faz U_a \cr \faz U_b \cr \faz U_c \cr
} = 
\hmatrix{
1 & 1 & 1 \cr
1 & \faz a^2 & \faz a \cr
1 & \faz a & \faz a^2 \cr
}
\hmatrix{
\faz U_0 \cr \faz U_1 \cr \faz U_2 \cr
}\qquad
%
\hmatrix{
\faz U_0 \cr \faz U_1 \cr \faz U_2 \cr
} =
{1 \over 3}
\hmatrix{
1 & 1 & 1 \cr
1 & \faz a & \faz a^2 \cr
1 & \faz a^2 & \faz a \cr
}
\hmatrix{
\faz U_a \cr \faz U_b \cr \faz U_c \cr
}.
$$

\subsection Diagonálne zložkové sústavy

Pre diagonálne zložkové sústavy platí
$$
[\faz U_{\rm o}, \faz U_{\rm m}, \faz U_{\rm n}] =
[\faz U_0, \faz U_{\alpha}, \faz U_{\beta}].
$$
Tieto sústavy nemajú jednoznačný fyzikálny význam, ale sú vhodné na
riešenie dvojfázových porúch.

Prevodnú maticu a maticu k~nej inverznú zapisujeme ako
$$
[D] = \hmatrix{
1 & 1 & 0 \cr
1 & -{1 \over 2} & {\sqrt{3} \over 2} \cr
1 & -{1 \over 2} & -{\sqrt{3} \over 2} \cr
}
\qquad
[D]^{-1} = {1 \over 3} \hmatrix{
1 & 1 & 1 \cr
2 & -1 & -1 \cr
0 & \sqrt{3} & -\sqrt{3} \cr
}
$$
a prevod medzi sústavami realizujeme ako
$$
\hmatrix{
\faz U_{\rm a} \cr \faz U_{\rm b} \cr \faz U_{\rm c} \cr
}
=
\hmatrix{
1 & 1 & 0 \cr
1 & -{1 \over 2} & {\sqrt{3} \over 2} \cr
1 & -{1 \over 2} & -{\sqrt{3} \over 2} \cr
}
\hmatrix{
\faz U_0 \cr \faz U_{\alpha} \cr \faz U_{\beta} \cr
}
\qquad
\hmatrix{
\faz U_0 \cr \faz U_{\alpha} \cr \faz U_{\beta} \cr
}
={1 \over 3}
\hmatrix{
1 & 1 & 1 \cr
2 & -1 & -1 \cr
0 & \sqrt{3} & -\sqrt{3} \cr
}
\hmatrix{
\faz U_{\rm a} \cr \faz U_{\rm b} \cr \faz U_{\rm c} \cr
}
$$

Aby platila invariantnosť výkonov v~zložkových sústavách, zavádzajú
s~normované zložky
$$
\displaylines{
[T_{\rm n}] = {1 \over \sqrt3} 
\left[\matrix{
1 & 1 & 1 \cr
1 & \faz a^2 & \faz a \cr
1 & \faz a & \faz a^2 \cr
}\right]\qquad
[T_{\rm n}]^{-1} = 
{1 \over \sqrt3}
\left[\matrix{
1 & 1 & 1 \cr
1 & \faz a & \faz a^2 \cr
1 & \faz a^2 & \faz a \cr
}\right]\cr
[D_{\rm n}] = {1 \over \sqrt3}
\left[\matrix{
1 & 1 & 0 \cr
1 & -{1 \over 2} & {\sqrt{3} \over 2} \cr
1 & -{1 \over 2} & -{\sqrt{3} \over 2} \cr
}\right]
\qquad
[D_{\rm n}]^{-1} = {1 \over 3} \left[\matrix{
1 & 1 & 1 \cr
2 & -1 & -1 \cr
0 & \sqrt{3} & -\sqrt{3} \cr
}\right]
}
$$

\subsection Súmerné zložky vyšších harmonických

Kmitočet $k$-tej harmonickej je $k$-krát väčší než základný. Ak je fázor
harmonickej veličiny $\faz N$ pre \hbox{$k$-tú} harmonickú vo fázi \uv{a} rovný
$N_{\rm aks}$, bude matica fázorov príslušná $k$-tej harmonickej
v~symetrickej sústave pre $k$-tu harmonickú v~pôvodnej sústave

$$
\hmatrix{
\faz N_{{\rm a}k{\rm s}} \cr
\faz N_{{\rm b}k{\rm s}} \cr
\faz N_{{\rm c}k{\rm s}} \cr
} = 
\hmatrix{
1 & 0 & 0 \cr
0 & \faz a^2 & 0 \cr
0 & 0 & \faz a \cr
}^k
\hmatrix{
\faz N_{{\rm a}k{\rm s}} \cr
\faz N_{{\rm a}k{\rm s}} \cr
\faz N_{{\rm a}k{\rm s}} \cr
} =
\hmatrix{
\faz N_{{\rm a}k{\rm s}} \cr
\faz a^{2k} \faz N_{{\rm a}k{\rm s}} \cr
\faz a^k \faz N_{{\rm a}k{\rm s}} \cr
}
$$

Hlavnú zložkovú sústavu pre harmonickú typu $k \in \rm N$ tvorí pri

{\advance\leftskip by 2cm
\item{$3k$} netočivá zložková sústava,
\item{$3k + 1$} súsledná zložková sústava,
\item{$3k + 2$} spätná zložková sústava.
\par
}

Pri nesymetriách platí pre diagonálnu maticu
$$
[\faz B] = 
\hmatrix{
1 & 0 & 0 \cr
0 & \faz b & 0 \cr
0 & 0 & \faz c \cr
}
$$
kde
$$
\faz b = {\faz N_{{\rm b}k} \over \faz N_{{\rm b}k{\rm s}}} \qquad
\faz c = {\faz N_{{\rm c}k} \over \faz N_{{\rm c}k{\rm s}}}.
$$

Transformáciu súmernej zložky $\faz N_{0k}$, $\faz N_{1k}$, $\faz
N_{2k}$ príslušnej harmonickej realizujeme ako
$$
\hmatrix{
\faz N_{0k} \cr
\faz N_{1k} \cr
\faz N_{2k} \cr
} = 
{1 \over 3}
\hmatrix{
1 & 1 & 1 \cr
1 & \faz a & \faz a^2 \cr
1 & \faz a^2 & \faz a \cr
}
\hmatrix{
1 & 0 & 0 \cr
0 & \faz b & 0 \cr
0 & 0 & \faz c \cr
}
\hmatrix{
1 & 0 & 0 \cr
0 & \faz a^{2k} & 0 \cr
0 & 0 & \faz a^k \cr
}
\hmatrix{
\faz N_{{\rm a}k{\rm s}} \cr
\faz N_{{\rm a}k{\rm s}} \cr
\faz N_{{\rm a}k{\rm s}} \cr
}
$$


