%{{{
\widowpenalty=10000
\clubpenalty=10000
\raggedbottom

\parskip=0pt
\parindent=1pc
\baselineskip12pt

% makra na zavadzanie fontov [TeXbook naruby] % {{{
\def\setsizes #1 #2 #3 {%
  \def\sizeT{#1}\def\sizeS{#2}\def\sizeSS{#3}}


\def\loadfonts #1 #2#3#4{% fam-t { file1 }{ file2 }{ file3 }
  \expandafter \font\csname#1\sizeT\endcsname=#2
  \expandafter \font\csname#1\sizeS\endcsname=#3
  \expandafter \font\csname#1\sizeSS\endcsname=#4 }

\def\setfamf #1=#2 #3#4{% fam-n = fam-t command size
  \expandafter \ifx\csname#2#4\endcsname \relax
  \else #3#1=\csname#2#4\endcsname \fi}

\def\setonefam #1=#2 {% fam-n = fam-t
  \setfamf#1=#2 \textfont\sizeT \setfamf#1=#2 \scriptfont\sizeS
  \setfamf#1=#2 \scriptscriptfont\sizeSS
  \expandafter \def\csname#2\endcsname{\fam#1%
     \csname#2\sizeT\endcsname}} % např. \def\rm{\fam0\rm10}

\def\setallfams{\setonefam0=rm \setonefam1=mi \setonefam2=sy
\setonefam3=ex \setonefam\itfam=it \setonefam\ttfam=tt
\setonefam\scfam=sc \setonefam\bffam=bf \setonefam\eurmfam=eurm
\setonefam\eurbfam=eurb \setonefam\mibfam=mib \setonefam\bifam=tbi } % }}}

% prevod ciselneho udaju o rodine fontu na sesnastkovu sustavu [tbn]
\def\sixt#1{\ifcase#10\or1\or2\or3\or4\or5\or6\or7\or8\or9\or
    A\or B\or C\or D\or E\or F\fi}

\def\greek#1{% {{{
  \mathchardef\alpha="0\sixt{#1}0B%
  \mathchardef\beta="0\sixt{#1}0C%
  \mathchardef\gamma="0\sixt{#1}0D%
  \mathchardef\delta="0\sixt{#1}0E%
  \mathchardef\epsilon="0\sixt{#1}0F%
  \mathchardef\zeta="0\sixt{#1}10%
  \mathchardef\eta="0\sixt{#1}11%
  \mathchardef\theta="0\sixt{#1}12%
  \mathchardef\iota="0\sixt{#1}13%
  \mathchardef\kappa="0\sixt{#1}14%
  \mathchardef\lambda="0\sixt{#1}15%
  \mathchardef\mu="0\sixt{#1}16%
  \mathchardef\nu="0\sixt{#1}17%
  \mathchardef\xi="0\sixt{#1}18%
  \mathchardef\pi="0\sixt{#1}19%
  \mathchardef\rho="0\sixt{#1}1A%
  \mathchardef\sigma="0\sixt{#1}1B%
  \mathchardef\tau="0\sixt{#1}1C%
  \mathchardef\upsilon="0\sixt{#1}1D%
  \mathchardef\phi="0\sixt{#1}1E%
  \mathchardef\chi="0\sixt{#1}1F%
  \mathchardef\psi="0\sixt{#1}20%
  \mathchardef\omega="0\sixt{#1}21%
  \mathchardef\varepsilon="0\sixt{#1}22%
  \mathchardef\vartheta="0\sixt{#1}23%
  \mathchardef\varpi="0\sixt{#1}24%
  \mathchardef\varphi="0\sixt{#1}27%
  \mathchardef\Gamma="0\sixt{#1}00%
  \mathchardef\Delta="0\sixt{#1}01%
  \mathchardef\Theta="0\sixt{#1}02%
  \mathchardef\Lambda="0\sixt{#1}03%
  \mathchardef\Xi="0\sixt{#1}04%
  \mathchardef\Pi="0\sixt{#1}05%
  \mathchardef\Sigma="0\sixt{#1}06%
  \mathchardef\Upsilon="0\sixt{#1}07%
  \mathchardef\Phi="0\sixt{#1}08%
  \mathchardef\Psi="0\sixt{#1}09%
  \mathchardef\Omega="0\sixt{#1}0A%
	\let\Varrho\Rho%
	\let\Varsigma\Sigma%
}% }}}

\newfam\eurmfam
\newfam\eurbfam
\newfam\mibfam
\newfam\scfam
\newfam\bifam

\setsizes 10 7 5
\loadfonts eurm  {eurm10}   {eurm7}   {eurm5}
\loadfonts eurb  {eurb10}   {eurb7}   {eurb5}
\loadfonts mib   {cmmib10}  {cmmib7}  {cmmib5}
\loadfonts mi    {cmmi10}   {cmmi7}   {cmmi5}
\loadfonts sy    {cmsy10}   {cmsy7}   {cmsy5}
\loadfonts rm    {csr10}    {csr7}    {csr5}
\loadfonts it    {csti10}   {csti7}   {csti5}
\loadfonts bf    {csbx10}   {csbx7}   {csbx5}
\loadfonts tt    {cstt10}   {cstt7}   {cstt5}
\loadfonts sc    {cscsc10}  {cscsc7}  {cscsc5}
\loadfonts tbi   {csbxti10} {csbxti7} {csbxti5}

\def\tenfonts{\setsizes 10 7 5 \setallfams\rm}

\setsizes 8 6 5
\loadfonts eurm  {eurm8}  {eurm6}  {eurm5}
\loadfonts eurb  {eurb8}  {eurb6}  {eurb5}
\loadfonts cmmib {cmmib8} {cmmib6} {cmmib5}
\loadfonts mi    {cmmi8}  {cmmi6}  {cmmi5}
\loadfonts sy    {cmsy8}  {cmsy6}  {cmsy5}
\loadfonts rm    {csr8}   {csr6}   {csr5}
\loadfonts it    {csti8}  {csti6}  {csti5}
\loadfonts bf    {csbx8}  {csbx6}  {csbx5}
\loadfonts tt    {cstt8}  {cstt6}  {cstt5}
\loadfonts sc    {cscsc8} {cscsc6} {cscsc5}
\loadfonts tbi   {csbxti8} {csbxti6} {csbxti5}


\def\eightfonts{\setsizes 8 6 5 \setallfams\rm}

\setsizes 17 9 8
\loadfonts eurm  {eurm9 at 17pt}  {eurm9}  {eurm8}
\loadfonts eurb  {eurb9 at 17pt}  {eurb9}  {eurb8}
\loadfonts cmmib {cmmib9 at 17pt} {cmmib9} {cmmib8}
\loadfonts mi    {cmmi9 at 17pt}  {cmmi9}  {cmmi8}
\loadfonts sy    {cmsy9 at 17pt}  {cmsy9}  {cmsy8}
\loadfonts rm    {csr9 at 17pt}   {csr9}   {csr8}
\loadfonts it    {csti9 at 17pt}  {csti9}  {csti8}
\loadfonts bf    {csbx9 at 17pt}  {csbx9}  {csbx8}
\loadfonts tt    {cstt9 at 17pt}  {cstt9}  {cstt8}
\loadfonts sc    {cscsc9 at 17pt} {cscsc9} {cscsc8}
\loadfonts tbi   {csbxti9 at 17pt} {csbxti9} {csbxti5}

\def\fonttitle{\setsizes 17 9 8 \setallfams\rm}

\setsizes 11 8 6
\loadfonts eurm  {eurm9 at 11pt}   {eurm8}   {eurm6}
\loadfonts eurb  {eurb9 at 11pt}   {eurb8}   {eurb6}
\loadfonts cmmib {cmmib9 at 11pt}  {cmmib8}  {cmmib6}
\loadfonts mi    {cmmi9 at 11pt}   {cmmi8}   {cmmi6}
\loadfonts sy    {cmsy9 at 11pt}   {cmsy8}   {cmsy6}
\loadfonts rm    {csr9 at 11pt}    {csr8}    {csr6}
\loadfonts it    {csti9 at 11pt}   {csti8}   {csti6}
\loadfonts bf    {csbx9 at 11pt}   {csbx8}   {csbx6}
\loadfonts tt    {cstt9 at 11pt}   {cstt8}   {cstt6}
\loadfonts sc    {cscsc9 at 11pt}  {cscsc8}  {cscsc6}
\loadfonts tbi   {csbxti9 at 11pt} {csbxti8} {csbxti6}

\def\fontsection{\setsizes 11 8 6 \setallfams\rm}

\tenfonts

\font\fontsubsection=csbxti10 % podnadpis


\let\oldrm=\rm
\let\oldbf=\bf

\def\rm{\oldrm\greek{\eurmfam}}
\def\bf{\oldbf\greek{\eurbfam}}
\def\bi{\fam\bifam\tbi\greek{\mibfam}}

\newcount\countsection
\newcount\countsubsection
\newcount\countfootnote
\newcount\countfig
\newcount\countequation
\newcount\countbiblio
\newcount\counttable
\newcount\countitem
\newcount\countitemitem

\newread\fileinput 

\newwrite\filefigures % obrazky
\newwrite\fileequations % rovnice
\newwrite\filetables % tabulky

% subor nacitat iba ak existuje [tbn]
\def\softinput #1 {
	\let\next=\relax \openin\fileinput=#1
	\ifeof\fileinput\message{TeX it!}
	\else\closein\fileinput\def\next{\input #1 }\fi
	\next
}

%%%% OBRAZKY

% toto sa bude zapisovat do externeho suboru
\def\figureitem#1#2{\expandafter\def\csname o:#1\endcsname{#2}}


\newtoks\filename \newtoks\label \newtoks\caption
\newtoks\width \newtoks\narrow

\def\beginfigure#1\endfigure{
#1
%Sem vlozim obrazok \the\filename\ so znackou \the\label, popisok bude
%\the\caption.\hfill\break 
%slovo \if:\the\width:Sirka standardna.\else Sirka bude \the\width.\fi
%slovo\hfill\break
%test \if:\the\narrow:Bez zuzenia.\else Zuzenie popisku o \the\narrow.\fi
%tes\hfill\break
	\global\advance\countfig by 1
	\immediate\write\filefigures{\string\figureitem{\the\label}{\the\countfig}}
	\expandafter\ifx\csname o:\the\label\endcsname\relax%
	\expandafter\xdef\csname o:\the\label\endcsname {\the\countfig}%
	\fi%
%%
	\vbox{%
		\vskip\baselineskip%
		\centerline{\pdfximage \if:\the\width:\else width \the\width\fi
			{\the\filename}\pdfrefximage\pdflastximage}%
		\vglue\baselineskip%
		%
		\vbox{\noindent%
		\if:\the\narrow:\narrow={0pt}\fi
			\leftskip=\the\narrow plus1fil%
			\rightskip=\the\narrow plus-1fil%
		\parfillskip=0pt plus2fil \parindent=0pt%
		{\eightfonts Obr.~\the\countfig. \the\caption}%
		}%
		%
		\vskip\baselineskip%
		}%
}

%\beginfigure
%	\filename={usko.jpg}
%	\label={usko}
%	\caption={Macko Usko}
%	\width={4cm}
%	\narrow={1em}
%\endfigure

% pouzitie v texte pri odkazovani sa na obrazky
\def\citefigure[#1]{\expandafter\ifx\csname o:#1\endcsname \relax
	\message{Warning: Undefined fig. number [#1]}Obr.~??%
	\else Obr.~\csname o:#1\endcsname\fi%
}

% nacitanie ulozenych veci z prveho behu a otvorenie suboru
\softinput \jobname.o
\immediate\openout\filefigures=\jobname.o

%%%% TABULKY

% toto sa bude zapisovat do externeho suboru
\def\tableitem#1#2{\expandafter\def\csname t:#1\endcsname{#2}}

% vytvori popisok nad tabulkou a zaroven zapise do suboru
\def\tablelabel[label=#1][caption=#2]{
\vskip\baselineskip
\centerline{%
	\global\advance\counttable by1%
	{\eightfonts Tab. \the\counttable. #2}%
	}\vglue\baselineskip
	\immediate\write\filetables{\string\tableitem{#1}{\the\counttable}}
	\expandafter\ifx\csname t:#1\endcsname\relax%
	\expandafter\xdef\csname t:#1\endcsname {\the\counttable}%
	\fi%
}

% pouzitie v texte pri odkazovani sa na tabulky
\def\citetable[#1]{\expandafter\ifx\csname t:#1\endcsname \relax
	\message{Warning: Undefined table number [#1]}Tab.~??%
	\else Tab.~\csname t:#1\endcsname\fi%
}

% umiestnenie tabulky do stredu
\def\centertable#1{\centerline{#1}\vskip\baselineskip}


% nacitanie ulozenych veci z prveho behu a otvorenie suboru
\softinput \jobname.t
\immediate\openout\filetables=\jobname.t

%%%% ROVNICE

% toto sa bude zapisovat do externeho suboru
\def\equationitem #1#2{\expandafter\ifx\csname eq:#1\endcsname \relax
          \expandafter\def \csname eq:#1\endcsname {#2}
       \else \errmessage{Error: Double eq. mark [#1]}\fi}

% toto treba umiestnit na koniec kazdeho display modu, aby ocisloval
% rovnicu
\def\equation[#1]{\global\advance\countequation by1
   \immediate\write\fileequations{\string\equationitem{#1}{\the\countequation}}
       \ifinner\else \eqno \fi (\the\countequation)
}

% pouzitie v texte pri odkazovani sa na rovnice
\def\citeequation[#1]{\expandafter\ifx\csname eq:#1\endcsname \relax
   \message{Warning: Undefined eq. number [#1]}(??)%
   \else(\csname eq:#1\endcsname)\fi}

% nacitanie ulozenych veci z prveho behu a otvorenie suboru
\softinput \jobname.eq
\immediate\openout\fileequations=\jobname.eq

%%%%%%%%%%%%%%%%%%

\def\starthead{
\global\headline={\eightfonts\ifodd\pageno\bf\hfill\folio\else\bf\folio\hfill\fi}}

\headline={\hfill X15PJE\starthead}
\footline={}

%%%% ODKAZY POD CIAROU

% to co bolo povodne nie prilis vyhovovalo, \footnote aj \vfootnote je
% teda predefinovane podla knizky "Typograficky system TeX" - funguju tu
% dobre vysky riadkov (aj ked je kod vyrazne dlhsi, ale to je detail)

% umiestnime vzdy uplne dole na stranku
\skip\footins=12pt plus 1fill

% bez sadzby odkazu v texte - iba umiestni poznamku do spodnej casti
\def\vfootnote #1#2{%
	\insert\footins{\eightfonts \baselineskip=.\baselineskip%
	\interlinepenalty=\interfootnotelinepenalty%
	\splittopskip=7.5pt%
	\splitmaxdepth=3pt \floatingpenalty=2000%
	\leftskip=0pt \rightskip=0pt%
	\textindent{}\vrule height7.5pt%
	width0pt\relax%
	$^{\eightfonts #1}$#2\vrule depth2.5pt width0pt\par}}

% poznamka aj s odkazom v texte, automaticke zvysovanie ciselnika
\def\footnote #1{\global\advance\countfootnote by 1%
	$^{\eightfonts \the\countfootnote}$%
	\vfootnote{\the\countfootnote}{#1}}%

%%%% ODRAZKY

\let\olditem=\item
\let\olditemitem=\itemitem

\def\beginitemize#1\enditemize{\par
	{{\narrower
	\parskip=3pt
	\def\item{\olditem{$\bullet$}}
	\def\itemitem{\olditemitem{$\circ$}}
	#1\par\vskip\baselineskip
	}}
}

\def\beginnumitemize#1\endnumitemize{\par
	{{\narrower
	\countitem=0
	\parskip=3pt
	\def\item{\global\advance\countitem by 1
		\countitemitem=0
		\olditem{\the\countitem. }}
	\def\itemitem{\global\advance\countitemitem by 1
		\olditemitem{\the\countitem.\the\countitemitem. }}
	#1\par\vskip\baselineskip
	}}
}

% zarovnanie na stred
\def\raggedcenter{%
  \parindent=0pt \rightskip0pt plus1em \leftskip0pt plus1em
  \spaceskip.3333em \xspaceskip.5em \parfillskip=0pt
  \hbadness=10000 % Last line will usually be underfull, so turn off
                  % badness reporting.
}

% hlavný titulok
\def\title#1{
\countfootnote=0
\countfig=0
\countequation=0
\counttable=0
% pripadne vlozenie vakatu
\ifodd\pageno\else\null\vfill\break\fi
\vbox{
\vglue6pc
\baselineskip=1.8\baselineskip\fonttitle\raggedcenter #1}}

\def\section#1#2\par{
\countsubsection=0
\vskip 24pt plus 4pt minus 1pt
\vbox{\raggedcenter{{\fontsection\bf
\if *#1
	#2
\else
	\global\advance\countsection by 1
	\the\countsection.~#1#2
\fi
}}
}
\vglue 12pt plus 1.5pt minus 1pt
}

\def\subsection#1#2\par{
\vskip 12pt plus 2.5pt minus 1pt
{\fontsubsection
\if *#1
	#2
\else
	\global\advance\countsubsection by 1
	\the\countsection.\the\countsubsection.~#1#2
\fi
}
\vglue 6pt plus 1.5pt minus 1pt
}

 % }}}

\def\zv#1#2#3{ % {{{ zvýraznovanie matematických rovníc
\smallskip
\setbox0=\vbox{$$#3$$}
\dimen0=\ht0 \advance\dimen0 by \dp0 \advance\dimen0 by 0.5em
\dimen1=\hsize \advance\dimen1 by 1em
\dimen2=\dimen0 \advance\dimen2 by -0.75em
\line{\hss
      \vbox{\offinterlineskip
            \hbox to 0pt{%
            #1\vrule height\ht0 depth\dp0 width\dimen1 \hss}%
            \vskip-\dimen2%
            \hbox to \hsize{#2\hfill\box0\hfill}}%
      \hss}%
\ifx\Default\undefined\pdfliteral{0 g 0 G}\else\Default\fi
\smallskip\noindent}%

\def\zvyrazni$$#1$${
\zv{\pdfliteral{0.128 0.022 0.00 0.0 k}}
{\pdfliteral{0 0 0 1 k 0 0 0 1 K}}
{#1}%
}

% }}}

% {{{

\input table.tex

\font\cyr=wncyr10

\shyph

\def\j{{\rm j\,}}
\def\d{{\rm d}}
\def\e{{\rm e}}
\let\rho\varrho
\let\epsilon\varepsilon

\def\,{\ifmmode\mskip\thinmuskip\else\leavevmode\thinspace\fi}

\def\az{\discretionary{\kern.3333em až}{}{--}}

\def\faz#1{{\it\hat{\bi#1}}%
}

\def\degree{{}^{\circ}}

\def\pust{{\buildrel\textstyle!\over=}}

\newtoks\tmpa \newtoks\tmpb \newtoks\tmpc


\def\obr#1{\pdfximage {#1} \pdfrefximage\pdflastximage}

\def\hmatrix#1{\left[\matrix{#1}\right]}

% }}}

%%%%%%%%%%%%%%%%%%%%%%%%%%%%%%%%%%%%%%%%%%%%%%%%%%%%%%%%%%%%%%%%%%%%%%%%%

\title{Prechodné javy v~elektroenergetike}

\input 01_parametre.tex
\input 02_vedenia.tex
\input 03_prvky.tex
\input 04_zlozky.tex
\input 05_ustalene.tex
\input 06_zemne.tex
\input 07_skraty.tex

%\input 0_symboly.tex

\headline={}
~\vfill
\centerline{\obr{img/ostatne/zajac.png}}
\vfill

\bye
