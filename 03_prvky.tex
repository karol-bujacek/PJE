\section Tlmivky, kondenzátory a transformátory v~trojfázovej sústave

\subsection Tlmivky

Sú to prístroje s~jediným vinutím v~každej fázi. Spotrebovávajú prevažne
jalový indukčný výkon. Ich spotreba činného výkonu (straty) má byť čo
najmenšia.

\subsection Tlmivky pozdĺžne (sériové) -- reaktory

Používame ich na obmedzenie prúdov pri skratoch za miestom ich
zabudovania na prijateľné hodnoty. Aplikujú sa v~sieťach do 35\,kV, ale
iba tam, kde nie je možné dosiahnuť obmedzenie skratových prúdov inými
prostriedkami. Pretože reaktancie sú relatívne malé, zhotovujú sa
jednotlivé cievky tvoriace tlmivky ako vzduchové. Musia vydržať tepelné a
mechanické namáhanie skratovými prúdmi. Pre $I_{\rm n} \le 200\,\rm A$
sa vyrábajú ako trojfázové, pre väčšie nominálne prúdy ako jednofázové,
obvykle s~betónovými výstužami.

\beginfigure
	\filename={img/tlmivka-seriova.png}
	\label={tlmivka-seriova}
	\caption={Náhradná schéma sériovej tlmivky}
\endfigure

Vždy platí, že $R_{\rm tl} \ll X_{\rm tl}$, takže ak nejde o~výpočet
strát výkonu, rezistanciu tlmivky môžme zanedbať.

Pre výrobu pozdĺžnych tlmiviek sa zadáva percentná reaktancia $X_{\rm
tl\%}$ vztiahnutá na prechádzajúci výkon $S_{\rm tl} = \sqrt{3} U_{\rm
n} I_{\rm n}$, menovité napätie sústavy $U_{\rm n}$ a menovitý prúd
tlmivky $I_{\rm n}$. Reaktanciu určíme podľa vzťahu
\zvyrazni
$$
X_{\rm tl} = {X_{\rm tl\%} U_{\rm n} \over 100 \sqrt{3} I_{\rm n}} =
{X_{\rm tl\%} U_{\rm n}^2 \over 100 S_{\rm tl}}
$$
a úbytok napätia v~bezporuchovom stave určíme podľa vzťahu
$$
\Delta \faz U_{\rm f} = \faz U_{\rm f1} - \faz U_{\rm f2} = (R_{\rm tl}
+ X_{\rm tl}) \faz I = \faz Z_{\rm tl} \faz I.
$$

Pre trojfázovú tlmivku môžeme zapísať rovnice v~maticovom tvare ako
$$
[\faz Z_{\rm tl\,abc}] = [\faz Z_{\rm tl\,012}] = \faz Z_{\rm tl} [E].
$$

Pretože tlmivka zväčšuje úbytok napätia, môže byť v~bezporuchovom stave
premostená, napríklad poistkou. V~prípade poruchy sa poistka preruší a
tlmivka sa uvedie do činnosti.

\subsection Tlmivky priečne (paralelné)

Najznámejšie je použitie v~sústavách s~menovitým napätím $U_{\rm n} \ge
220\,\rm kV$ na kompenzáciu kapacitných (nabíjacích) prúdov vedenia pri
chode naprázdno alebo pri malých zaťaženiach. Pre výrobu je treba zadať
trojfázový menovitý (jalový) výkon $Q_{\rm tl\,n}$ a menovité napätie
$U_{\rm tl\, n}$. Tlmivka má jadro zložené zo železných plechov a
pracuje v~nenasýtenej oblasti magnetizačnej charakteristiky.

Pri zanedbaní rezistancie, čo si môžme dovoliť pretože $R_{\rm tl} \ll
X_{\rm tl}$, platí
\zvyrazni
$$
X_{\rm tl} = {U_{\rm tl\,n} \over \sqrt{3} I_{\rm tl\,n}} =
{U_{\rm tl\,n}^2 \over Q_{\rm tl\,n}}.
$$

\subsection * Tlmivka galvanicky spojená s~vedením

Takto zapojená tlmivka kompenzuje nabíjací výkon vedenia, ku ktorému je
pripojená. Nominálne napätie tlmivky je rovnaké ako nominálne napätie
siete. Uzol vinutia do hviezdy sa spája so zemou iba v~dobe cyklu
opätovného zapínania špeciálnym vypínačom~V. Uzol je chránený proti
prepätiu bleskoistkou~B.

\beginfigure
	\filename={img/tlmivka-paralelna.png}
	\label={tlmivka-paralelna}
	\caption={Základná schéma paralelnej tlmivky galvanicky spojenej
s~vedením}
\endfigure

\subsection * Tlmivka pripojená do terciárneho vinutia transformátoru

Takto zapojená tlmivka môže mať menšie nominálne napätie, ktoré je dané
nominálnym napätím terciárneho vinutia transformátoru, ktoré je od
10\,kV do 35\,kV. Vznikajú však problémy pri vypínaní, pretože vypínač
má vypnúť prakticky čisto induktívnu záťaž.

Vzhľadom na zapojenie do hviezdy s~izolovaným uzlom nemôže tlmivkou
prechádzať prúd netočivej zložkovej sústavy:
$$
\faz Z_{\rm tl1} = \faz Z_{\rm tl2} = \faz Z_{\rm tl} \qquad
\faz Z_{\rm tl0} \to \infty.
$$

\beginfigure
	\filename={img/tlmivka-tercial.png}
	\label={tlmivka-tercial}
	\caption={Základná schéma paralelnej tlmivky zapojenej do terciárneho
	vinutia transformátoru}
\endfigure

\subsection Tlmivky uzlové

V~trojfázových sústavách, ktoré nemajú uzol spojený so zemou priamo, je
možné medzi uzol a zem vložiť impedanciu na kompenzáciu prúdov pri
zemných spojeniach, teda pri poruche izolácie fázy voči zemi. Ak je uzol
spojený so zemou priamo alebo cez rezistanciu, predstavujú poruchy
izolácie voči zemi skraty.

V~prípade zemného spojenia ide o~takmer čisto kapacitný prúd. Ak vložíme
medzi uzol vinutia transformátoru a zem tlmivku, pre ktorú platí $R_{\rm
tl} \ll X_{\rm tl}$, privedieme do miesta poruchy od tlmivky takmer
indukčný prúd.

Ak nastavíme reaktanciu tlmivky tak, aby veľkosť jej indukčného prúdu
bola veľmi podobná kapacitnému prúdu, prúd prechádzajúci miestom poruchy
sa zmenší a spôsobí zhasnutie prípadne existujúceho elektrického oblúku.
Pretože veľkosť kapacitného prúdu záleží na rozsahu siete a teda aj na
zapínaní a vypínaní určitých častí, musíme dokázať meniť reaktanciu
tlmivky. To dosahujeme zmenou vzduchovej medzery jej magnetického obvodu.
Tlmivku nazývame kompenzačná, prípadne zhášacia. 

\beginfigure
	\filename={img/tlmivka-uzlova.png}
	\label={tlmivka-uzlova}
	\caption={Základná schéma uzlovej tlmivky}
\endfigure

Touto tlmivkou pri zložkovom pojatí môže prechádzať iba prúd netočivej
zložkovej sústavy, v~ktorej schémach sa rešpektuje hodnotou $3 X_{\rm
tl} = X_0$, pretože tlmivkou prechádza trojnásobne väčší prúd netočivej
zložky, než v~jednotlivých fázach.

\subsection Kondenzátory

Spravidla nejde o~kondenzátory, ale o~kondenzátorové batérie, ktoré
vzniknú sériovo-paralelným radením kondenzátorov. Straty činného výkonu
nedosahujú ani 0,5\,\% ich menovitého výkonu.

\subsection * Kondenzátory sériové

Používajú sa v~sieťach vn na zlepšenie napäťových pomerov a v~prípade
dlhých vedení vvn na úpravu parametrov. Pre napätie na kondenzátore a
výkon platí
\zvyrazni
$$
\displaylines{
\faz U_{c} = -\j X_c \faz I = -\j {1 \over \omega C} \faz I,\cr
\faz S_{c} = 3 \faz U_c^* \faz I = \j 3 X_c I^2
}
$$
a je vidieť, že napätie a výkon sa mení so zaťažením.

\beginfigure
	\filename={img/kondenzator-seriovy.png}
	\label={kondenzator-seriovy}
	\caption={Základné zapojenie sériového kondenzátoru}
\endfigure

Nevýhodou je, že v~prípade nadprúdov a skratov na nich vzniká prepätie.
Proti ich účinkom chránime sériový kondenzátor špeciálnymi prepäťovými
ochranami s~veľmi rýchlym pôsobením.

Sériový kondenzátor musíme izolovať voči zemi, napríklad umiestnením na
izolačné podpery, alebo ich inštalujeme na plošinách, ktoré sú nesené
závesnými izolátormi.

Nevýhodou sériových kondenzátorov je, že umožňujú prestup prúdov vyšších
harmonických zložiek, čo sa môže prejaviť pri spínaní. 

Taktiež je nimi možné dosiahnuť rozdelenie prúdov na paralelné prenosové
cesty. Zmenou rozdelenia prúdov dosahujeme to, že napríklad nedôjde
k~preťaženiu jednej z~prenosových ciest pri nevyužitej druhej, alebo sa
uskutoční prenos s~najmenšími stratami činného výkonu.

\beginfigure
	\filename={img/seriovy-kondenzator-rozdelenie.png}
	\label={seriovy-kondenzator-rozdelenie}
	\caption={Rozdelenie prúdov na paralelné prenosové cesty pomocou
	sériového kondenzátoru}
\endfigure

\subsection * Kondenzátory paralelné

\beginfigure
	\filename={img/kondenzator-paralelny.png}
	\label={kondenzator-paralelny}
	\caption={Základné zapojenie paralelného kondenzátoru}
\endfigure

Paralelné kondenzátory sa používajú predovšetkým v~priemyslových sieťach
do 1\,kV. Môžu byť zapojené do hviezdy alebo do trojuholníka (v~sieťach
nn). Pri zanedbaní strát činného výkonu a uvážení súmernosti platí pre
jalový výkon odoberaný jednou fázou a trojfázový výkon
\zvyrazni
$$
Q_{\rm f} = U I_c = U^2 \omega C_{\rm\Delta} \quad
Q = 3 U^2 \omega C_{\rm\Delta}
$$
v~prípade zapojenia do trojuholníka a 
\zvyrazni
$$
Q_{\rm f} = U_{\rm f} I_c = U_{\rm f}^2 \omega C_{\rm Y} \quad
Q = 3 U_{\rm f}^2 \omega C_{\rm Y} = U^2 \omega C_{\rm Y}
$$
v~prípade zapojenia do hviezdy. Pri rovnakom jalovom výkone dostávame
vzťah
$$
3 U^2 \omega C_{\rm\Delta} = U^2 \omega C_{\rm Y} \longrightarrow
C_{\rm Y} = 3 C_{\rm\Delta}.
$$

Pri zaraďovaní do hviezdy je potrebná kapacita kondenzátoru trojnásobná,
preto v~prípade sietí do 1\,kV používame zapojenie do trojuholníka všade
tam, kde je to možné.

Pri zanedbaní činného výkonu kondenzátoru odoberá zo siete iba kapacitný
jalový výkon a dodáva do siete indukčný jalový výkon. Túto vlastnosť
používame na kompenzáciu jalového výkonu. Spotrebič odoberá zo siete
výkon $P - \j Q$. Podľa hodnoty $Q_c$ rozlišujeme tri základné prípady:
\beginitemize
	\item $Q_c < Q$ --- podkompenzované, kondenzátor nedodáva celý jalový
	výkon požadovaný spotrebičom,
	\item $Q_c = Q$ --- presná kompenzácia, sieť dodáva iba činný výkon,
	\item $Q_c > Q$ --- prekompenzované, kondenzátor dodá celý jalový
	výkon pre spotrebič, ale aj do siete. 
\enditemize

Kompenzácia je možná individuálna pre každý spotrebič, alebo skupinová
pre niekoľko spotrebičov respektíve pre celý závod. Pri individuálnej
kompenzácii sa kompenzuje najmenší jalový výkon, ktorý spotrebič
požaduje pri svojej prevádzke. Pri skupinovej kompenzácii sú
kondenzátory rozdelené na stupne, ktoré sú podľa potreby pripájané a
odpájané pri zmenách zaťaženia.

\subsection Parametre transformátorov s~dvoma vinutiami

Pri zanedbaní nesúmerností v~priestorovom usporiadaní môžme uvažovať
každú fázu zvlášť. Náhradná schéma sa obvykle realizuje pomocou T
článku.

\beginfigure
	\filename={img/transformator-dvojvinutovy.png}
	\label={transformator-dvojvinutovy}
	\caption={Náhradná schéma transformátoru pomocou T článku}
\endfigure

Pre rozptylové reaktancie platia vzťahy
$$
\faz Z_{\rm \sigma p} = R_{\rm p} + \j X_{\rm \sigma p} \qquad
\faz Z_{\rm \sigma s} = R_{\rm s} + \j X_{\rm \sigma s}
$$
a pre priečnu admitanciu vzťah
$$
\faz Y_{\rm q} = G_{\rm q} - \j B_{\rm q}.
$$
Znamienko mínus vyjadruje, že v priečnej vetve je zapojená indukčnosť,
netreba sa nechať pomýliť podobnosťou s náhradným T článkom, kde priečna
vetva predstavuje kapacitu.

Hodnoty jednotlivých veličín sa zistia výpočtom a overia meraniami
naprázdno a nakrátko na hotovom výrobku. Skúšky poskytnú straty činného
výkonu naprázdno $\Delta P_0$~(W), pomerný prúd naprázdno $i_0$~(\%),
straty činného výkonu nakrátko $\Delta P_{\rm k}$ a  pomerné napätie
nakrátko rovné pomernej impedancii nakrátko $u_{\rm k} = z_{\rm k}$~(\%).

Pre priečnu vetvu platia vzťahy vztiahnuté na menovitý zdanlivý výkon
$S_{\rm n}$ a menovité napätie $U_{\rm n}$
$$
\displaylines{
g_{\rm q} = {\Delta P_0 \over S_{\rm n}} \qquad
y_{\rm q} = {i_{0\%} \over 100} \qquad
b_{\rm q} = \sqrt{y_{\rm q}^2 - g_{\rm q}^2}\cr
\faz y_{\rm q} = g_{\rm q} - \j b_{\rm q} =
{\Delta P_0 \over S_{\rm n}} - \j \sqrt{\left(i_{0\%} \over 100\right)^2 -
\left({\Delta P_0 \over S_{\rm n}}\right)^2}\cr
}
$$
prípadne prevodný vzťah
$$
\faz Y_{\rm q} =
G_{\rm q} - \j B_{\rm q} =
\faz y_q {S_{\rm n} \over U_{\rm n}^2} =
{S_{\rm n} \over U_{\rm n}^2}
\left[
{\Delta P_0 \over S_{\rm n}} - \j \sqrt{\left(i_{0\%} \over 100\right)^2 -
\left({\Delta P_0 \over S_{\rm n}}\right)^2}
\right].
$$

Pre pozdĺžnu vetvu platia vzťahy v~pomerných jednotkách
$$
\displaylines{
r_{\rm k} = {\Delta P_{\rm k} \over S_{\rm n}} \quad
z_{\rm k} = {u_{\rm k\%} \over 100} \quad
x_{\rm k} = \sqrt{z_{\rm k}^2 - r_{\rm k}^2} \cr
\faz z_{\rm k} =
r_{\rm k} + \j z_{\rm k} = 
{\Delta P_{\rm k} \over S_{\rm n}}
+ \j
\sqrt{
\left({u_{\rm k\%} \over 100}\right)^2  +
\left({\Delta P_{\rm k} \over S_{\rm n}}\right)^2
}}
$$
prípadne prevodný vzťah
$$
\faz Z_{\rm k} = R_{\rm k} + \j X_{\rm k} =
\faz z_{\rm k} {U_{\rm n}^2 \over S_n} =
{U_{\rm n}^2 \over S_{\rm n}}
\left[
{\Delta P_{\rm k} \over S_{\rm n}}
+ \j
\sqrt{
\left({u_{\rm k\%} \over 100}\right)^2  +
\left({\Delta P_{\rm k} \over S_{\rm n}}\right)^2}
\right].
$$

Impedanciu $\faz Z_{\rm \sigma ps} = \faz Z_{\rm k} = (R_{\rm p} +
R_{\rm s}) + \j (X_{\rm\sigma p} + X_{\rm\sigma s})$ rozkladáme na dve
časti tak, že kladieme
$$
\faz Z_{\rm\sigma p} =  0{,}5 \faz Z_{\rm\sigma ps} = \faz Z_{\rm\sigma s}.
$$

Fyzikálne toto rozdelenie nie je bez vady -- rozptylové toky a
rezistancie nie sú rovnaké. Pre niektoré úlohy, napríklad pri výpočte
uzlových sietí, nie je vhodné použitie T článku -- vadí nám, že zavádza
ďalší uzol. Preto niekedy používame $\rm\pi$ článok alebo $\rm\Gamma$
článok podľa \citefigure[trafo-pi-gama]. 

\beginfigure
	\filename={img/trafo-pi-gama.png}
	\label={trafo-pi-gama}
	\caption={Príklad náhradnej schémy transformátoru pomocou ${\rm\pi}$ a
	${\rm\Gamma}$ článku}
\endfigure

\subsection Parametre transformátorov s~troma vinutiami

Ide o~transformátory s~troma vinutiami na fázu. Obvykle majú veľký
výkon a môžu byť konštrukčne riešené ako tri jednofázové jednotky. Pre
označenie jednotlivých vinutí -- primárneho, sekundárneho a terciárneho
-- používame indexy p, s, t. Môžu slúžiť na napájanie sústavy z~dvoch
alternátorov, prípadne pre pružné napájanie dvoch sietí s~rôznymi
menovitými napätiami z~jedného alternátoru.

\beginfigure
	\filename={img/transformator-trojvinutovy.png}
	\label={transformator-trojvinutovy}
	\caption={Základná schéma zapojenia s~troma vinutiami na fázu}
\endfigure

\beginfigure
	\filename={img/transformator-trojvinutovy-nahrsch.png}
	\label={transformator-trojvinutovy-nahrsch}
	\caption={Náhradná schéma transformátoru s~troma vinutiami na fázu}
\endfigure

Parametre pre náhradnú schému, ktorá je znázornená na
\citefigure[transformator-trojvinutovy-nahrsch]  určíme výpočtom pri
návrhu a overíme skúškami na hotovom výrobku.

Pri skúške naprázdno určíme straty činného výkonu naprázdno $\Delta P_0$
a prúd naprázdno $i_0$. Určíme priečnu admitanciu vztiahnutú na menovitý
výkon a menovité napätie primárneho vinutia
$$
\faz y_q = g_q - \j b_q = {\Delta P_0 \over S_{\rm p\,n}} -
\j \sqrt{
	\left({i_{0\%} \over 100}\right)^2 -
	\left({\Delta P_0 \over S_{\rm p\,n}}\right)^2
}
$$
prípadne ako
$$
\faz Y_q =
\faz q_z {S_{\rm p\,n} \over U_{\rm p\,n}^2} =
G_q - \j B_q =
{S_{\rm p\,n} \over U_{\rm p\,n}^2}
\left[{\Delta P_0 \over S_{\rm p\,n}} -
\j \sqrt{
	\left({i_{0\%} \over 100}\right)^2 -
	\left({\Delta P_0 \over S_{\rm p\,n}}\right)^2
}
\right]
$$

Pri skúške nakrátko, ktoré sú tri, jedno vinutie napájame, druhé
je skratované a tretie je naprázdno. Pretože menovité výkony
jednotlivých vinutí obecne môžu byť rôzne (napríklad pre napájanie
jednej sústavy z~dvoch alternátorov môže platiť $S_{\rm s\,n} = S_{\rm
t\,n} = 0{,}5 S_{\rm p\,n}$), uskutočňujú sa merania nakrátko pri prúde,
ktorý zodpovedá vinutiu s~menším výkonom a potom sa prepočítavajú na
výkon vyšší.

\tablelabel[label=trafo3-nakratko][caption=Meranie nakrátko pri
predpoklade $S_{\rm p\,n} \ne S_{\rm s\,n} \ne S_{\rm t\,n}$]
\centertable{
\begintable
\begintableformat
\center " \center " \center " \center
\endtableformat
\-
\br{\:|} merané medzi | p--s | p--t | s--t \er{|}
\-
\br{\:|} straty nakrátko (W) | $\Delta P_{\rm k\,ps}$ |  $\Delta P_{\rm k\,pt}$ | $\Delta P_{\rm k\,st}$ \er{|}
\br{\:|} napätie nakrátko (\%) | $u_{\rm k\,ps}$ | $u_{\rm k\,pt}$ | $u_{\rm k\,st}$ \er{|}
\br{\:|} straty nakrátko (W) | $S_{\rm s\,n}$ |  $S_{\rm t\,n}$ | $S_{\rm t\,n}$ \er{|}
\-
\endtable
}

Nasledovný rozbor platí pre skúšku s--t. Má sa zistiť impedancia
nakrátko
$$
\faz Z_{\rm st} = \faz Z_{\rm\sigma s} + \faz Z_{\rm\sigma t}
\hbox{ pričom }
\faz Z_{\rm\sigma s} = R_{\rm s} + \j X_{\rm\sigma s}.
$$

Straty nakrátko zistíme pri menovitom prúde $I_{\rm t\,n}$ ako
$$
\Delta P_{\rm k\,st} = 3R_{\rm st}^+  I_{\rm t\,n}^2
$$
kde $R_{\rm st}^+$ je rezistancia sekundárneho a terciárneho vinutia
vztiahnutá na $U_{\rm n\,t}$. Po dosadení za prúd $I_{\rm t\,n}$
dostaneme
$$
\Delta P_{\rm k\,st} =
3R_{\rm st}^+  
\left({S_{\rm t\,n} \over \sqrt{3} U_{\rm t\,n}}\right)^2
\longrightarrow
R_{\rm st}^+ = {\Delta P_{\rm k\,st} \over S_{\rm t\,n}^2} U_{\rm
t\,n}^2.
$$
Pri prevedení v~pomere kvadrátov napätí z~terciárneho na primárne
vinutie dostávame
$$
R_{\rm st} = R_{\rm st}^+ {U_{\rm p\,n}^2 \over U_{\rm t\,n}^2}
\longrightarrow
R_{\rm st} = R_{\rm s} + R_{\rm t} =
{\Delta P_{\rm k\,st} \over S_{\rm t\,n}^2} U_{\rm p\,n}^2
$$
kde $R_{\rm s}$ respektíve $R_{\rm t}$ je rezistancia sekundárneho
respektíve terciárneho vinutia prepočítaná na primárne vinutie.

Impedancia nakrátko v~percentách zodpovedá napätiu nakrátko $u_{\rm
k\,st}$ v~percentách. Pomerné impedanciu nakrátko sekundárneho a
terciárneho vinutia, vztiahnutú na $S_{\rm p\,n}$ a $U_{\rm p\,n}$ a
k~nej zodpovedajúcu impedanciu nakrátko prevedená na menovité napätie
primárneho vinutia vyjadríme ako
$$
z_{\rm st} = {u_{\rm k\,st\%} \over 100}{S_{\rm p\,n} \over S_{\rm t\,n}}
\qquad
Z_{\rm st} = z_{\rm st} {U_{\rm p\,n}^2 \over S_{\rm p\,n}} =
{u_{\rm k\,st\%} \over 100}{U_{\rm p\,n}^2 \over S_{\rm t\,n}}
$$
a platí
$$
\displaylines{
\faz z_{\rm st} = r_{\rm st} + \j x_{\rm st} \quad
x_{\rm st} = \sqrt{z_{\rm st}^2 - r_{\rm st}^2} \quad
x_{\rm st} = x_{\rm\sigma s} + x_{\rm\sigma t},\cr
\faz Z_{\rm st} = R_{\rm st} + \j X_{\rm st} \quad
X_{\rm st} = \sqrt{Z_{\rm st}^2 - R_{\rm st}^2} \quad
X_{\rm st} = X_{\rm\sigma s} + X_{\rm\sigma t}.\cr
}
$$

Po dosadení môžme pre všetky tri merania nakrátko písať vzťahy typu
$$
\faz z_{\rm st} = r_{\rm st} + \j x_{\rm st},
$$
kam postupne dosádzame indexy ps, pt, st.

Znalosť impedancie nakrátko umožňuje stanoviť rezistancie a rozptylové
reaktancie jednotlivých vinutí, ktoré sú tvorené zložkami rozptýlenej
reaktancie, ako
$$
\displaylines{
\faz Z_{\rm\sigma p} = R_{\rm p} + \j X_{\rm\sigma p} =
0{,}5 (\faz Z_{\rm ps} + \faz Z_{\rm pt} + \faz Z_{\rm st})\cr
\faz Z_{\rm\sigma s} = R_{\rm s} + \j X_{\rm\sigma s} =
0{,}5 (\faz Z_{\rm ps} + \faz Z_{\rm st} + \faz Z_{\rm pt})\cr
\faz Z_{\rm\sigma t} = R_{\rm t} + \j X_{\rm\sigma t} =
0{,}5 (\faz Z_{\rm pt} + \faz Z_{\rm st} + \faz Z_{\rm ps})\cr
}
$$

Znalosť pozdĺžnych impedancií a priečnych admitancií umožňuje sledovať
napäťové a výkonové pomery trojvinuťových transformátorov.



