\section Vedenia s~rovnomerne rozdelenými parametrami

\subsection Náhradná schéma a vstupná impedancia dvojvodičového vedenia
nekonečne dlhého

Vedenie, tvorené dvoma paralelnými priamkovými vodičmi dĺžky $\l \to
\infty$ vo vzdialenosti $d \ll l$, nahradíme úsekmi rovnakej dĺžky
$\Delta x$ s~pozdĺžnou impedanciou $\faz Z_{\rm le}$ a priečnou
admitanciou $\faz Y_{\rm qe}$:
$$
\displaylines{
\faz Z_{\rm le} = (R_1 + \j \omega L_1)\Delta x = \faz Z_{\rm l1}\Delta x,\cr
\faz Y_{\rm qe} = (G_1 + \j \omega C_1)\Delta x = \faz Y_{\rm q1}\Delta x.\cr
}
$$

\beginfigure
	\filename={img/vedenia-nahrada.png}
	\label={vedenia-nahrada}
	\caption={Náhradná schéma vedenia s~rozloženými parametrami}
\endfigure

Pretože ide o~vedenie neobmedzenej dĺžky, môžme náhradnú schému zobraziť
tak ako je uvedené na \citefigure[vedenia-nahradna-zjednodusenie].

\beginfigure
	\filename={img/vedenia-nahradna-zjednodusenie.png}
	\label={vedenia-nahradna-zjednodusenie}
	\caption={Zjednodušenie náhradnej schémy pre vedenie neobmedzenej
	dĺžky}
\endfigure


Pre vstupnú impedanciu dostávame vzťah
$$
\faz Z_{\rm v} =
\faz Z_{\rm le} +
{\faz Z_{\rm v} \faz Y_{\rm qe}^{-1} \over
\faz Z_{\rm v} + \faz Y_{\rm qe}^{-1}}
$$
a po formálnych úpravách ho môžme písať ako
$$
\faz Z_{\rm v}^2 - \faz Z_{\rm le} \faz Z_{\rm v} - \faz Z_{\rm le} \faz
Y_{\rm qe}^{-1} = 0.
$$

Spojitému rozdeleniu parametrov sa priblížime pre $\Delta x \to 0$. Ak
dosadíme za $\faz Z_{\rm le}$ a $\faz Y_{\rm qe}$, dostaneme
$$
\faz Z_{\rm v} =
\lim_{\Delta x \to 0} \left[
	{\faz Z_{\rm l1} \over 2} \Delta x \pm
	\sqrt{
		\left[{\faz Z_{\rm l1} \over 2}\Delta x\right]^2
		+ {\faz Z_{\rm l1} \over \faz Y_{\rm q1}}
	}
\right]
%
\longrightarrow
\faz Z_{\rm v} =
\sqrt{\faz Z_{\rm l1} \over \faz Y_{\rm q1}} =
\faz Y_{\rm v}^{-1}.
$$

Vstupná impedancia nekonečne dlhého vedenia sa nazýva vlnová impedancia
$\faz Z_{\rm v}$ (${\rm \Omega}$). Jej reciprokou hodnotou $\faz Z_{\rm
v}^{-1} = \faz Y_{\rm v}$ je vlnová admitancia.

\subsection Základné rovnice pre priestorové rozloženie napätia a prúdu

\beginfigure
	\filename={img/vedenie-rozlozene.png}
	\label={vedenie-rozlozene}
	\caption={Prvok vedenia s~rozloženými parametrami}
\endfigure

Prvok vedenia s~rovnomerne rozloženými parametrami dĺžky $\d x$ možno
popísať dvojbranom podľa \citefigure[vedenie-rozlozene]. Pre
prvok vo vzdialenosti $x$ od počiatku možno napísať podľa Kirchhoffových
zákonov vzťahy
$$
\displaylines{
- \faz U(x) + \faz Z_{\rm l1} \faz I(x) \d x +
\left[\faz U(x) + {\d \faz U(x) \over \d x} \d x\right] = 0,\cr
\faz I(x) - {\d \faz I(x) \over \d x}\d x = \faz I(x) + \faz U(x)\faz
Y_{\rm q1} \d x\cr
}
$$
z~čoho po úpravách dostávame
$$
\displaylines{
-{\d \faz U(x) \over \d x} = \faz Z_{\rm l1} \faz I(x),\cr
-{\d \faz I(x) \over \d x} = \faz Y_{\rm q1} \faz U(x).\cr
}
$$

Ak jednu z~rovníc derivujeme a dosadíme do druhej, dostaneme telegrafné
rovnice pre ustálený chod (uvažujeme sínusový priebeh)
$$
{\d^2 \faz U(x) \over \d x^2} = \faz \gamma^2 \faz U(x)\qquad
{\d^2 \faz I(x) \over \d x^2} = \faz \gamma^2 \faz I(x)
$$
kde konštanta prenosu je rovná
$$
\faz \gamma = \sqrt{\faz Z_{l1} \faz Y_{q1}}.
$$

Riešením telegrafných rovníc získame rovnice pre napätie a prúd
$$
\eqalign{
\faz U(x)&=\faz K_1 \exp(-\faz\gamma x) + \faz K_2 \exp(\faz\gamma x),\cr
\faz I(x)&=\faz K_1 \faz Y_{\rm v} \exp(-\faz\gamma x) - \faz K_2 \faz
Y_{\rm v} \exp(\faz\gamma x).\cr
}
$$
Komplexné konštanty $\faz K_1$ a $\faz K_2$ určíme z~okrajových
podmienok. Pre vedenie konečnej dĺžky $l$, teda $x \in \langle0,
l\rangle$, sú spravidla zadané veličiny napätí a prúd buď na začiatku
(pre $x = 0$, platí index 1) alebo na konci (pre $x = l$, platí index
2). Pomocou nich vyjadríme konštanty $\faz K_1$ a $\faz K_2$.

Pre známe pomery na začiatku vedenia, $\faz U(0) = \faz U_1$, $\faz I(0)
= \faz I_1$, dostávame
$$
\eqalign{
\faz U_1 = \faz K_1 + \faz K_2 &\quad
\faz I_1 = \faz K_1 \faz Y_{\rm v} - \faz K_2 \faz Y_{\rm v} \cr
\faz K_1 = {1 \over 2} (\faz U_1 + \faz Z_{\rm v}\faz I_1) &\quad
\faz K_2 = {1 \over 2} (\faz U_1 - \faz Z_{\rm v}\faz I_1)\cr
}
$$


Ak uvážime vzťahy
$$
{1 \over 2} \big(\exp(-\faz\gamma x) + \exp(\faz\gamma x)\big) =
\cosh(\faz\gamma x) \quad
{1 \over 2} \big(\exp(\faz\gamma x) - \exp(-\faz\gamma x)\big) =
\sinh(\faz\gamma x)
$$
a upravíme predošlé rovnice, získame vzťahy pre napätie a prúd
v~ľubovoľnom mieste vedenia $x$:
$$
\eqalign{
\faz U(x) &= \faz U_1 \cosh(\faz\gamma x) - \faz I_1\faz Z_{\rm v}
\sinh(\faz\gamma x)\cr 
\faz I(x) &= - \faz U_1 \faz Y_{\rm v} \sinh(\faz\gamma x) + \faz I_1
\cosh(\faz\gamma x)\cr
\faz U(x) &= \faz D(x) \faz U_1 - \faz B(x) \faz I_1\cr
\faz I(x) &= - \faz C(x) \faz U_1 + \faz A(x) \faz I_1\cr
}
$$

Hodnoty $\faz A(x)$, $\faz B(x)$, $\faz C(x)$ a $\faz D(x)$ nazývame
Blondelove konštanty a platí pre ne rovnosť
$$
\faz A(x) \faz D(x) - \faz B(x) \faz C(x) = 1.
$$

Pre známe pomery na konci vedenia, $\faz U(l) = \faz U_2$, $\faz I(l) =
\faz I_2$, hľadáme
$$
\faz U(x) = f_3(\faz U_2, \faz I_2) \quad
\faz I(x) = f_4(\faz U_2, \faz I_2).
$$
Pre $x = l$ dostávame vzťahy
$$
\faz U_2 =
\faz K_1 \exp(-\faz\gamma l) +
\faz K_2 \exp(\faz\gamma l)
\quad
\faz I_2 =
\faz Y_{\rm v} \faz K_1 \exp(-\gamma l) -
\faz Y_{\rm v} \faz K_2 \exp(\faz\gamma l)
$$
a odtiaľ určíme $\faz K_1$ a $\faz K_2$ ako
$$
\eqalign{
\faz K_1 &= {1 \over 2} \exp(\faz\gamma l) (\faz U_2 + \faz Z_{\rm v} \faz I_2)\cr
\faz K_2 &= {1 \over 2} \exp(-\faz\gamma l) (\faz U_2 - \faz Z_{\rm v} \faz I_2)\cr
}
$$

Rovnice pre napätie a prúd v~ľubovoľnom mieste $x$ vyjadríme ako
$$
\eqalign{
\faz U(x) &= \faz U_2 \cosh(\faz \gamma(l-x))v + \faz I_2 \faz Z_{\rm v}
\sinh(\gamma(l-x))\cr
\faz I(x) &= \faz U_2 \faz Y_{\rm v} \sinh(\gamma(l-x)) + \faz I_2
\cosh(\gamma(l-x))\cr
\faz U(x) &= \faz A(l-x) \faz U_2 - \faz B(l-x) \faz I_2\cr
\faz I(x) &= \faz C(l-x) \faz U_2 + \faz D(l-x) \faz I_2\cr
}
$$

\subsection Ideálne bezstratové vedenie

Pre dlhé vedenia vvn a zvn platí obvykle $R_1 \ll \omega L_1$ a $G_1 \ll
\omega C_1$, čo umožňuje pre niektoré úvahy zanedbať $R_1$ a $G_1$.
Vytvárame tak pomyselné vedenie, ktoré sa používa ako približná náhrada
za skutočné vedenie, pretože vzťahy sú jednoduchšie a je možné
jednoduchšie formulovať približne platné závery. Definičné rovnice
ideálneho vedenia sú
$$
R_1 = 0, G_1 = 0 \longrightarrow \faz Z_{\rm l1i} = \j X_{\rm l1}, \faz
Y_{\rm q1i} = \j B_{\rm q1}.
$$

Pre konštantu prenosu potom platí vzťah
$$
\faz \gamma_{\rm i} = \j \omega \sqrt{L_1 C_1}
$$
a pre vlnovú impedanciu vzťah
$$
\faz Z_{\rm vi} = \sqrt{{L_1 \over C_1}}.
$$

Rovnice pre napätie a prúd v~mieste $x$ pre známe hodnoty na začiatku,
respektíve na konci, vedenia zapíšeme v~maticovom tvare ako
$$
\eqalign{
\left[\matrix{
\faz U_{\rm f}(x) \cr
\faz I(x) \cr
}\right]
&=
\left[\matrix{
\cos(\beta_i x) & - \j Z_{\rm vi} \sin(\beta_i x) \cr
-\j Y_{\rm vi} \sin(\beta_i x) & \cos(\beta_i x)\cr
}\right]
\left[\matrix{
\faz U_{\rm f1} \cr
\faz I_1 \cr
}\right]\cr
\left[\matrix{
\faz U_{\rm f}(x) \cr
\faz I(x) \cr
}\right]
&=
\left[\matrix{
\cos(\beta_i (l-x)) & \j Z_{\rm vi} \sin(\beta_i (l-x)) \cr
\j Y_{\rm vi} \sin(\beta_i (l-x)) & \cos(\beta_i (l-x)) \cr
}\right]
\left[\matrix{
\faz U_{\rm f1} \cr
\faz I_1 \cr
}\right]\cr
}
$$

\section Náhradné dvojbrany prvkov rozvodnej sústavy

Ak nie je potrebné vyšetrovať podrobnejšie pomery vo vnútri prvku
sústavy, postačí zistiť pre výpočet ustáleného chodu vzájomné vzťahy
medzi veličinami na vstupe a na výstupe, ktoré možno zapísať v~maticovom
tvare.

\beginfigure
	\filename={img/dvojbran.png}
	\label={dvojbran}
	\caption={Lineárny neautonómny dvojbran}
\endfigure

Pre zadané veličiny na konci vedenia platí 
\zvyrazni
$$
\left[\matrix{
\faz U_{\rm f1} \cr
\faz I_1\cr
}\right]
= \left[\matrix{
\faz A & \faz B \cr
\faz C & \faz D \cr
}\right]
\left[\matrix{
\faz U_{\rm f2} \cr
\faz I_2 \cr
}\right]
$$
a pre zadané veličiny na začiatku vedenia
\zvyrazni
$$\left[\matrix{
\faz U_{\rm f2} \cr
\faz I_2\cr
}\right]
= \left[\matrix{
\faz D & - \faz B \cr
- \faz C & \faz A \cr
}\right]
\left[\matrix{
\faz U_{\rm f1} \cr
\faz I_1 \cr
}\right].
$$
Pre konštanty platí v~prípade pasívneho dvojbranu vzťah
\zvyrazni
$$
\faz A \faz D - \faz B \faz C = 1
$$
a v~prípade symetrického dvojbranu vzťah
\zvyrazni
$$
\faz A = \faz D.
$$

\subsection Dvojbran v~tvare ${\bf\pi}$ článku

Patrí k~najpoužívanejším v~elektroenergetike. V~pozdĺžnej vetvi má
impedanciu $\faz Z_{\rm l}$, v~priečnych vetvách admitancie $\faz
Y_{{\rm q}a}$ a
$\faz Y_{{\rm q}b}$. V~prípade symetrického článku platí
$$
\faz Y_{{\rm q}a} = \faz Y_{{\rm q}b} = 0{,}5 \faz Y_{\rm q}.
$$
Používa sa najmä ako náhradná schéma vonkajších vedení do dĺžky asi
250\,km a káblových vedení do asi 100\,km.

\beginfigure
	\filename={img/pi-clanok.png}
	\label={pi-clanok}
	\caption={Náhradný ${\rm\pi}$ článok}
\endfigure

Základné rovnice sú podľa Kirchhoffovych zákonov
$$
\displaylines{
-\faz U_{\rm f1} + \faz Z_{\rm l} (\faz I_2 + \faz Y_{{\rm q}b} \faz U_{\rm f2}) +
\faz U_{\rm f2} = 0
\longrightarrow \faz U_{\rm f1} = (1 + \faz Z_{\rm l} \faz Y_{{\rm q}b}) \faz U_{\rm
f2} + \faz Z_{\rm l} \faz I_2,\cr
\faz I_1 = \faz Y_{{\rm q}a} \faz U_{\rm f1} + \faz Y_{{\rm q}b} \faz U_{\rm f2} +
\faz I_2,\cr
\faz I_1 = (\faz Y_{{\rm q}a} + \faz Y_{{\rm q}b} + \faz Y_{{\rm
q}a}\faz Y_{{\rm q}b}\faz
Z_l)\faz U_{\rm f2} + (1+\faz Z_{\rm l} \faz Y_{{\rm q}a})\faz I_2.\cr
}
$$

Pre obecný, nesymetrický, ${\rm\pi}$ článok sú Blondelove konštanty rovné
$$
\displaylines{
\faz A_{\rm\pi n} = 1 + \faz Z_{\rm l} \faz Y_{{\rm q}b}, \cr
\faz B_{\rm\pi n} = \faz Z_{\rm l}, \cr
\faz C_{\rm\pi n} = \faz Y_{{\rm q}a} + \faz Y_{{\rm q}b} + \faz Y_{{\rm
q}a} \faz Y_{{\rm q}b}
\faz Z_l, \cr
\faz D_{\rm\pi n} = 1 + \faz Z_{\rm l} \faz Y_{{\rm q}a} \cr
}
$$
a pre symetrický bude platiť
$$
\displaylines{
\faz A_{\rm\pi s} = \faz D_{\rm\pi s} = 1 + 0{,}5 \faz Z_{\rm l} \faz
Y_{\rm q}, \cr
\faz B_{\rm\pi s} = \faz Z_{\rm l}, \cr
\faz C_{\rm\pi s} = \faz Y_{\rm q} + 0{,}25 \faz Z_{\rm l} \faz Y_{\rm
q}^2.\cr
}
$$

\subsection * Grafické riešenie ${\bf\pi}$ článku

\beginfigure
	\filename={img/pi-clanok-graficky.png}
	\label={pi-clanok-graficky}
	\caption={Schéma pre grafické riešenie ${\rm\pi}$ článku}
\endfigure

Dvojbran na \citefigure[pi-clanok-graficky], zadaný svojimi prvkami, je
na výstupe zaťažený záťažou kapacitného charakteru. Grafický výpočet
potom vykonáme tak, ako je naznačené na obrázku
\citefigure[pi-clanok-fazdiag].

\beginfigure
	\filename={img/pi-clanok-fazdiag.png}
	\label={pi-clanok-fazdiag}
	\caption={Fázorový diagram ${\rm\pi}$ článku}
\endfigure


\subsection Dvojbran v~tvare T článku

T článok je menej používaný než $\rm\pi$ článok. Zavádza do riešenia
ďalší uzol a to je často nevhodné. Je možné ho použiť ako náhradnú
schému pre vonkajšie vedenia do dĺžky 200\,km a káblové do 80\,km. Časté
je použitie pre trojfázové transformátory s~dvoma vinutiami.

Ak nahrádza vedenie, tak platí, že
$$
\faz Z_{{\rm l}a} =
\faz Z_{{\rm l}b} =
0{,}5 \faz Z_{{\rm l}1} l \quad
\faz Y_{\rm q} = \faz Y_{{\rm q}1} l.
$$

\beginfigure
	\filename={img/t-clanok.png}
	\label={t-clanok}
	\caption={Náhradný T článok}
\endfigure

Pre tento článok platia základné rovnice pre prúdy z~Kirchhoffovych zákonov
$$
\eqalign{
\faz I_1 - \faz Y_{\rm q} (\faz U_{\rm f2} + \faz Z_{{\rm l}b} \faz I_2) - \faz I_2 = 0
\longrightarrow
&\faz I_1 = \faz Y_{\rm q} \faz U_{\rm f2} + (\faz Z_{{\rm l}b} \faz
Y_{\rm q} + 1)\faz I_2\cr
&\faz I_1 = \faz C \faz U_{\rm f2} + \faz D \faz I_2 \cr
}
$$
a pre napätie dostávame rovnicu
$$
-\faz U_{\rm f1} + \faz Z_{{\rm l}a} \faz I_1 + \faz Z_{{\rm l}b} \faz I_2 + \faz
U_{\rm f2} = 0
$$
do ktorej môžme dosadiť za $\faz I_1$ a dostaneme rovnicu
$$
\eqalign{
\faz U_{\rm f1} &= (1 + \faz Z_{{\rm l}a} \faz Y_{\rm q} ) \faz U_{\rm f2} + (\faz
Z_{{\rm l}a} + \faz Z_{{\rm l}b} +\faz Z_{la} \faz Z_{lb} \faz Y_q) \faz I_2\cr
\faz U_{\rm f1} &= \faz A \faz U_{\rm f2} + \faz B \faz I_2 \cr
}
$$

Pre obecný, nesymetrický, T článok sú Blondelove konštanty rovné
$$
\displaylines{
\faz A_{\rm Tn} = 1 + \faz Z_{la} \faz Y_q, \cr
\faz B_{\rm Tn} = \faz Z_{la} + \faz Z_{lb} + \faz Z_{la} \faz Z_{lb} \faz Y_q, \cr
\faz C_{\rm Tn} = \faz Y_q, \cr
\faz D_{\rm Tn} = 1 + \faz Z_{lb} \faz Y_q \cr
}
$$
a pre symetrický bude platiť
$$
\displaylines{
\faz A_{\rm Ts} = \faz D_{\rm Ts} = 1 + 0{,}5 \faz Z_l \faz Y_q, \cr
\faz B_{\rm Ts} = \faz Z_l + 0{,}25 \faz Z_l^2 \faz Y_q, \cr
\faz C_{\rm Ts} = \faz Y_q. \cr
}
$$

\subsection * Grafické riešenie T článku

\beginfigure
	\filename={img/t-clanok-graficky.png}
	\label={t-clanok-graficky}
	\caption={Schéma pre grafické riešenie T článku}
\endfigure

Dvojbran na \citefigure[t-clanok-graficky], zadaný svojimi prvkami, je
na výstupe zaťažený 
záťažou indukčného charakteru. Grafický výpočet potom vykonáme tak, ako
je naznačené na \citefigure[t-clanok-fazdiag].

\beginfigure
	\filename={img/t-clanok-fazdiag.png}
	\label={t-clanok-fazdiag}
	\caption={Fázorový diagram T článku}
\endfigure

\subsection Článok $\bf \Gamma$ priečnym prvkom na vstupe a na výstupe

Použitie týchto článkov je pomerne malé. Možno ich využiť ako náhradnú
schému kratších vedení (vonkajšie asi do 80\,km a káblové asi do 25\,km)
a transformátorov, kde sa im dáva prednosť pretože nezavádzajú ďalší
uzol.

\beginfigure
	\filename={img/gama-clanok-vstup.png}
	\label={gama-clanok-vstup}
	\caption={Náhradný ${\rm\Gamma}$ článok s~priečnym prvkom na vstupe}
\endfigure

Pri zapojení s~priečnym prvkom na vstupe, podľa
\citefigure[gama-clanok-vstup] dostaneme rovnice
$$
\eqalign{
\faz U_{\rm f1} = \faz U_{\rm f2} + \faz Z_{\rm l} \faz I_2 \qquad &\sim \qquad \faz U_{\rm
f1} = \faz A \faz U_{\rm f2} + \faz B \faz I_2 \cr
\faz I_1 = \faz Y_{\rm q} \faz U_{\rm f2} + (1 + \faz Z_{\rm l} \faz Y_{\rm q}) \faz I_2
\qquad &\sim \qquad \faz I_1 = \faz C \faz U_{\rm f2} + \faz D \faz I_2 \cr
}
$$
a pre Blondelove konštanty platí
$$
\displaylines{
\faz A_{\rm\Gamma} = 1 \cr
\faz B_{\rm\Gamma} = \faz Z_{\rm l} \cr
\faz C_{\rm\Gamma} = \faz Y_{\rm q} \cr
\faz D_{\rm\Gamma} = 1 + \faz Z_{\rm l} \faz Y_{\rm q} \cr
}
$$

\beginfigure
	\filename={img/gama-clanok-vystup.png}
	\label={gama-clanok-vystup}
	\caption={Náhradný ${\rm\Gamma}$ článok s~priečnym prvkom na výstupe}
\endfigure

Pri zapojení s~priečnym prvkom na výstupe, podľa
\citefigure[gama-clanok-vystup] dostaneme rovnice
$$
\eqalign{
0 = -\faz U_{\rm f1} + \faz Z_{\rm l} (\faz Y_{\rm q} \faz U_{\rm f2} + \faz I_2) + \faz
U_{\rm f2} \quad &\cr
\faz U_{\rm f1} = (1 + \faz Z_{\rm l} \faz Y_{\rm q})
\faz U_{\rm f2} + \faz Z_{\rm l} \faz I_2 \quad &\sim \quad
\faz U_{\rm f1} = (1 + \faz Z_{\rm l} \faz Y_{\rm q}) \faz U_{\rm f2} + \faz Z_{\rm l}
\faz I_2 \cr
\faz I_1 = \faz Y_{\rm q} \faz U_{\rm f2} + \faz I_2 \quad &\sim \quad
\faz I_1 = \faz C \faz U_{\rm f2} + \faz I_2 \cr
}
$$
a pre Blondelove konštanty platí
$$
\displaylines{
\faz A_{\rm\Gamma} = 1 + \faz Z_{\rm l} \faz Y_{\rm q} \cr
\faz B_{\rm\Gamma} = \faz Z_{\rm l} \cr
\faz C_{\rm\Gamma} = \faz Y_{\rm q} \cr
\faz D_{\rm\Gamma} = 1 \cr
}
$$
