\section Skraty

Skraty predstavujú najrozšírenejšie poruchy v~elektrizačnej sústave.
Skrat vznikne poruchovým spojením fáz navzájom, alebo spojením so zemou
v~sústave s~uzemneným uzlom.


Pri skrate sa celková impedancia skratom postihnutej oblasti siete
zmenšuje a zväčšujú sa prúdy. To vedie k~zníženiu napätí v~miestach
blízkych skratu. Obvykle v~mieste skratu vznikajú prechodové
odpory (odpor elektrického oblúku).

\subsection Druhy skratov

V~prípade vonkajších vedení sú trojfázové skraty málo časté, zato
v~prípade káblových vedení sú časté kvôli tomu, že pôsobením oblúka
prechádzajú ostatné typy skratov na trojfázový.

\tablelabel[label=skraty][caption=Druhy skratov] 
\centertable{
\begintable
\begintableformat &\center \endtableformat
\-
\br{\:|} | | \use{3} Pravdepodobnosť výskytu (\%) \er{|}
\br{|} \zb{Druh skratu} | \zb{Schéma} | \use{3} \- \er{|}
\br{\:|} | | vn | 110\,kV | 220\,kV \er{|}
\fulltablerule{3\tr}
\br{\:|} trojfázový | \obr{img/skrat-trojfazovy.png} | 5 | 0,6 | 0,9 \er{|}
\-
\br{\:|} dvojfázový | \obr{img/skrat-dvojfazovy.png}| 10 | 4,8 | 0,6 \er{|}
\-
\br{\:|} dvojfázový zemný | \obr{img/skrat-dvojfazovy-zemny.png} | 20 | 3,8 | 5,4 \er{|}
\-
\br{\:|} jednofázový | \obr{img/skrat-jednofazovy.png} | * | 91 | 93,1 \er{|}
\-
\endtable
}


\subsection Časové priebehy skratových prúdov

\subsection * Súmerný skratový prúd

Predpokladáme, že pred skratom bola sústava v~chode naprázdno, že
činné odpory sú zanedbateľné a skratový prúd obmedzujú iba reaktancie.
Skratový prúd má predovšetkým indukčný charakter a je oneskorený za
napätím o~$\rm\pi/2$. Nastáva vtedy, ak ku skratu dochádza v~okamihu
prechodu napätia maximom.

\beginfigure
	\filename={img/skratovy-prud.png}
	\label={skratovy-prud}
	\caption={Priebeh súmerného skratového prúdu}
\endfigure

Symetrický skratový prúd má tri zložky:
\beginitemize
	\item rázová zložka --- má sínusový priebeh s~kmitočtom sústavy,
	exponenciálne klesá s~časovou konštantou $T_k''$,
	\item prechodná zložka --- má sínusový priebeh s~kmitočtom sústavy,
	exponenciálne klesá s~časovou konštantou $T_k'$,
	\item ustálená zložka --- má sínusový priebeh so stálou amplitúdou a
s~kmitočtom sústavy.
\enditemize

Názvoslovie popisu skratového prúdu:
\beginitemize
	\item {\bf Súmerný skratový prúd} v~sústave $I_{ks}$ je efektívna
	hodnota súčtu ustálenej prechodnej a rázovej zložky skratového prúdu.
	\item {\bf Rázový skratový prúd} v~sústave $I_k''$ je efektívna
	hodnota súmerného skratového prúdu v~období trvania rázovej zložky
	skratového prúdu, teda prakticky v~období $t = 0 \div 3T_k''$.
	\item {\bf Počiatočný rázový skratový prúd} v~sústave $I_{k0}''$ je
	hodnota rázového skratového prúdu v~okamžiku vzniku skratu pre $t=0$.
	\item {\bf Prechodný skratový prúd} v~sústave $I_k'$ je súmerný
	skratový prúd v~období od vzniku rázovej zložky do zániku prechodnej
	zložky, teda prakticky v~období $t = 3 T_k'' \div 3 T_k'$.
	\item {\bf Počiatočný prechodný skratový prúd} v~sústave $I_{k0}'$ je
	efektívna hodnota súčtu trvalej a prechodnej zložky skratového prúdu
v~okamžiku vzniku skratu $t=0$. Pri vyvinutí rázového skratového prúdu
	sa k~nej dospeje predĺžením exponenciály prechodného skratového prúdu
	do okamžiku vzniku skratu.
	\item {\bf Ustálený skratový prúd} v~sústave $I_{ku}$ je súmerný
	skratový prúd po zaniknutí prechodných zložiek, teda pre $ t > 3
	T_k'$.
\enditemize

\subsection * Nesúmerný skratový prúd

Tento prípad nastáva, ak skrat vznikne v~okamihu nulového napätia.
Fázovo posunutý prúd by mal začínať zo svojej maximálnej hodnoty.
Skoková zmena prúdu však nie je možná, preto sa vytvorí ďalšia zložka
prúdu:
\beginitemize
	\item jednosmerná zložka --- exponenciálne zanikajúca s~časovou
	konštantou $T_a$.
\enditemize

K~tomu sa viažu ďalšie pojmy:
\beginitemize
	\item {\bf Počiatočná jednosmerná zložka} $I_{ka}$ je jednosmerná
	zložka skratového prúdu v~čase $t=0$. Jej veľkosť je taká, aby po
	superpozícii so striedavými zložkami skratového prúdu výsledný
	skratový prúd začínal z~nulovej hodnoty.
	\item {\bf Nárazový prúd} $I_{km}$ je vrcholová hodnota prvej
	polperiody skratového prúdu pri najväčšej možnej jednosmernej zložke.
\enditemize

\beginfigure
	\filename={img/skratovy-prud-nesumerny.png}
	\label={skratovy-prud-nesumerny}
	\caption={Priebeh nesúmerného skratového prúdu}
\endfigure

\subsection Metóda súmerných zložiek

V~trojfázových sústavách môžme každú hviezdicu nesymetrických napätí
nahradiť zložením symetrických fázorov sústavy súslednej, spätnej a
netočivej, tak ako je znázornené na \citefigure[rozklad-zlozky].

\beginfigure
	\filename={img/rozklad-zlozky.png}
	\label={rozklad-zlozky}
	\caption={Náhrada nesymetrických fázorov zložkovými fázormi}
\endfigure

Vzťah medzi napätiami fáz a napätiami v~zložkách
$$
\displaylines{
\faz U_a = \faz U_{a1} + \faz U_{a2} + \faz U_{a0},\cr
\faz U_b = \faz U_{b1} + \faz U_{b2} + \faz U_{b0},\cr
\faz U_c = \faz U_{c1} + \faz U_{c2} + \faz U_{c0}.\cr
}
$$

Ak zvolíme za referenčný bod fázu A, môžme tieto rovnice prepísať do
tvaru
$$
\eqalign{
\faz U_a &= \faz U_1 + \faz U_2 + \faz U_0, \cr
\faz U_b &= \faz a^2 U_1 + \faz a U_2 + \faz U_0,\cr
\faz U_c &= \faz a U_1 + \faz a^2 U_2 + U_0\cr
}
$$
a rovnice pre prúdy môžme zapísať ako
$$
\eqalign{
\faz I_a &= \faz I_1 + \faz I_2 + \faz I_0,\cr
\faz I_b &= \faz a^2 I_1 + \faz a I_2 + \faz I_0,\cr
\faz I_c &= \faz a I_1 + \faz a^2 I_2 + I_0\cr
}
$$
pričom pre $\faz a$ platí
$$
\faz a = \e^{\j 120\degree} = -{1\over 2}+ \j {\sqrt{3} \over 2} \qquad
\faz a^2 = \e^{-\j 120\degree} = -{1\over 2} - \j {\sqrt3 \over 2}
$$

Tieto rovnice môžme zapísať v~maticovom tvare ako
$$
\hmatrix{U_a \cr U_b \cr U_c \cr} =
\hmatrix{1 & 1 & 1\cr
\faz a^2 & \faz a & 1 \cr
\faz a & \faz a^2 & 1 \cr}
\cdot
\hmatrix{U_1 \cr U_2 \cr U_0 \cr}
\qquad
\faz U = \faz F \faz U_F
$$
respektíve
$$
\faz U_F = \faz F^{-1} \faz U \qquad
\faz F^{-1} = {1 \over 3} \hmatrix{
1 & \faz a & \faz a^2 \cr
1 & \faz a^2 & \faz a \cr
1 & 1 & 1\cr}.
\equation[uffu]
$$

Z~rovnice \citeequation[uffu] plynú vzťahy
$$
\eqalign{
	\faz U_1 &= {1\over 3} \left(\faz U_a + \faz a U_b + \faz a^2 U_c\right)\cr
	\faz U_2 &= {1\over 3} \left(\faz U_a + \faz a^2 U_b + \faz a U_c\right)\cr
	\faz U_0 &= {1\over 3} \left(\faz U_a + \faz U_b + \faz U_c\right)\cr
}
$$


\subsection Symetické články v~podmienkach nesymetrického prevádzkového
stavu

Pri rozbore nesymetrických prevádzkových stavov ES sa obmedzíme na
prípady jedinej miestnej nesymetrie, teda budeme predpokladať, že
nesymetria je spôsobená jediným nesymetrickým článkom a že ostatné
články ES sú symetrické. Sieť medzi zdrojmi a miestom nesymetrie je možné
vyjadriť kombináciou symetrických pasívnych článkov.

\subsection * Pozdĺžne články

\beginfigure
	\filename={img/clanok-pozdlzny.png}
	\label={clanok-pozdlzny}
	\caption={Pozdĺžny symetrický statický článok}
\endfigure

Pre úbytky na článku zapojenom podľa \citefigure[clanok-pozdlzny] platia
rovnice
$$
\hmatrix{\Delta U_a \cr \Delta U_b \cr \Delta U_c\cr} =
\hmatrix{Z & Z_m & Z_m \cr
Z_m & Z & Z_m \cr
Z_m & Z_m & Z \cr}
\hmatrix{I_a \cr I_b \cr I_c \cr}
\qquad
\Delta \faz U = \faz Z \faz I.
$$

Prejdeme k~súmerným zložkám a dostaneme
$$
\faz F \Delta \faz U_F = \faz Z \faz F \faz I_F \Longrightarrow
\Delta \faz U_F = \faz F^{-1} \faz Z \faz F \faz I_F = \faz Z_s \faz I_F
$$
kde
$$
\faz Z_s = \faz F^{-1} \faz Z \faz F =
\hmatrix{
\faz Z-\faz Z_m & 0 & 0 \cr
0 & \faz Z -\faz Z_m & 0 \cr
0 & 0 & \faz Z + 2\faz Z_m \cr
}.
$$

Úbytky napätí na pozdĺžne zapojenom statickom článku sú v~sústave
súmerných zložiek závislé iba na prúde príslušnej zložkovej sústavy a na
impedanciách daného článku,
$$
\eqalign{
\Delta \faz U_1 &= (\faz Z - \faz Z_m) \faz I_1  = \Delta \faz U_F = \faz  Z_1 \faz I_1 \cr
\Delta \faz U_2 &= (\faz Z - \faz Z_m) \faz I_1  = \Delta \faz U_F = \faz  Z_2 \faz I_2 \cr
\Delta \faz U_0 &= (\faz Z + 2\faz Z_m) \faz I_0  = \Delta \faz U_F = \faz  Z_0 \faz I_0 \cr
}
$$

Pre statické symetrické články pozdĺžne zapojené platí
$$
\faz Z_1 = \faz Z_2 = \faz Z - \faz Z_m \qquad \faz Z_0 = \faz Z + 2
\faz Z_m.
$$

\subsection * Priečne články

\beginfigure
	\filename={img/clanok-priecny.png}
	\label={clanok-priecny}
	\caption={Priečny symetrický statický článok}
\endfigure

Zapojeniu z~\citefigure[clanok-priecny] zodpovedajú rovnice
$$
\faz U = \faz Z \faz I + \faz Z_N \faz I
$$
kde
$$
\faz Z_N = \hmatrix{
Z_N & Z_N & Z_N \cr
Z_N & Z_N & Z_N \cr
Z_N & Z_N & Z_N \cr
}
$$

Prejdeme k~súmerným zložkám
$$
\displaylines{
\faz F \faz U_F  = \faz Z \faz F \faz I_F + \faz Z_N \faz F \faz I_F\cr
\faz U_F = \faz F^{-1} \faz Z \faz F \faz I_F + \faz F^{-1} \faz Z_N
\faz F \faz I_F = \faz Z_p \faz I_F\cr
}
$$
kde $\faz Z_p$ je matica zložkových sústav impedancií priečneho článku a
platí pre ňu
$$
\faz Z_p = \hmatrix{
\faz Z - \faz Z_m & 0 & 0 \cr
0 &\faz Z - \faz Z_m & 0 \cr
0 & 0 & \faz Z + 2 \faz Z_m + 3 \faz Z_N  \cr
}.
$$

Každé zložkové napätie na priečne zapojenom priečnom článku je závislé
iba na prúde príslušnej zložkovej sústavy a na impedanciách daného
článku,
$$
\eqalign{
\faz U_1 &= (\faz Z - \faz Z_m) \faz I_1 = \faz Z_1 \faz I_1 \cr
\faz U_2 &= (\faz Z - \faz Z_m) \faz I_2 = \faz Z_2 \faz I_2 \cr
\faz U_0 &= (\faz Z + 2\faz Z_m + 3 \faz Z_N) \faz I_0 = \faz Z_0 \faz I_0 \cr
}
$$

\subsection Napäťové rovnice v~sústave súmerných zložiek

Pri odvodzovaní napäťových rovníc pre ES s~miestnou nesymetriou
nahradíme pre jednoduchosť všetky zdroje ES jedným ekvivalentným zdrojom
a sieť pasívnych symetrických článkov medzi svorkami a miestnu
nesymetriu jedným ekvivalentným pasívnym článkom, ako je znázornená na
\citefigure[nahradna-schema-ES].

\beginfigure
	\filename={img/nahradna-schema-ES.png}
	\label={nahradna-schema-ES}
	\caption={Náhradná schéma ES s~miestnou priečnou nesymetriou}
\endfigure

Nesymetrický článok (miestna nesymetria) je nahradený nesymetrickými
fázovými napätiami $\faz U_a$, $\faz U_b$, $\faz U_c$.

Napäťové rovnice vo fázových súradniciach
$$
\faz E = \faz Z \faz I  \faz U
$$
transformujeme do súmerných zložiek a dostávame
$$
\eqalignno{
\faz F \faz E &= \faz Z \faz F \faz I_F + \faz F \faz U_F\cr
\faz E_F &= \faz F^{-1} \faz Z \faz F \faz I_F + \faz U_F =
\faz Z_s \faz I_F + \faz U_F& \equation[v sumernych]\cr
}
$$
kde
$$
\faz Z_s = \hmatrix{
\faz Z_{c1} & 0 & 0 \cr
0 & \faz Z_{c2} & 0 \cr
0 & 0 & \faz Z_{c0} \cr
}.
$$

Impedancie $\faz Z_{c1}$, $\faz Z_{c2}$ a $\faz Z_{c0}$ sú celkové
impedancie súslednej, spätnej a netočivej zložky medzi miestom
nesymetrie a nulovým bodom príslušnej zložkovej sústavy.

Pre zložkové napätia platí
$$
\faz E_F = \faz F^{-1} \faz E = {1 \over 3} \hmatrix{
1 & \faz a & \faz a^2 \cr
1 & \faz a^2 & \faz a \cr
1 & 1 & 1 \cr
}
\hmatrix{
\faz E \cr
\faz a^2 \faz E \cr
\faz a \faz E \cr
}
= 
\hmatrix{\faz E \cr 0 \cr 0 \cr}.
$$

Po dosadení za $\faz Z_s$ a $\faz E_F$ do rovnice \citeequation[v
sumernych] môžme zapísať v~maticovom tvare
$$
\eqalignno{
\faz E &= \faz Z_{c1} \faz I_1 + \faz U_1, \cr
0 &= \faz Z_{c2} \faz I_2  + \faz U_2, \cr
0 &= \faz Z_{c0} \faz I_0 + \faz U_0. & \equation[zlozkove-napatia] \cr
}
$$

Z~toho plynú dôležité závery:
\beginitemize
	\item Elektromotorické napätie zdroja pôsobí iba v~súslednej zložkovej
	sústave, v~ostatných je nulové.
	\item V~každej zložkovej sústave pôsobia iba napätia a prúdy danej
	zložkovej sústavy, preto sú jednotlivé zložkové sústavy až do miesta
	vzniku nesymetrie vzájomne nezávislé. Môžme použiť náhradnú schému
	zobrazenú na \citefigure[nahr-sch-mpn].
	\item Napäťové rovnice predstavujú tri základné rovnice pre riešenie
	prevádzkových pomerov pri miestnej nesymetrii. Rovnice
	\citeequation[zlozkove-napatia] doplníme ďalšími troma rovnicami,
	ktoré sú určené charakterom vzniknutej miestnej nesymetrie. 
\enditemize

\beginfigure
	\filename={img/nahr-sch-mpn.png}
	\label={nahr-sch-mpn}
	\caption={Náhradná schéma zložkových sústav ES s~miestnou priečnou
	nesymetriou}
\endfigure

\subsection Poruchy miestneho charakteru

Medzi najčastejšie poruchy patria nesymetrické skraty (miestna priečna
nesymetria) a prerušenia fáz niektorého článku (miestna pozdĺžna
nesymetria).

Ľubovoľný druh priečnej alebo pozdĺžnej nesymetrie v niektorom mieste ES
môžme modelovať zapojením obecného nesymetrického priečneho alebo
pozdĺžneho článku v tomto mieste. Našou úlohou bude analýza najčastejšie
sa vyskytujúcich prípadov miestnej nesymetrie.

\subsection * Jednofázový skrat

\beginfigure
	\filename={img/skrat-1f.png}
	\label={skrat-1f}
	\caption={Jednofázový skrat}
\endfigure

Na základe \citefigure[skrat-1f] môžme zapísať charakteristické rovnice
ako
$$
\faz U_a = 0 \quad \faz I_b = 0 \quad \faz I_c = 0.
$$

Prechodom k zložkovým veličinám získame pre zložkové prúdy rovnice
$$
\faz I_F = \faz F^{-1} \faz I =
{1 \over 3} \hmatrix{
1 & \faz a & \faz a^2 \cr
1 & \faz a^2 & \faz a \cr
1 & 1 & 1 \cr
}
\hmatrix{\faz I_a \cr 0 \cr 0 \cr}=
{1 \over 3} \hmatrix{\faz I_a \cr\faz I_a \cr\faz I_a \cr}
$$
z ktorých plynie vzťah
$$
\faz I_1 = \faz I_2 = \faz I_0 = {1 \over 3} \faz I_a
$$
na základe ktorého môžme vzájomne prepojiť zložkové schémy.

\beginfigure
	\filename={img/skrat-1f-zlozky.png}
	\label={skrat-1f-zlozky}
	\caption={Prepojenie náhradných schém zložkových sústav pri
	jednofázovom skrate}
\endfigure

Pre zložkové prúdy platia vzťahy
$$
\faz I_1 = {\faz E \over \faz Z_{c1} + \faz Z_{c2} + \faz Z_{c0}} = 
\faz I_2 = \faz I_0.
$$

Súčet zložkových napätí v mieste skratu je nulový, teda
$$
\faz U_1 + \faz U_2 + \faz U_0 = 0.
$$

Pre zložkové napätia v mieste skratu môžme napísať rovnice
$$
\displaylines{
\faz U_0 = - \faz Z_{c0}\faz I_0 = - \faz Z_{c0} \faz I_1, \cr
\faz U_2 = - \faz Z_{c2}\faz I_2 = - \faz Z_{c2} \faz I_1, \cr
\faz U_1 = (\faz Z_{c2} + \faz Z_{c1}) \faz I_1. \cr
}
$$

Pre fázové prúdy platia vzťahy
$$
\faz I_a = 3 \faz I_1 \quad \faz I_b = 0 \quad \faz I_c = 0.
$$

Vzťahy pre fázové napätia v mieste skratu získame lineárnou
transformáciou zložkových napätí
$$
\faz U = \faz F \faz U_F =
\hmatrix{
1 & 1 & 1 \cr
\faz a^2 & \faz a & 1 \cr
\faz a & \faz a^2 & 1 \cr
}
\hmatrix{\faz U_1 \cr \faz U_2 \cr \faz U_0 \cr} =
\hmatrix{
0 \cr
(\faz a^2 - \faz a) \faz Z_{c2} + (\faz a^2 - 1) \faz Z_{c0} \cr
(\faz a - \faz a^2) \faz Z_{c2} + (\faz a - 1) \faz Z_{c0} \cr
}
\faz I_1.
$$


\beginfigure
	\filename={img/skrat-1f-prud-fd.png}
	\label={skrat-1f-prud-fd}
	\caption={Fázorový diagram prúdov pri jednofázovom skrate}
\endfigure

\beginfigure
	\filename={img/skrat-1f-napatie-fd.png}
	\label={skrat-1f-napatie-fd}
	\caption={Fázorový diagram napätí v mieste jednofázového skratu}
\endfigure


\subsection * Dvojfázový skrat

\beginfigure
	\filename={img/skrat-2f.png}
	\label={skrat-2f}
	\caption={Dvojfázový skrat}
\endfigure

Na základe \citefigure[skrat-2f] môžme zapísať charakteristické rovnice
ako
$$
\faz I_a = 0 \quad \faz I_b = \faz I_c \quad \faz U_b = \faz U_c.
$$

Prechodom k zložkovým veličinám získame pre zložkové prúdy rovnice
$$
\faz I_F = \faz F^{-1} \faz I =
{1 \over 3} \hmatrix{
1 & \faz a & \faz a^2 \cr
1 & \faz a^2 & \faz a \cr
1 & 1 & 1 \cr
}
\hmatrix{0 \cr \faz I_b \cr -\faz I_b \cr}=
{1 \over 3} \hmatrix{\j \sqrt3 \faz I_b \cr -\j \sqrt3 \faz I_b \cr 0 \cr}
$$
z ktorých plynú vzťahy
$$
\faz I_1 = - \faz I_2 \quad \faz I_0 = 0
$$
na základe ktorých môžme vzájomne prepojiť zložkové schémy.

\beginfigure
	\filename={img/skrat-2f-zlozky.png}
	\label={skrat-2f-zlozky}
	\caption={Prepojenie náhradných schém zložkových sústav pri
	dvojfázovom skrate}
\endfigure

Pre zložkové prúdy platia podľa \citefigure[skrat-2f-zlozky] vzťahy
$$
\faz I_1 = {\faz E \over \faz Z_{c1} + \faz Z_{c2}}\quad
\faz I_2 = - \faz I_1 \quad
\faz I_0 = 0.
$$

Zložkové napätia sú
$$
\displaylines{
\faz U_2 = - \faz Z_{c2}\faz I_2 = \faz Z_{c2} \faz I_1, \cr
\faz U_1 = \faz U_2 = \faz Z_{c2} \faz I_1, \cr
\faz U_0 = 0. \cr
}
$$

Pre fázové prúdy platia vzťahy
$$
\faz I_a = 0 \quad \faz I_b = {3 \faz I_1 \over \j \sqrt3} =
-\j \sqrt3 I_1 \quad \faz I_c = -\faz I_b = \j \sqrt3 \faz I_1.
$$

Vzťahy pre fázové napätia v mieste skratu získame lineárnou
transformáciou zložkových napätí
$$
\faz U = \faz F \faz U_F =
\hmatrix{
1 & 1 & 1 \cr
\faz a^2 & \faz a & 1 \cr
\faz a & \faz a^2 & 1 \cr
}
\hmatrix{\faz U_1 \cr \faz U_2 \cr \faz U_0 \cr} =
\hmatrix{
2\faz u_1 \cr
-\faz U_1 \cr
-\faz U_1 \cr
}
$$
z ktorých po dosadení za zložkové napätia dostaneme vzťahy
$$
\displaylines{
\faz U_a = 2 \faz Z_{c2} \faz I_1 \cr
\faz U_b = -\faz Z_{c2} \faz I_1 = - {1 \over 2} \faz U_a \cr
\faz U_c = - \faz Z_{cs} \faz I_1 = - {1 \over 2} \faz U_a = \faz U_b
\cr
}
$$

\beginfigure
	\filename={img/skrat-2f-prud-fd.png}
	\label={skrat-2f-prud-fd}
	\caption={Fázorový diagram prúdov pri dvojfázovom skrate}
\endfigure

\beginfigure
	\filename={img/skrat-2f-napatie-fd.png}
	\label={skrat-2f-napatie-fd}
	\caption={Fázorový diagram napätí v mieste dvojfázového skratu}
\endfigure

\subsection * Dvojfázový zemný skrat

\beginfigure
	\filename={img/skrat-2fzemny.png}
	\label={skrat-2fzemny}
	\caption={Dvojfázový zemný skrat}
\endfigure

Na základe \citefigure[skrat-2fzemny] môžme zapísať charakteristické rovnice
ako
$$
\faz I_a = 0 \quad \faz U_b = 0 \quad \faz U_c = 0.
$$
Lineárnou transformáciou napätí pri rešpektovaní týchto
charakteristických rovníc dostávame vzťahy
$$
\faz U_F = \faz F^{-1} \faz U = {1 \over 3}
\hmatrix{
1 & \faz a & \faz a^2 \cr
1 & \faz a^2 & \faz a \cr
1 & 1 & 1 \cr
}
\hmatrix{\faz U_a \cr 0 \cr 0 \cr} =
{1 \over 3} \hmatrix{\faz U_a \cr \faz U_a \cr \faz U_a \cr}
$$
z ktorých plynú vzťahy medzi zložkovými napätiami v mieste skratu
$$
\faz U_1 = \faz U_2 = \faz U_0.
$$
Na základe týchto rovníc môžme zostrojiť schému zložkových sústav pri
dvojfázovom zemnom skrate.

\beginfigure
	\filename={img/skrat-2fzemny-zlozky.png}
	\label={skrat-2fzemny-zlozky}
	\caption={Prepojenie náhradných schém zložkových sústav pri
	dvojfázovom zemnom skrate}
\endfigure

Na základe schémy (\citefigure[skrat-2fzemny-zlozky]) môžme písať vzťahy
pre prúdy zložkových sústav
$$
\displaylines{
\faz I_1 = {\faz E \over \faz Z_{c1} + {\faz Z_{c2} \faz Z_{c0} \over
\faz Z_{c2} + \faz Z_{c0}}}\cr
\faz I_2 = - {Z_{c0} \over Z_{c2} + Z_{c0}} \faz I_1 \cr
\faz I_0 = - {Z_{c2} \over Z_{c2} + Z_{c0}} \faz I_1 \cr
}
$$

Pre napätia v zložkových sústavách platia vzťahy
$$
\faz U_2 = - \faz Z_{c2} \faz I_2 =
{\faz Z_{c2} \faz Z_{c0} \over \faz Z_{c2} + \faz Z_{c0}} \faz I_1 \qquad
\faz U_1 = \faz U_2 = \faz U_0
$$

Vzťahy pre fázové prúdy získame lineárnou transformáciou zložkových
prúdov
$$
\faz I = \faz F \faz I_F =
\hmatrix{
1 & 1 & 1 \cr
\faz a^2 & \faz a & 1 \cr
\faz a & \faz a^2 & 1 \cr
}
\hmatrix{
1 \cr
-{\faz Z_{c0} \over \faz Z_{c2} + \faz Z_{c0}} \cr
-{\faz Z_{c2} \over \faz Z_{c2} + \faz Z_{c0}} \cr
}
\faz I_1 =
\hmatrix{
0 \cr
-{(\faz a^2 - 1) \faz Z_{c2} + (\faz a^2 - \faz a) \faz Z_{c0} \over
\faz Z_{c2} + \faz Z_{c0}}\cr
-{(\faz a - 1) \faz Z_{c2} + (\faz a - \faz a^2) \faz Z_{c0} \over
\faz Z_{c2} + \faz Z_{c0}}\cr
}
\faz I_1
$$

Fázové napätia v mieste skratu je možné určiť lineárnou transformáciou
zložkových napätí,
$$
\faz U = \faz F \faz U_F =
\hmatrix{
1 & 1 & 1 \cr
\faz a^2 & \faz a & 1 \cr
\faz a & \faz a^2 & 1 \cr
}
\hmatrix{\faz U_1 \cr \faz U_2 \cr \faz U_3 \cr} =
\hmatrix{3 \faz U_1 \cr 0 \cr 0 \cr}.
$$

Ak do tejto rovnice dosadíme za zložkové napätia, získame vzťahy pre
fázové napätia
$$
\faz U_a =
3 {\faz Z_{c2} \faz Z_{c0} \over \faz Z_{c2} + \faz Z_{c0}} \faz I_1
\qquad
\faz U_b = \faz U_c = 0
$$

\beginfigure
	\filename={img/skrat-2fzemny-fd.png}
	\label={skrat-2fzemny-fd}
	\caption={Fázorový diagram prúdov a napätí v mieste skratu pri
	dvojfázovom zemnom skrate}
\endfigure

\subsection * Trojfázový skrat

\beginfigure
	\filename={img/skrat-3f.png}
	\label={skrat-3f}
	\caption={Trojfázový skrat}
\endfigure

Pre trojfázový skrat platia podmienky
$$
	\faz U_a = \faz U_b = \faz U_c = 0.
$$

Transformáciou do zložkových sústav
$$
\faz U = \faz F \faz U_F =
\hmatrix{
1 & 1 & 1 \cr
\faz a^2 & \faz a & 1 \cr
\faz a & \faz a^2 & 1 \cr
}
\hmatrix{\faz U_1 \cr \faz U_2 \cr \faz U_0 \cr}
=
\hmatrix{0\cr 0\cr 0\cr}
$$
z čoho plynú vzťahy
$$
\displaylines{
\faz U_0 = \faz U_1 = \faz U_2 = 0\cr
\faz I_1 = {\faz E \over \faz Z_{c1}} \qquad \faz I_2 = 0 \qquad \faz
I_0 = 0\cr
}
$$

Pri spätnej transformácii dostaneme pre fázové prúdy vzťahy
$$
\faz I_a = {\faz E \over \faz Z_{c1}} \faz I_1 \qquad
\faz I_b = \faz a^2 \faz I_a \qquad
\faz I_c = \faz a \faz I_a.
$$

Pri trojfázovom skrate sa uplatňuje iba súsledná zložková sústava.

\subsection * Ekvivalencia nesymetrických skratov s trojfázovým skratom

\tablelabel[label=skraty-prehlad][caption=Prehľad jednotlivých druhov
skratov] 
\centertable{
\begintable
\begintableformat &\center \endtableformat
\-
\br{\:|} Druh skratu | trojfázový | dvojfázový | jednofázový | dvojfázový zemný \er{|}
\-
\br{\:|} $\faz Z_{\Delta}$ | 0 | $\faz Z_{c2}$ | $\faz Z_{c2} + \faz
Z_{c0}$ | ${\faz Z_{c2} \faz Z_{c0} \over Z_{c2} + Z_{c0}}$ \er{|}
\-
\br{\:|} Zložky prúdu | $\faz I_1$ | $\faz I_2 = - \faz I_1$ | $\faz I_1
= \faz I_2 = \faz I_0$ | $\faz I_1 = -(\faz I_2 + \faz I_0)$ \er{|}
\-
\br{\:|} $m$ | 1 | $\sqrt{3}$ | 3 | $\sqrt3 \sqrt{1 - {X_{c2} X_{c0}
\over (X_{c2} + X_{c0})^2}}$ $\vphantom{{\strut\over\strut}}$ \er{|}
\-
\br{\:|} Náhradná schéma |
\obr{img/skrat-schemicka-3f.png} |
\obr{img/skrat-schemicka-2f.png} | 
\obr{img/skrat-schemicka-1f.png} |
\obr{img/skrat-schemicka-2fz.png}
 \er{|}
\-
\endtable
}


Pri výpočte skratov môžme používať pre súsledný prúd jeden vzťah,
$$
\faz I_1 =  {\faz E \over \faz Z_{c1}  + \faz Z_{\Delta}},
$$
do ktorého dosadíme za $\faz Z_{\Delta}$ hodnotu podľa
\citetable[skraty-prehlad].

Impedancia $\faz Z_{\Delta}$ je prídavná impedancia. Je závislá na druhu
skratu a je určovaná iba impedanciami $\faz Z_{c2}$ a $\faz Z_{c0}$. Nie
je závislá na parametroch súslednej zložkovej sústavy. Pre každý druh
skratu v danom mieste ES zostáva konštantná po celú dobu trvania skratu.

Veľkosť fázového skratového prúdu môžme vyjadriť pomocou koeficientu $m$
z \citetable[skraty-prehlad] ako
$$
I_k = m I_1.
$$

\subsection * Vplyv oblúku pri skrate

V prvom priblížení je možné vplyv oblúku nahradiť rezistanciou, ako je
znázornené na \citefigure[obluk-1f]. Pri
jednofázovom skrate sa zväčší o $R$ vo všetkých zložkových sústavách
celková skratová impedancia. Ináč sa tento prípad vôbec nelíši od
jednofázového.

\beginfigure
	\filename={img/obluk-1f.png}
	\caption={Schéma náhrady oblúku rezistanciou pri jednosmernom skrate}
	\label={obluk-1f}
\endfigure


Pre súsledný prúd máme vzťah
$$
\faz I_1 = {\faz E \over \faz Z_{c1} + \faz Z_{c2} + \faz Z_{c0} +3 R}.
$$

\beginfigure
	\filename={img/obluk-1f-zlozky.png}
	\caption={Náhradná schéma jednofázového skratu pri náhrade skratového
	oblúku rezistanciou}
	\label={obluk-1fzlozky}
\endfigure

Ak chceme rešpektovať vplyv oblúku pri výpočte dvojfázového skratu, do
skratovej odbočky zaradíme symetrický článok s rezistanciami $R/2$, ako
je znázornené na \citefigure[obluk-2f].

\beginfigure
	\filename={img/obluk-2f.png}
	\caption={Schéma náhrady oblúku rezistanciou pri dvojfázovom skrate}
	\label={obluk-2f}
\endfigure

Pre súsledný prúd platí vzťah
$$
\faz I_1 = {\faz E \over \faz Z_{c1} + \faz Z_{c2} + R}.
$$

\beginfigure
	\filename={img/obluk-2fzlozky.png}
	\caption={Náhradná schéma dvojfázového skratu pri náhrade skratového
	oblúku rezistanciou}
	\label={obluk-2fzlozky}
\endfigure

% TODO: dvojpolovy zemny?

\subsection * Prerušenie fázy

Prerušenie jednej fázy môžme chápať ako zapojenie nesymetrického článku
v mieste poruchy, charakterizovaného úbytkami napätia v jednotlivých
fázach.

\beginfigure
	\filename={img/prerusenie-jedna.png}
	\label={prerusenie-jedna}
	\caption={Schématické znázornenie prerušenia jednej fázy}
\endfigure

Charakteristické rovnice sú
$$
\faz I_a = 0 \qquad \Delta \faz U_b = 0 \qquad \Delta \faz U_c = 0.
$$
Ak porovnáme tieto rovnice s charakteristickými rovnicami pre dvojfázový
skrat, tak vidíme, že sú formálne zhodné. Pre zložkové napätia teda
môžme písať vzťahy
$$
\Delta \faz U_1 = \Delta \faz U_2 = \Delta \faz U_0 = {1 \over 3} \Delta
\faz U_a
$$

\beginfigure
	\filename={img/prerusenie-jedna-zlozky.png}
	\label={prerusenie-jedna-zlozky}
	\caption={Náhradná schéma zložkových sústav pri prerušení fáze}
\endfigure

Pre súsledný prúd dostávame vzťah
$$
\faz I_1 = {\faz E \over \faz Z_{c1} + 
{\faz Z_{c2} \faz Z_{c0} \over \faz Z_{c2} + \faz Z_{c0}}} =
{\faz E \over \faz Z_{c1} + \Delta \faz Z}
$$

Pri prerušení dvoch fáz zapojíme v mieste poruchy nesymetrický článok,
charakterizovaný úbytkami napätia podľa \citefigure[prerusenie-dve].
Charakteristické rovnice sú
$$
\faz I_b = 0 \qquad \faz I_c = 0 \qquad \Delta \faz U_a = 0.
$$

\beginfigure
	\filename={img/prerusenie-dve.png}
	\label={prerusenie-dve}
	\caption={Schematické znázornenie prerušenia dvoch fáz}
\endfigure

Charakteristické rovnice sú zhodné s charakteristickými rovnicami
platnými pre jednofázový skrat a preto môžme písať
$$
\faz I_1 = \faz I_2 = \faz  I_0 = {1 \over 3} \faz I_a
$$

\beginfigure
	\filename={img/prerusenie-dve-zlozky.png}
	\label={prerusenie-dve-zlozky}
	\caption={Prepojenie náhradných schém zložkových sústav pri prerušení
	dvoch fáz}
\endfigure

\subsection * Analógia medzi prerušeniami a skratmi

Prerušenia fáz a skraty sú podobné, ale predsa rozdielne. Prídavná
impedancia $\faz Z_{\Delta}$ sa zapája na iné miesto. V prípade skratu
sú $Z_{c1}, Z_{c2}, Z_{c0}$ impedancie medzi miestom skratu a nulou
príslušnej zložkovej sústavy a v prípade prerušenia fáz tieto impedancie
zodpovedajú výsledným impedanciám zložkových sústav vypočítaným medzi
bodmi po oboch stranách prerušenia príslušnej zložkovej sústavy.

\beginfigure
	\filename={img/skrat-prerusenie-porovnanie.png}
	\label={skrat-prerusenie-porovnanie.png}
	\caption={Porovnanie prerušenia fázy a skratu}
\endfigure

\subsection * Viacnásobné nesymetrie v elektrizačnej sústave

FIXME

\subsection * Skratová impedančná matica

FIXME

\section * Použité materiály

\beginitemize
	\item Němeček: Přenos a rozvod elektrické energie. ČVUT, Praha 1983.
	\item Fejt, Čermák: Elektroenergetika. ČVUT, Praha 1985.
	\item Trojánek, Hájek, Kvasnica: Přechodné jevy v elektrizačních
	soustavách. SNTL, Praha 1987.
	\item Materiály k predmetu X15PJE.
	http://www.powerwiki.cz/wiki/X15PJE.
\enditemize


