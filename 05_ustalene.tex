\section Ustálené chody v~uzlových sieťach

\subsection Metóda uzlových napätí

V~uvažovanom uzle $k$ uzlovej siete, ktorej časť je znázornená na
\citefigure[uzlove-napatia], platí podľa Kirchhoffovho zákona o~prúdoch
v~uzle pri prijatých orientáciách fázorov prúdov
$$
\faz I_k + \sum_{m \in {\rm M}_k} \left(\faz I_{km} + \faz
I_{km0}\right) = 0.
\equation[prudy-v-uzle]
$$

\beginfigure
	\filename={img/uzlove-napatia.png}
	\label={uzlove-napatia}
	\caption={Základná schéma pri riešení pomocou metódy uzlových napätí}
\endfigure

Označenie ${\rm M}_k$ predstavuje množinu indexov uzlov bezprostredne
susediacich s~uzlom $k$ neobsahujúcu tento uzol.

Pre prúdy platia vzťahy
$$
\displaylines{
\faz I_{km} =
\faz Y_{km} (\faz U_{k\rm f} - \faz U_{m \rm f}) =
- \faz I_{mk},\cr
\faz I_{km0} = \faz Y_{km0} \faz U_{k\rm f}.\cr
}
$$

Ak do rovnice \citeequation[prudy-v-uzle] dosadíme za prúdy, dostaneme
vzťahy
$$
\displaylines{
\faz I_k = - \sum_{m \in {\rm M}_k}
\left[
\faz Y_{km}(\faz U_{k\rm f}-\faz U_{m\rm f})+\faz Y_{km0} \faz U_{k\rm f}
\right]\cr
\faz I_{k} =
\faz U_{k\rm f} \sum_{m \in {\rm M}_k}
\left[-(\faz Y_{km0} +\faz Y_{km})\right]
+\sum_{m \in {\rm M}_k} \faz Y_{km} \faz U_{m\rm f}\cr
}
$$

Definične zavedieme vlastnú uzlovú impedanciu
\zvyrazni
$$
\faz Y_{kk} = - \sum_{m \in {\rm M}_k} (\faz Y_{km0} + \faz Y_{km}.
$$
Jej zavedenie umožňuje zápis rovníc v~tvare
$$
\faz I_k =
\faz Y_{kk} \faz U_{k\rm f} +
\sum_{m \in {\rm M}_k} \faz Y_{km} \faz U_{m\rm f}
\longrightarrow
\faz I_{k} = \sum_{m \in {\rm M}_{kk}} \faz Y_{km} U_{m\rm f}
\equation[katy prud]
$$
kde ${\rm M}_{kk}$ je množina uzlov bezprostredne susediaca s~uzlom $k$
vrátane tohto uzla.

Na základe rovnice \citeequation[katy prud] dostávame v~maticovom tvare
rovnice
\zvyrazni
$$
[\faz I] = [\faz Y] [\faz U_{\rm f}] \qquad \sqrt{3} [\faz I] = [\faz Y][\faz U].
$$

\subsection Metóda uzlových napätí so zadaným napätím v~jednom uzle

Uzlu so zadaným fázorom napätia pridelíme index 1. V~zostávajúcich $n-1$
uzloch (indexy 2 až $n$) sú zadané prúdy svojimi fázormi. Usporiadanie
zodpovedá napríklad napájaniu z~jedného napájacieho uzlu.


V~maticovom tvare môžme pre tento prípad zostaviť rovnicu
$$
\hmatrix{
\faz I_1 \cr \faz I_2 \cr \faz I_3 \cr
\vdots \cr
\faz I_{n-1} \cr \faz I_{n}\cr
}
=
\hmatrix{
\faz Y_{11} & \faz Y_{12} & \faz Y_{13} & \cdots & \faz Y_{1(n-1)} & \faz Y_{1n} \cr
\faz Y_{21} & \faz Y_{22} & \faz Y_{23} & \cdots & \faz Y_{2(n-1)} & \faz Y_{2n} \cr
\faz Y_{31} & \faz Y_{32} & \faz Y_{33} & \cdots & \faz Y_{3(n-1)} & \faz Y_{3n} \cr
\vdots & \vdots & \vdots & \ddots & \vdots & \vdots \cr
\faz Y_{(n-1)1} & \faz Y_{(n-1)2} & \faz Y_{(n-1)3} & \cdots & \faz Y_{(n-1)(n-1)} & \faz Y_{(n-1)n} \cr
\faz Y_{n1} & \faz Y_{n2} & \faz Y_{n3} & \cdots & \faz Y_{n(n-1)} & \faz Y_{nn} \cr
}
\hmatrix{
\faz U_{1\rm f} \cr
\faz U_{2\rm f} \cr
\faz U_{3\rm f} \cr
\vdots \cr
\faz U_{(n-1)\rm f} \cr
\faz U_{n\rm f}\cr
}
$$
ktorú môžme pre prehľadnosť zapísať pomocou blokov v~tvare
$$
\hmatrix{
[\faz I_1] \cr
[\faz I_{\rm p}] \cr
}
=
\hmatrix{
[\faz Y_{11}] & [\faz Y_{1p}] \cr
[\faz Y_{p1}] & [\faz Y_{pp}] \cr
}
\hmatrix{
[\faz U_{1\rm f}] \cr
[\faz U_{p\rm f}] \cr
}
\qquad p = (2, 3, \dots, n)
$$

Uvažujeme, že bloky $[\faz Y_{11}]$ a $[\faz Y_{pp}]$ sú regulárne a
teda majú inverznú maticu. Z~predošlého vzťahu plynú rovnosti
$$
[\faz I_1] =
[\faz Y_{11}] [\faz U_{1\rm f}] + [\faz Y_{1p}] [\faz U_{p\rm f}]
\qquad
[\faz I_p] =
[\faz Y_{p1}][\faz U_{1\rm f}] + [\faz Y_{pp}] [\faz U_{p\rm f}].
$$
Po násobení inverznou maticou $[\faz Y_{pp}]^{-1}$ a úpravách dostávame
vzťahy
$$
\displaylines{
[\faz U_{p{\rm f}}] = [\faz Y_{pp}]^{-1}[\faz I_p] - [\faz
Y_{pp}]^{-1}[\faz Y_{p1}][\faz U_{1\rm f}] \quad\hbox{respektíve}\cr
[\faz Y_{pp}][\faz U_{p\rm f}] =
[\faz I_p] - [\faz Y_{p1}][\faz U_{1\rm f}]\cr
}
$$
ktoré predstavujú v~maticovom zápise sústavu $n-1$ lineárnych rovníc
s~obecne komplexnými čislami. Tie budeme riešiť.

\subsection Gauss-Seidlova metóda

Ide o~iteratívnu metódu určenú na riešenie nelineárnych rovníc. Nie vždy
dobre konverguje.

\subsection * Základná úvaha

Rovnicu $f(x) = 0$ prepíšeme do tvaru $x = g(x)$. Ak je $x^{(k)}$ odhad
v~$k$-tom kroku, potom ďalšia iterácia je $x^{(k+1)} = g(x^{(k)})$.
Takto pokračujeme pokiaľ rozdiel nasledujúcich iterácií nie je menší než
stanovená presnosť $\epsilon$, 
$$
\left|x^{(k+1)} - x^{(k)}\right| \le \epsilon.
$$

Niekedy je možné konvergenciu zlepšiť takzvaným akceleračným faktorom
$\alpha$,
$$
x^{(k+1)} = x^{(k)} + \alpha \left[g(x^{(k)}) -x^{(k)}\right].
$$

\bigskip

Majme zadanú sústavu $n$ rovníc s~$n$ neznámymi
$$
\displaylines{
f_1(x_1, x_2, \dots, x_n) = c_1,\cr
f_2(x_1, x_2, \dots, x_n) = c_2,\cr
\dots\cr
f_n(x_1, x_2, \dots, x_n) = c_n\cr
}
$$
a z~každej rovnice vyjadrime jednu neznámu
$$
\displaylines{
x_1 = c_1 + g_1(x_1, x_2, \dots, x_n), \cr
x_2 = c_2 + g_2(x_1, x_2, \dots, x_n), \cr
\dots\cr
x_n = c_n + g_n(x_1, x_2, \dots, x_n). \cr
}
$$

Pri výpočte $k$-tej iterácie sa využívajú aj $k$-té aproximácie
z~predchádzajúcich rovníc,
$$
x_m^{(k)} = c_1 + g_1\left(x_1^{(k)}, x_2^{(k)}, \dots, x_{m-1}^{(k)}, \dots,
x_n^{(k-1)}\right).
$$

Konvergenciu testujeme pre každú premennú zvlášť.

\subsection Newton-Raphsonova metóda

Ide o~najrozšírenejšiu metódu vhodnú pre riešenie nelineárnych rovníc.
Pri jej použití využívame Taylorov polynom. Riešenie nelineárnych rovníc
prevádza na riešenie rovníc lineárnych, odhad je postupne spresňovaný.

\subsection * Základná úvaha

Majme rovnicu $f(x) = c$. Ak je $x^{(0)}$ počiatočný odhad a $\Delta
x^{(0)}$ odchýlka od správneho riešenia, potom
$$
f\left(x^{(0)} + \Delta x^{(0)}\right) = c.
$$

Rozvojom do Taylorovej rady dostaneme
$$
f\left(x^{(0)}\right) + \left({\d f \over \d x}\right)^{(0)} \Delta x^{(0)} + {1
\over 2!} \left({\d^2 f \over \d x^2}\right)^{(0)} \left(\Delta
x^{(0)}\right)^2 + \dots = c.
$$

Vyššie rády však môžme zanedbať (linearizácia) a dostávame
$$
\Delta c^{(0)} \approx
\left({\d f \over \d x}\right)^{(0)} \Delta x^{(0)}
$$
kde
$$
\Delta c^{(0)} = c-f\left(x^{(0)}\right)
$$
je takzvaný defekt.

Pričítaním $\Delta x^{(0)}$ k~počiatočnému odhadu získame druhú
aproximáciu
$$
x^{(1)} = x^{(0)} + {\Delta c^{(0)} \over \left({\d f \over \d
x}\right)^{(0)}}.
$$
(pre nultú deriváciu nie je možné).

Rovnakými vzťahmi v~ďalších krokoch získame algoritmus metódy:
$$
\displaylines{
\Delta c^{(k)} = c - f\left(x^{(k)}\right),\cr
\Delta x^{(k)} =
{\Delta c^{(k)} \over \left({\d f \over \d x}\right)^{(k)}},\cr
x^{(k+1)} = x^{(k)} + \Delta x^{(k)},\cr
\Delta c^{(k+1)} = c - f\left(x^{(k+1)}\right).\cr
}
$$

\subsection Systém $\bi n$ rovníc s~$\bi n$ neznámymi

Majme rovnice
$$
\displaylines{
f_1(x_1, x_2, \dots, x_n) = c_1, \cr
f_2(x_1, x_2, \dots, x_n) = c_2, \cr
\dots\cr
f_n(x_1, x_2, \dots, x_n) = c_n.\cr
}
$$

Z~nich rozvojom do Taylorovej rady dostaneme
$$
\displaylines{
(f_1)^{(0)} +
\left({\partial f_1 \over \partial x_1}\right)^{(0)} \Delta x_1^{(0)} + 
\left({\partial f_1 \over \partial x_2}\right)^{(0)} \Delta x_2^{(0)} +
\dots
+ \left({\partial f_1 \over \partial x_n}\right)^{(0)} \Delta x_n^{(0)}
= c_1,\cr
%
(f_2)^{(0)} +
\left({\partial f_2 \over \partial x_1}\right)^{(0)} \Delta x_1^{(0)} + 
\left({\partial f_2 \over \partial x_2}\right)^{(0)} \Delta x_2^{(0)} +
\dots
+ \left({\partial f_2 \over \partial x_n}\right)^{(0)} \Delta x_n^{(0)}
= c_2,\cr
%
\vphantom{|}\dots\cr
(f_n)^{(0)} +
\left({\partial f_n \over \partial x_1}\right)^{(0)} \Delta x_1^{(0)} + 
\left({\partial f_n \over \partial x_2}\right)^{(0)} \Delta x_2^{(0)} +
\dots
+ \left({\partial f_n \over \partial x_n}\right)^{(0)} \Delta x_n^{(0)}
= c_n.\cr
}
$$

Z~toho po zápise do maticového tvaru dostávame
$$
\hmatrix{
c_1 - \left(f_1^{(0)}\right)\cr
c_2 - \left(f_2^{(0)}\right)\cr
\vdots \cr
c_n - \left(f_3^{(0)}\right)\cr
}
=
\hmatrix{
\left({\partial f_1 \over \partial x_1}\right)^{(0)} &
\left({\partial f_1 \over \partial x_2}\right)^{(0)} &
\dots &
\left({\partial f_1 \over \partial x_n}\right)^{(0)} \cr
%
\left({\partial f_2 \over \partial x_1}\right)^{(0)} &
\left({\partial f_2 \over \partial x_2}\right)^{(0)} &
\dots &
\left({\partial f_2 \over \partial x_n}\right)^{(0)} \cr
\vdots & \vdots & \ddots & \vdots \cr
\left({\partial f_n \over \partial x_1}\right)^{(0)} &
\left({\partial f_n \over \partial x_2}\right)^{(0)} &
\dots &
\left({\partial f_n \over \partial x_n}\right)^{(0)} \cr
}
\cdot
\hmatrix{
\Delta x_1^{(0)} \cr
\Delta x_2^{(0)} \cr
\vdots \cr
\Delta x_n^{(0)} \cr
}
$$
a to môžme skrátene zapísať ako
$$
\Delta C^{(0)} = J^{(0)} \Delta X^{(0)}.
$$
Potom platí rovnosť
$$
\Delta X^{(0)} = [J^{(0)}]^{-1} \cdot \Delta C^{(0)}.
$$

Algoritmus metódy teda je 
$$
\displaylines{
\Delta C^{(k)} =
\hmatrix{
c_1 - \left(f_1^{(k)}\right) \cr
c_2 - \left(f_2^{(k)}\right) \cr
\vdots \cr
c_n - \left(f_n^{(k)}\right) \cr
}\cr
\Delta X^{(k)} = \left[J^{(k)}\right]^{-1} \cdot \Delta C^{(k)}\cr
X^{(k+1)} = X^{(k)} + \Delta X^{(k)}\cr
\Delta C^{(k+1)} =
\hmatrix{
c_1 - \left(f_1^{(k+1)}\right) \cr
c_2 - \left(f_2^{(k+1)}\right) \cr
\vdots \cr
c_n - \left(f_n^{(k+1)}\right) \cr
}
}
$$
kde
$$
\Delta X^{(k)} =
\left[
\Delta x_1^{(k)}, \Delta x_2^{(k)}, \dots, \Delta x_n^{(k)}
\right].
$$

Matica $J^{(k)}$ je Jakobiho matica, s~predpokladom regulárnosti, a
platí pre ňu vzťah
$$
J^{(k)} = \hmatrix{
\left({\partial f_1 \over \partial x_1}\right)^{(k)} &
\left({\partial f_1 \over \partial x_2}\right)^{(k)} &
\dots &
\left({\partial f_1 \over \partial x_n}\right)^{(k)} \cr
%
\left({\partial f_2 \over \partial x_1}\right)^{(k)} &
\left({\partial f_2 \over \partial x_2}\right)^{(k)} &
\dots &
\left({\partial f_2 \over \partial x_n}\right)^{(k)} \cr
\vdots & \vdots & \ddots & \vdots \cr
\left({\partial f_n \over \partial x_1}\right)^{(k)} &
\left({\partial f_n \over \partial x_2}\right)^{(k)} &
\dots &
\left({\partial f_n \over \partial x_n}\right)^{(k)} \cr
}.
$$


\subsection Riešenie výkonových tokov (load flow)

Pretože v~energetike sa častejšie požaduje znalosť výkonu a napätia, než
prúdu a napätia, môžme systém rovníc upraviť tak, že dostaneme vzťah
$$
\faz S_k = 3 \faz S_{k\rm f} = 3 \faz U_{k\rm f} I_k^* =
\sqrt{3} U_{k\rm f}
\sum_{m \in {\rm M}_kk} \sqrt3 \faz Y_{km}^* U_{m\rm f}^*
$$
a po úprave
$$
\faz S_k = \faz U_k \sum_{m \in {\rm M}_kk} \sqrt{3} \faz Y_{km}^* U_m^*.
$$

V~maticovom zápise platia pre výkon fázový $\faz S_{k\rm f}$ a trojfázový
$\faz S_k$ v~uzle $k$ rovnice
\zvyrazni
$$
\eqalign{
[\faz S_f] = [\faz U_{\rm f\,diag}^*] [\faz I]
&\qquad
[\faz S_f] = [\faz U_{\rm f\,diag}^*] [\faz Y] [\faz U_{\rm f}]\cr
[\faz S] = [\faz U_{\rm diag}^*] [\faz I] \sqrt{3}
&\qquad
[\faz S] = [\faz U_{\rm diag}^*] [\faz Y] [\faz U]\cr
}
$$

Poslednú rovnicu môžme rozpísať ako
$$
\hmatrix{ \faz S_1 \cr \vdots \cr \faz S_k \cr \vdots \cr \faz S_n \cr }
=
\hmatrix{
\faz U_1^* & \dots & 0 & \dots & 0 & \dots & 0 & \dots & 0 \cr
\vdots &&\vdots &&\vdots &&\vdots &&\vdots \cr
0 & \dots & 0 & \dots & \faz U_k^* & \dots & 0 & \dots & 0 \cr
\vdots &&\vdots &&\vdots &&\vdots &&\vdots \cr
0 & \dots & 0 & \dots & 0 & \dots & 0 & \dots & \faz U_n^* \cr
}
\cdot
\hmatrix{
\faz Y_{11} & \dots & \faz Y_{1k} & \dots & \faz Y_{1n} \cr
\vdots & \ddots & \vdots & &\vdots \cr
\faz Y_{11} & \dots & \faz Y_{1k} & \dots & \faz Y_{1n} \cr
\vdots &  & \vdots & \ddots&\vdots \cr
\faz Y_{11} & \dots & \faz Y_{1k} & \dots & \faz Y_{1n} \cr
}
\cdot
\hmatrix{
\faz U_1 \cr \vdots \cr \faz U_k \cr \vdots \cr \faz U_n \cr
}
$$

\item{--} zadané výkony $\to$ nelinearita

Ciel: určenie $\faz P, \faz Q, \faz U, \faz \delta$ v~uzloch a vetvách

\subsection * Jednotlivé typy uzlov

\tablelabel[label=typy uzlov][caption=Rôzne možnosti zadania veličín
v~uzloch] 
\centertable{
\begintable
\begintableformat
\center " \center " \center " \center " \center
\endtableformat
\-
\br{\:|} | \use{2} výkon v~uzle | \use{2} zložky fázoru napätia v~uzle \er{|}
\-
\br{\:|}  Kategória uzla | zadaný | má sa určiť | zadaný | má sa určiť \er{|}
\-
\br{\:|} slack $U\delta$ | --- | $P$, $Q$ | $U$, $\vartheta$ | --- \er{|}
\-
\br{\:|} $PQ$ | $P$, $Q$ | --- | --- | $U$, $\vartheta$ \er{|}
\-
\br{\:|} $PU$ | $P$ | $Q$ | $U$ | $\vartheta$ \er{|}
\-
\br{\:|} $Q\delta$ | $Q$ | $P$ | $\vartheta$ | $U$ \er{|}
\-
\endtable
}

Slack predstavuje bilančný uzol. K~tomuto uzlu sa vzťahujú fázové posuny
napätí všetkých ostatných uzlov. Spravidla obsahuje zdroj s~veľkým
výkonom, môže tvoriť mohutnú sústavu. Dorovnáva činný aj jalový výkon.

Uzly $PQ$ predstavujú záťaže a elektrárne s~výkonom určeným podľa
harmonogramu.

Uzly $PU$ predstavujú uzly s~odberom a s~kompenzátorom, slúžiacim na
udržanie veľkosti napätia.

Veličiny rozdeľujeme na\
\beginitemize
	\item pevné --- požiadavky ($P$, $Q$ záťaže, $P$ generátora),
	\item stavové --- nezávisle premenné ($U$, $\delta$ záťaže, $\delta$
	generátora),
	\item riadiace --- tu nemenné ($U$ bilančného uzla a generátorov),
	menia sa pri optimalizáciách.
\enditemize

Pre uzlový prúd a uzlový výkon platia vzťahy
$$
\displaylines{
I_i = U_i \sum_{j=0}^n y_{ij} - \sum_{j=1}^n y_{ij} U_j \quad j \ne i\cr
P_i + \j Q_i = U_i I_i^*\cr
I_i = {P_i - \j Q_i \over U_i^*}\cr
}
$$
čo po úprave je možné zapísať ako
$$
{P_i - \j Qi \over U_i^*} =
U_i \sum_{j=0}^n y_{ij} - \sum_{j=1}^n y_{ij} U_j \quad j \ne i
$$

\subsection * Gauss-Seidel Power Flow Solution

FIXME

\subsection * Newton-Raphson Power Flow Solution

FIXME


